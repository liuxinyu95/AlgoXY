\ifx\wholebook\relax \else
% ------------------------

\documentclass[UTF8]{article}
%------------------- Other types of document example ------------------------
%
%\documentclass[twocolumn]{IEEEtran-new}
%\documentclass[12pt,twoside,draft]{IEEEtran}
%\documentstyle[9pt,twocolumn,technote,twoside]{IEEEtran}
%
%-----------------------------------------------------------------------------
%
% loading packages
%

\RequirePackage{ifpdf}
\RequirePackage{ifxetex}

%
%
\ifpdf
  \RequirePackage[pdftex,%
       bookmarksnumbered,%
              colorlinks,%
          linkcolor=blue,%
              hyperindex,%
        plainpages=false,%
       pdfstartview=FitH]{hyperref}
\else\ifxetex
  \RequirePackage[bookmarksnumbered,%
               colorlinks,%
           linkcolor=blue,%
               hyperindex,%
         plainpages=false,%
        pdfstartview=FitH]{hyperref}
\else
  \RequirePackage[dvipdfm,%
        bookmarksnumbered,%
               colorlinks,%
           linkcolor=blue,%
               hyperindex,%
         plainpages=false,%
        pdfstartview=FitH]{hyperref}
\fi\fi
%\usepackage{hyperref}

% other packages
%--------------------------------------------------------------------------
\usepackage{graphicx, color}
\usepackage{subfig}
\usepackage{tikz}
\usetikzlibrary{matrix,positioning}

\usepackage{amsmath, amsthm, amssymb} % for math
\usepackage{exercise} % for exercise
\usepackage{import} % for nested input

%
% for programming
%
\usepackage{verbatim}
\usepackage{listings}
%\usepackage{algorithmic} %old version; we can use algorithmicx instead
%\usepackage[plain]{algorithm} %remove rule (horizontal line on top/below the algorithm
\usepackage{algorithm} %to remove rules change to \usepackage[plain]{algorithm}
%\usepackage{algorithm2e}
\usepackage[noend]{algpseudocode} %for pseudo code, include algorithmicsx automatically
\usepackage{appendix}
\usepackage{makeidx} % for index support
\usepackage{titlesec}

\usepackage[cm-default]{fontspec}
\usepackage{xunicode}
%\usepackage{fontenc}
\usepackage{textcomp}

% detect and select Chinese font
% ------------------------------
% the following cmd can list all availabe Chinese fonts in host.
% fc-list :lang=zh
\def\myfont{STSong}  % Under Mac OS X
\def\linuxfallback{WenQuanYi Micro Hei} % Under Linux
\def\winfallback{SimSun} % Under Windows
\suppressfontnotfounderror1 % Avoid setting exit code (error level) to break make process
\count255=\interactionmode
\batchmode
\font\foo="\myfont"\space at 10pt
\ifx\foo\nullfont
  \font\foo = "\linuxfallback"\space at 10pt
  \ifx\foo\nullfont
    \font\foo = "\winfallback"\space at 10pt
    \ifx\foo\nullfont
      \errorstopmode
      \errmessage{no suitable Chinese font found}
    \else
      \let\myfont=\winfallback % Windows
    \fi
  \else
    \let\myfont=\linuxfallback % Linux
  \fi
\fi
\interactionmode=\count255
\setmainfont[Mapping=tex-text]{\myfont}
\setmonofont{Monaco}   % 英文等宽字体

\XeTeXlinebreaklocale "zh"  % to solve the line breaking issue
\XeTeXlinebreakskip = 0pt plus 1pt minus 0.1pt

\titleformat{\paragraph}
{\normalfont\normalsize\bfseries}{\theparagraph}{1em}{}
\titlespacing*{\paragraph}
{0pt}{3.25ex plus 1ex minus .2ex}{1.5ex plus .2ex}

\lstdefinelanguage{Smalltalk}{
  morekeywords={self,super,true,false,nil,thisContext}, % This is overkill
  morestring=[d]',
  morecomment=[s]{"}{"},
  alsoletter={\#:},
  escapechar={!},
  literate=
    {BANG}{!}1
    {UNDERSCORE}{\_}1
    {\\st}{Smalltalk}9 % convenience -- in case \st occurs in code
    % {'}{{\textquotesingle}}1 % replaced by upquote=true in \lstset
    {_}{{$\leftarrow$}}1
    {>>>}{{\sep}}1
    {^}{{$\uparrow$}}1
    {~}{{$\sim$}}1
    {-}{{\sf -\hspace{-0.13em}-}}1  % the goal is to make - the same width as +
    %{+}{\raisebox{0.08ex}{+}}1		% and to raise + off the baseline to match -
    {-->}{{\quad$\longrightarrow$\quad}}3
	, % Don't forget the comma at the end!
  tabsize=2
}[keywords,comments,strings]

% for literate Haskell code
\lstdefinestyle{Haskell}{
  flexiblecolumns=false,
  basewidth={0.5em,0.45em},
  morecomment=[l]--,
  literate={+}{{$+$}}1 {/}{{$/$}}1 {*}{{$*$}}1 {=}{{$=$}}1
           {>}{{$>$}}1 {<}{{$<$}}1 {\\}{{$\lambda$}}1
           {\\\\}{{\char`\\\char`\\}}1
           {->}{{$\rightarrow$}}2 {>=}{{$\geq$}}2 {<-}{{$\leftarrow$}}2
           {<=}{{$\leq$}}2 {=>}{{$\Rightarrow$}}2
           {\ .}{{$\circ$}}2 {\ .\ }{{$\circ$}}2
           {>>}{{>>}}2 {>>=}{{>>=}}2
           {|}{{$\mid$}}1
}

\lstloadlanguages{C, C++, Lisp, Haskell, Python, Smalltalk}

\lstset{
  basicstyle=\small\ttfamily,
  commentstyle=\rmfamily,
  texcl=true,
  showstringspaces = false,
  upquote=true,
  flexiblecolumns=false
}

% ======================================================================

\def\BibTeX{{\rm B\kern-.05em{\sc i\kern-.025em b}\kern-.08em
    T\kern-.1667em\lower.7ex\hbox{E}\kern-.125emX}}

%
% mathematics
%
\newcommand{\be}{\begin{equation}}
\newcommand{\ee}{\end{equation}}
\newcommand{\bmat}[1]{\left( \begin{array}{#1} }
\newcommand{\emat}{\end{array} \right) }
\newcommand{\VEC}[1]{\mbox{\boldmath $#1$}}

% numbered equation array
\newcommand{\bea}{\begin{eqnarray}}
\newcommand{\eea}{\end{eqnarray}}

% equation array not numbered
\newcommand{\bean}{\begin{eqnarray*}}
\newcommand{\eean}{\end{eqnarray*}}

\newtheorem{theorem}{Theorem}[section]
\newtheorem{lemma}[theorem]{引理}
\newtheorem{proposition}[theorem]{Proposition}
\newtheorem{corollary}[theorem]{Corollary}

% 中文书籍设置
% ====================================
\renewcommand\contentsname{目\ 录}
%\renewcommand\listfigurename{插图目录}
%\renewcommand\listtablename{表格目录}
\renewcommand\figurename{图}
\renewcommand\tablename{表}
\renewcommand\proofname{证明}
\renewcommand\ExerciseName{练习}
%\renewcommand{\algorithmcfname}{算法}

\ifx\wholebook\relax
\renewcommand\bibname{参\ 考\ 文\ 献}                    %book类型
%\newtheorem{Definition}[Theorem]{定义}
\newtheorem{Theorem}{定理}[chapter]
\newtheorem{example}{例题}[chapter]
\else
\renewcommand\refname{参\ 考\ 文\ 献}
\fi

%\setcounter{secnumdepth}{4}
\titleformat{\chapter}
  {\normalfont\bfseries\Large}
  {第\arabic{chapter}章}
  {12pt}{\Large}
%% \titleformat{\subsection}
%%   {\normalfont\bfseries\large}
%%   {\CJKnumber{\arabic{subsection}}、}
%%   {12pt}{\large}
%% \titleformat{\subsubsection}
%%   {\normalfont\bfseries\normalsize}
%%   {\arabic{subsubsection}.}
%%   {12pt}{\normalsize}

%\renewcommand{\baselinestretch}{1.5}                        %文章行间距为1.5倍。

\setcounter{tocdepth}{4}
\setcounter{secnumdepth}{4}


\setcounter{page}{1}

\begin{document}

%--------------------------

% ================================================================
%                 COVER PAGE
% ================================================================

\title{并不复杂的红黑树}

\author{刘新宇
\thanks{{\bfseries 刘新宇} \newline
  Email: liuxinyu95@gmail.com \newline}
  }

\maketitle
\fi

\markboth{红黑树}{初等算法}

\ifx\wholebook\relax
\chapter{并不复杂的红黑树}
\numberwithin{Exercise}{chapter}
\fi

% ================================================================
%                 Introduction
% ================================================================
\section{简介}
\label{introduction} \index{红黑树}

\subsection{二叉搜索树的不足}
上一章中,我们给出了一个例子:统计一篇文章中每个单词出现的次数。思路是把二叉搜索树当作一个字典来计数。

一个自然的想法是使用二叉搜索树来处理电话黄页\footnote{一种公共发布的电话号码簿,因用黄色纸张印刷所以称为黄页。},并用它来查询某联系人的电话。

可以把上一章中的例子代码稍做改动来实现这一功能:

\begin{lstlisting}
int main(int, char** ) {
  ifstream f("yp.txt");
  map<string, string> dict;
  string name, phone;
  while(f>>name && f>>phone)
    dict[name]=phone;
  for(;;) {
    cout<<"\nname: ";
    cin>>name;
    if(dict.find(name) == dict.end())
      cout<<"not found";
    else
      cout<<"phone: "<<dict[name];
  }
}
\end{lstlisting}

这段程序运行良好。但是,如果我们把STL库提供的map换成普通的二叉搜索树,程序的性能就变差了,尤其是搜索诸如Zara、Zed、Zulu等姓名时特别明显。

这是因为电话黄页通常是按照字典字母顺序(lexicographic order)印刷的。因此姓名实际上是按照升序排列的。如果我们依次把数字1, 2, 3, ..., $n$插入二叉搜索树,就会得到一棵如图\ref{fig:unbalanced-tree}所示的树。

\begin{figure}[htbp]
    \centering
	\includegraphics[scale=0.5]{img/unbalanced.ps}
    \caption{不平衡的树} \label{fig:unbalanced-tree}
\end{figure}

这是一棵极不平衡的二叉树。对于高度为$h$的二叉搜索树,查找算法的复杂度为$O(h)$。如果树比较平衡,我们就能够达到$O(\lg n)$的性能。但在这一极端情况下,查找的性能退化为$O(n)$。几乎和普通的链表一样。

\begin{Exercise}

\begin{itemize}
\item 对于巨大的电话黄页列表,为了加快速度,一个想法是使用两个并发的任务(task,可以是线程或进程)。一个任务从头部向后读姓名――电话信息,另外一个任务从尾部向前读。当两个任务在中间相遇时程序结束。这样构建出的二叉搜索树是什么样子的?如果我们把黄页列表分割成更多片断,使用更多的任务会得到什么结果?
\item 参考图\ref{fig:unbalanced-trees}中的一些不平衡树,你能找到更多的情况造成二叉搜索树表现不佳么?
\end{itemize}

\end{Exercise}

\begin{figure}[htbp]
       \centering
       \subfloat[]{\includegraphics[scale=0.4]{img/unbalanced-2.ps}}
       \subfloat[]{\includegraphics[scale=0.4]{img/unbalanced-zigzag.ps}} \\
       \subfloat[]{\includegraphics[scale=0.4]{img/unbalanced-3.ps}}
       \caption{一些不平衡的二叉树}
       \label{fig:unbalanced-trees}
\end{figure}

\subsection{如何保证树的平衡}

为了避免出现上面描述的不平衡情况,我们可以预先使用随机算法将输入序列打乱(参见\cite{CLRS}的12.4节)。但是有时无法使用这个方法,例如序列是由用户交互输入的,每当用户输入一个key后,树就会被更新。也就是说,树是随着用户输入逐步构建的。

人们已经发现了好几种方法用于解决二叉搜索树的平衡问题。由很多方法依赖于二叉树的旋转(rotation)操作。旋转操作可以在保持元素顺序的情况下,改变树的结构。因此可以用来提高二叉搜索树的平衡性。

本章介绍红黑树――一种被广泛使用的自平衡二叉搜索树(self-adjusting balanced binary search tree)。下一章我们介绍另外一种自平衡树――AVL树。在后面介绍二叉堆的章节中,我们还会遇到另外一种有趣的树――splay树,它能够随着操作,逐渐使得树变得越来越平衡。

\subsection{树的旋转}
\index{树的旋转}

\begin{figure}[htbp]
   \centering
   \setlength{\unitlength}{1cm}
   \begin{picture}(10, 4)
   \put(5, 2){$\Longleftrightarrow$}
   \subfloat[]{\includegraphics[scale=0.4]{img/rotate-r.ps}}
   \subfloat[]{\includegraphics[scale=0.4]{img/rotate-l.ps}}
   \end{picture}
   \\
   \begin{picture}(1, 0.5)\end{picture} %pad
   \caption{树的旋转,左旋操作将左侧图中的树变换为右侧的树;右旋操作是左旋的逆变换。}
   \label{fig:tree-rotation}
\end{figure}

树的旋转是一种特殊的操作,它在保持中序遍历结果不变的情况下,改变树的结构。这是因为存在多个不同的二叉搜索树对应到一个特定的中序遍历顺序。图\ref{fig:tree-rotation}描述了旋转操作。图中左侧的二叉搜索树,经过左旋可以变换为右侧的树,而右旋是左旋的逆变换。

旋转操作可以用过程式的方法来描述。通过使用模式匹配(pattern matching)我们也可以用简单的函数来定义它们。将非空的二叉树记为三元组$T=(T_l, k, T_r)$,下面的函数定义了左右旋转。

\be
rotateL(T) = \left \{
  \begin{array}
  {r@{\quad:\quad}l}
  ((a, X, b), Y, c) & T = (a, X, (b, Y, c)) \\
  T & otherwise
  \end{array}
\right .
\ee

\be
rotateR(T) = \left \{
  \begin{array}
  {r@{\quad:\quad}l}
  (a, X, (b, Y, c)) & T = ((a, X, b), Y, c)) \\
  T & otherwise
  \end{array}
\right .
\ee

用伪代码描述时,除了左右分支和key,还需要设置好父节点。

\begin{algorithmic}[1]
\Function{Left-Rotate}{$T, x$}
  \State $p \gets$ \Call{Parent}{$x$}
  \State $y \gets$ \Call{Right}{$x$} \Comment{Assume $y \ne NIL$}
  \State $a \gets$ \Call{Left}{$x$}
  \State $b \gets$ \Call{Left}{$y$}
  \State $c \gets$ \Call{Right}{$y$}
  \State \Call{Replace}{$x, y$}
  \State \Call{Set-Children}{$x, a, b$}
  \State \Call{Set-Children}{$y, x, c$}
  \If{$p = NIL$}
    \State $T \gets y$
  \EndIf
  \State \Return $T$
\EndFunction

\Statex

\Function{Right-Rotate}{$T, y$}
  \State $p \gets$ \Call{Parent}{$y$}
  \State $x \gets$ \Call{Left}{$y$} \Comment{Assume $x \ne NIL$}
  \State $a \gets$ \Call{Left}{$x$}
  \State $b \gets$ \Call{Right}{$x$}
  \State $c \gets$ \Call{Right}{$y$}
  \State \Call{Replace}{$y, x$}
  \State \Call{Set-Children}{$y, b, c$}
  \State \Call{Set-Children}{$x, a, y$}
  \If{$p = NIL$}
    \State $T \gets x$
  \EndIf
  \State \Return $T$
\EndFunction

\Statex

\Function{Set-Left}{$x, y$}
  \State \Call{Left}{$x$} $\gets y$
  \If{$y \ne NIL$}
    \Call{Parent}{$y$} $\gets x$
  \EndIf
\EndFunction

\Statex

\Function{Set-Right}{$x, y$}
  \State \Call{Right}{$x$} $\gets y$
  \If{$y \ne NIL$}
    \Call{Parent}{$y$} $\gets x$
  \EndIf
\EndFunction

\Statex

\Function{Set-Children}{$x, L, R$}
  \State \Call{Set-Left}{$x, L$}
  \State \Call{Set-Right}{$x, R$}
\EndFunction

\Statex

\Function{Replace}{$x, y$}
  \If{\Call{Parent}{$x$} $= NIL$}
    \If{$y \ne NIL$}
      \Call{Parent}{$y$} $\gets NIL$
    \EndIf
  \ElsIf{\textproc{Left}(\Call{Parent}{$x$}) $= x$}
    \State \textproc{Set-Left}(\Call{Parent}{$x$}, $y$)
  \Else
    \State \textproc{Set-Right}(\Call{Parent}{$x$}, $y$)
  \EndIf
  \State \Call{Parent}{$x$} $\gets NIL$
\EndFunction
\end{algorithmic}

对比伪代码和使用模式匹配的函数,后者主要从树结构变化的角度出发,而前者集中于描述变换的过程。如本章题目所讲,红黑树并不像看起来那样复杂。很多教科书使用传统的过程式方法来讲解红黑树,需要处理许多不同的情况。每种情况都必须仔细处理节点种的数据。如果换成函数式的思路,虽然会牺牲一些性能,但很多问题会变得简单直观。

本章中的大部份内容来自Chris Okasaki的成果\cite{okasaki}。

% ================================================================
% Definition
% ================================================================
\section{红黑树的定义}
\index{红黑树}

红黑树式一种自平衡二叉搜索树\cite{wiki}\footnote{红黑树是2-3-4树的等价形式(有关2-3-4树,可参考B树一章)。也就是说,对于任一2-3-4树,都存在至少一个红黑树与之有着相同顺序的元素。}。通过对节点进行着色和旋转,红黑树可以简单直观地保持树的平衡。

我们可以在普通的二叉搜索树上增加一个颜色信息。一个节点可以被涂成红色或黑色。如果一棵二叉搜索树满足下面的全部5条性质,我们称之为红黑树\cite{CLRS}。
\index{红黑树!红黑性质}

\begin{enumerate}
\item 任一节点要么是红色,要么是黑色。
\item 根节点为黑色。
\item 所有的叶节点(NIL节点)为黑色。
\item 如果一个节点为红色,则它的两个子节点都是黑色。
\item 对任一节点,从它出发到所有叶子节点的路径上包含相同数量的黑色节点。
\end{enumerate}

为什么这5条性质能保证红黑树的平衡性呢?因为它们有一个关键的特性:从根节点出发到达叶节点的所有路径中,最长路径不会超过最短路径长度的两倍。

注意到第四条性质,它意味着不存在两个连续的红色节点。因此,最短的路径只含有黑色的节点,任何比最短路径长的路径上都分散有一些红色节点。根据性质五,从根节点出发的所有的路径都含有相同数量的黑色节点,这就最终保证了没有任何路径超过最短路径长度的两倍\cite{wiki}。图\ref{fig:rbt-example}的例子展示了一棵红黑树。

\begin{figure}[htbp]
  \centering
  \includegraphics[scale=0.5]{img/rbt-example.ps}
  \caption{红黑树} \label{fig:rbt-example}
\end{figure}

红黑树沿用所有二叉搜索树中不改变树结构的操作,包括查找、获取最大、最小值等。只有插入和删除操作是特殊的。

如前面的单词统计的例子所示,有很多集合(set)和map容器是使用红黑树来实现的。包括C++标准库中STL\cite{sgi-stl}。

由于只增加了一个颜色信息,我们可以复用二叉搜索树的节点定义。如下面的C++代码所示:

\lstset{language=C++}
\begin{lstlisting}
enum Color {Red, Black};

template <class T>
struct node{
  Color color;
  T key;
  node* left;
  node* right;
  node* parent;
};
\end{lstlisting}

对于函数式环境,我们可以在代数数据类型(algebraic data type)的定义中增加颜色参数,如下面的Haskell例子所示:

\lstset{language=Haskell}
\begin{lstlisting}
data Color = R | B
data RBTree a = Empty
              | Node Color (RBTree a) a (RBTree a)
\end{lstlisting}

\begin{Exercise}

\begin{itemize}
\item 利用红黑树的性质证明$n$个节点的红黑树的高度不会超过$2 \lg (n+1)$。
\end{itemize}

\end{Exercise}

% ================================================================
%                 Insertion
% ================================================================
\section{插入}
\index{红黑树!插入}

由于插入操作会改变二叉搜索树的结构,树有可能因此变得不平衡。为了保持红黑树的性质,我们需要在插入操作后进行变换来修复平衡问题。

当插入一个key时,可以总是把新节点染成红色。只要它不是根节点,除了第四条外的所有红黑树性质都可以满足。唯一的问题就是可能引入两个相邻的红色节点。

函数式和命令式实现使用不同的方法来修复平衡。前者直观简单,但是存在一点性能损失;后者有些复杂,但是具有更高的性能。大多数的算法书籍介绍后一种方法。本章中,我们关注函数式的方法,并展示这一方法极为简洁的特性。我们也会给出传统的命令式实现以作为对比。

Chris Okasaki指出,共有四种情况会违反红黑树的第四条性质。它们都带有两个相邻的红色节点。而且,它们可以被统一修复为一个形式\cite{okasaki},如图 \ref{fig:insert-fix}所示。

\begin{figure}[htbp]
  \centering
     \setlength{\unitlength}{1cm}
     \begin{picture}(15, 15)
        % arrows
        \put(4.5, 9.5){\vector(1, -1){1}}
        \put(4.5, 5){\vector(1, 1){1}}
        \put(10, 9.5){\vector(-1, -1){1}}
        \put(10, 5){\vector(-1, 1){1}}
        % graphics
	\put(0, 7){\includegraphics[scale=0.5]{img/insert-ll.ps}}
        \put(0, 0){\includegraphics[scale=0.5]{img/insert-lr.ps}}
        \put(7, 7){\includegraphics[scale=0.5]{img/insert-rr.ps}}
        \put(8.5, 0){\includegraphics[scale=0.5]{img/insert-rl.ps}}
        \put(2, 5){\includegraphics[scale=0.5]{img/insert-fixed.ps}}
      \end{picture}
     \caption{插入后需要修复的四种情况} \label{fig:insert-fix}
\end{figure}

注意到这一变换会把红色向上“移动”一层。因此它是一种自底向上的递归修复,最后一步会把根节点染成红色。根据红黑树的第二条性质,根节点必须是黑色的。因此我们最后要把根节点再染回黑色。为了方便,我们把节点的颜色记为$\mathcal{C}$,它有两个值,黑色$\mathcal{B}$和红色$\mathcal{R}$。这样一棵非空的红黑树可以表达为一个四元组$T=(\mathcal{C}, T_l, k, T_r)$。

\be
balance(T) = \left \{
  \begin{array}
  {r@{\quad:\quad}l}
  (\mathcal{R}, (\mathcal{B}, A, x, B), y, (\mathcal{B}, C, z, D)) & match(T) \\
  T & otherwise
  \end{array}
\right .
\ee

其中,函数$match(T)$用以判断树是否符合图\ref{fig:insert-fix}中的四种需要修复的情况。定义如下:

\[
match(T) = \left \{ T = \begin{array}{l}
         (\mathcal{B}, (\mathcal{R}, (\mathcal{R}, A, x, B), y, C), z, D) \lor \\
         (\mathcal{B}, (\mathcal{R}, A, x, (\mathcal{R}, B, y, C), z, D)) \lor \\
         (\mathcal{B}, A, x, (\mathcal{R}, B, y, (\mathcal{R}, C, z, D))) \lor \\
         (\mathcal{B}, A, x, (\mathcal{R}, (\mathcal{R}, B, y, C), z, D)) \\
         \end{array} \right \}
\]

定义好函数$balance(T)$后,我们就可以修改普通二叉搜索树的插入函数,使其支持红黑树的插入。

\be
insert(T, k) = makeBlack(ins(T, k))
\ee

其中$ins(T, k)$函数定义如下:

\be
ins(T, k) = \left \{
  \begin{array}
  {r@{\quad:\quad}l}
  (\mathcal{R}, \phi, k, \phi) & T = \phi \\
  balance((ins(T_l, k), k', T_r)) & k < k' \\
  balance((T_l, k', ins(T_r, k))) & otherwise
  \end{array}
\right.
\ee

如果待插入的树为空,则创建一个新的红色节点,节点的key就是待插入的$k$;否则,记树的左右分支和key分别为$T_l$、$T_r$和$k'$,我们比较$k$和$k'$的大小,递归地将它插入的分支中,然后再用$balance$函数恢复平衡。最后,我们使用$makeBlack(T)$函数把根节点染成黑色。

\be
makeBlack(T) = (\mathcal{B}, T_l, k, T_r)
\ee

在支持模式匹配的语言中,例如Haskell,插入算法可以实现为下面的程序:

\lstset{language=Haskell}
\begin{lstlisting}
insert t x = makeBlack $ ins t where
    ins Empty = Node R Empty x Empty
    ins (Node color l k r)
        | x < k     = balance color (ins l) k r
        | otherwise = balance color l k (ins r) --[3]
    makeBlack(Node _ l k r) = Node B l k r

balance B (Node R (Node R a x b) y c) z d =
                Node R (Node B a x b) y (Node B c z d)
balance B (Node R a x (Node R b y c)) z d =
                Node R (Node B a x b) y (Node B c z d)
balance B a x (Node R b y (Node R c z d)) =
                Node R (Node B a x b) y (Node B c z d)
balance B a x (Node R (Node R b y c) z d) =
                Node R (Node B a x b) y (Node B c z d)
balance color l k r = Node color l k r
\end{lstlisting} %$

程序中的\texttt{balance}函数略微有些该动,它的参数不是一棵树,而是节点的颜色、左侧分支、key和右侧分支。这样可以节省一对boxing和unboxing的操作。

例子程序没有处理试图插入重复key的情况。实际中,我们可以选择覆盖,也可以跳过不做处理,还可以选择在节点中用一个链表来来存储重复的数据\cite{CLRS}。

图\ref{fig:insert-example}中给出了两个红黑树的例子。左侧的是按照顺序将11, 2, 14, 1, 7, 5, 8, 4这些值插入的结果。右侧的是将序列1, 2, ..., 8插入的结果。可以看到,即使输入已序序列,产生的红黑树仍然保持了平衡。

\begin{figure}[htbp]
  \centering
  \includegraphics[scale=0.5]{img/insert-haskell.ps}
  \caption{根据不同的序列,使用插入算法产生的两棵红黑树} \label{fig:insert-example}
\end{figure}

通过将四种不同的不平衡树统一变换成同一种形式,我们得到了一个极为简洁的算法。即使在不支持模式匹配的语言中,我们仍然可以用编程检验树是否满足一定的模式。这样得到的结果和传统方法相比,仍然要简洁很多。读者可以参考本章附带的Lisp方言Scheme的例子代码。

对于含有$n$个节点的红黑树,这一插入算法的复杂度为$O(\lg n)$。

\begin{Exercise}

\begin{itemize}
\item 使用一种命令式语言,例如C、C++或Python来实现本节介绍的插入算法。注意:如果语言本身不支持模式匹配,需要编程检查四种不同的情况。
\end{itemize}

\end{Exercise}

% ================================================================
%                 Deletion
% ================================================================

\section{删除}
\index{红黑树!删除}

在上一章我们曾指出,二叉搜索树的删除操作只在imperative的环境下才有意义。这一结论同样适用于红黑树。大多数情况下,通常一次性将树构建好,然后频繁地进行查找。Okasaki曾经解释过为什么他没有给出红黑树的删除算法\cite{okasaki-blog}。其中一个原因就是删除比插入要麻烦得多。

我们希望通过本节的介绍,使读者了解到,在纯函数式的环境中,删除算法是可以实现的。纯函数式数据结构决定了树不会真的被改变,我们实际上是重建了一棵树\footnote{大多数函数式编程环境通过使用persistent,可以复用树中没有改变的部份,从而减小重建的开销。}。实际应用中,往往是由用户(也就是编程者)来决定使用那种具体的方案。例如,我们可以不做任何操作,而仅仅给每个要删除的节点做一个标记。当带有删除标记的节点超过50\%的时候,用所有未标记的节点重建一棵树。

不仅是函数式环境,imperative的删除算法也比插入算法要复杂。这主要是因为在删除时,我们面临更多的情况需要修复平衡。删除也会破坏红黑树的性质,因此需要后继的处理以恢复平衡。

本节介绍的删除算法主要来自\cite{lyn}。只有在删除一个黑色的节点时才会引发问题。因为这样会破坏红黑树的第五条性质,使得某一路径上的黑色节点数目少于其他的路径。

在删除一个黑色节点时,我们可以通过引入“双重黑色”\cite{CLRS}的概念来恢复第五条红黑树性质。也就是说,虽然节点被删除了,我们把它的黑色保存在它的父节点中。如果父节点是红色的,我们将其变为黑色;但如果父节点已经是黑色的,它就会变成一个“双重黑色”的节点。

为了表达双重黑色的节点,我们需要修改一下红黑树的定义,如下面的Haskell示例代码:

\lstset{language=Haskell}
\begin{lstlisting}
data Color = R | B | BB -- BB: doubly black for deletion
data RBTree a = Empty | BBEmpty -- doubly black empty
              | Node Color (RBTree a) a (RBTree a)
\end{lstlisting}

删除一个节点时,我们先调用普通二叉搜索树的删除算法。如果被删除节点时黑色的,我们接下来进行修复以保持红黑树的第五条性质。删除函数定义如下:

\be
delete(T, k) = blackenRoot(del(T, k))
\ee

其中

\be
del(T, k) = \left \{
  \begin{array}
  {r@{\quad:\quad}l}
  \phi & T = \phi \\
  fixBlack^2((\mathcal{C}, del(T_l, k), k', T_r)) & k < k' \\
  fixBlack^2((\mathcal{C}, T_l, k', del(T_r, k))) & k > k' \\
  \left \{
    \begin{array}{r@{\quad:\quad}l}
    mkBlk(T_r) & \mathcal{C} = \mathcal{B} \\
    T_r & otherwise
    \end{array}
  \right. & T_l = \phi \\
  \left \{
    \begin{array}{r@{\quad:\quad}l}
    mkBlk(T_l) & \mathcal{C} = \mathcal{B} \\
    T_l & otherwise
    \end{array}
  \right.  & T_r = \phi \\
  fixBlack^2((\mathcal{C}, T_l, k'', del(T_r, k''))) & otherwise
  \end{array}
\right.
\ee

具体的删除,主要由函数$del$来实现。如果树为空,这种边界情况下的删除结果也为空$\phi$;否则,如果待删除的key比当前节点的key小,我们递归地在左侧分支进行删除;如果比当前节点的key大,则递归地在右侧分支删除。由于可能引入双重黑色,所以需要后继的修复处理。

如果待删除的key恰好等于当前节点的key,我们需要将其“切下”。若当前节点有一个子分支为空,我们只要用另外一个分支替换当前节点,并且保持当前节点的黑色属性就可以了。否则,如果两个分支都不为空,我们从右侧子分支中找到最小值$k''=min(T_r)$,将其“切下”并替换掉当前的节点的key。

最后,函数$delete$通过调用$blackenRoot$强制将$del$返回结果的根节点染成黑色。

\be
blackenRoot(T) = \left \{
  \begin{array}
  {r@{\quad:\quad}l}
  \phi & T = \phi \\
  (\mathcal{B}, T_l, k, T_r) & otherwise \\
  \end{array}
\right .
\ee

和红黑树插入算法中的$makeBlack$函数相比,$blackenRoot$仅仅多了对树为空的情况的处理。这一处理仅在删除时才需要考虑,因为插入操作不可能产生一棵空树,而删除操作是有可能的。

函数$mkBlk$用于保持被“切下”节点的黑色属性。如果被“切下”的节点为黑色,它将红色的结果改为黑色,把黑色的结果改为双重黑色。如果结果为空,则将其标记为双重黑色的空,记为$\Phi$。

\be
mkBlk(T) = \left \{
  \begin{array}
  {r@{\quad:\quad}l}
  \Phi & T = \phi \\
  (\mathcal{B}, T_l, k, T_r) & \mathcal{C} = \mathcal{R} \\
  (\mathcal{B}^2, T_l, k, T_r) & \mathcal{C} = \mathcal{B} \\
  T & otherwise
  \end{array}
\right .
\ee

其中,符号$\mathcal{B}^2$表示双重黑色。

将目前的函数定义翻译为Haskell可以得到如下例子程序。

\begin{lstlisting}
delete t x = blackenRoot(del t x) where
    del Empty _ = Empty
    del (Node color l k r) x
        | x < k = fixDB color (del l x) k r
        | x > k = fixDB color l k (del r x)
        -- x == k, delete this node
        | isEmpty l = if color==B then makeBlack r else r
        | isEmpty r = if color==B then makeBlack l else l
        | otherwise = fixDB color l k' (del r k') where k'= min r
    blackenRoot (Node _ l k r) = Node B l k r
    blackenRoot _ = Empty

makeBlack (Node B l k r) = Node BB l k r -- doubly black
makeBlack (Node _ l k r) = Node B l k r
makeBlack Empty = BBEmpty
makeBlack t = t
\end{lstlisting}

删除算法中,唯一还没有定义的函数是$fixBlack^2$。我们需要在这个函数中,通过树的旋转操作和重新染色,最终去掉“双重黑色”。

我们先从双重黑色的空节点开始。对于任何节点,如果它的一个子节点为双重黑色的空节点,而另外的一个子分支不为空,我们可以安全地用一个普通的空节点代替双重黑色的空节点。

如图\ref{fig:db-fix-1-nil}所示,如果删除节点4(图中只显示了树的一部份),我们会用一个双重黑色的空节点代替它。图中带有两个圈的节点,表示双重黑色节点。于是对于节点5,它的左侧子节点是双重黑色的空节点,而右侧子分支不为空(一个key为6的叶子节点)。这种情况下,我们可以安全用普通空节点替换双重黑色,而不会破坏任何红黑树的性质。

\begin{figure}[htbp]
   \centering
   \subfloat[从树中删除4。]{\includegraphics[scale=0.4]{img/db-fix.ps}} \\
   \subfloat[4被“切下”后,变成了一个双重黑色的空节点。]{\includegraphics[scale=0.4]{img/db-fix-1-nil-before.ps}}
   \subfloat[我们可以安全地将其变为普通的NIL。]{\includegraphics[scale=0.4]{img/db-fix-1-nil-after.ps}}
   \caption{一个子节点为双重黑色空节点,另外一个子分支不为空。} \label{fig:db-fix-1-nil}
\end{figure}

On the other hand, if a node has a doubly-black empty node and the other child is
empty, we have to push the doubly-blackness up one level. For example in figure
\ref{fig:db-fix-2-nil}, if we want
to delete node 1 from the tree, the program will use a doubly-black empty node
to replace 1. Then node 2 has a doubly-black empty node and has an empty right
node. In such case we must mark node 2 as doubly-black after change its
left child back to empty.

\begin{figure}[htbp]
   \centering
   \subfloat[Delete 1 from the tree.]{\includegraphics[scale=0.4]{img/db-fix.ps}} \\
   \subfloat[After 1 is sliced off, it is doubly-black empty.]{\includegraphics[scale=0.4]{img/db-fix-2-nil-before.ps}}
   \subfloat[We must push the doubly-blackness up to node 2.]{\includegraphics[scale=0.4]{img/db-fix-2-nil-after.ps}}
   \caption{One child is doubly-black empty node, the other child is empty.} \label{fig:db-fix-2-nil}
\end{figure}

Based on above analysis, in order to fix the doubly-black empty node, we define
the function partially like the following.

\be
fixBlack^2(T) = \left \{
  \begin{array}
  {r@{\quad:\quad}l}
  node(\mathcal{B}^2, \phi, Key, \phi) & (L = \phi \land R = \Phi) \lor (L = \Phi \land R = \phi) \\
  node(\mathcal{C}, \phi, Key, R) & L=\Phi \land R \neq \phi \\
  node(\mathcal{C}, L, Key, \phi) & R=\Phi \land L \neq \phi \\
  ... & ...
  \end{array}
\right .
\label{eq:db-nil}
\ee

After dealing with doubly-black empty node, we need to fix the case that the
sibling of the doubly-black node is black and it has one red child.
In this situation, we can fix the doubly-blackness with one rotation.
Actually there are 4 different sub-cases, all of them can be transformed
to one uniformed pattern. They are shown in the figure \ref{fig:del-case1}.
These cases are described in \cite{CLRS} as case 3 and case 4.

\begin{figure}[htbp]
   \centering
   \includegraphics[scale=0.4]{img/del-case1.eps}
   \caption{Fix the doubly black by rotation, the sibling of the doubly-black node is black, and it has one red child.} \label{fig:del-case1}
\end{figure}

The handling of these 4 sub-cases can be defined on top of formula \ref{eq:db-nil}.

\be
fixBlack^2(T) = \left \{
  \begin{array}
  {r@{\quad:\quad}l}
  ... & ... \\
  node(\mathcal{C}, node(\mathcal{B}, mkBlk(A), x, B), y, node(\mathcal{B}, C, z, D)) & p 1.1 \\
  node(\mathcal{C}, node(\mathcal{B}, A, x, B), y, node(\mathcal{B}, C, z, mkBlk(D))) & p 1.2 \\
  ... & ...
  \end{array}
\right .
\label{eq:db-case-1}
\ee

where $p 1.1$ and $p 1.2$ each represent 2 patterns as the following.

\[
p 1.1 = \left \{ \begin{array}{l}
  node(\mathcal{C}, A, x, node(\mathcal{B}, node(\mathcal{R}, B, y, C), z, D)) \land Color(A) = \mathcal{B}^2 \\
  \lor \\
  node(\mathcal{C}, A, x, node(\mathcal{B}, B, y, node(\mathcal{R}, C, z, D))) \land Color(A) = \mathcal{B}^2
  \end{array} \right \}
\]

\[
p 1.2 = \left \{ \begin{array}{l}
  node(\mathcal{C}, node(\mathcal{B}, A, x, node(\mathcal{R}, B, y, C)), z, D) \land Color(D) = \mathcal{B}^2 \\
  \lor \\
  node(\mathcal{C}, node(\mathcal{B}, node(\mathcal{R}, A, x, B), y, C), z, D) \land Color(D) = \mathcal{B}^2
  \end{array} \right \}
\]

Besides the above cases, there is another one that not only the sibling
of the doubly-black node is black, but also its two children are black.
We can change the color of the sibling node to red; resume the
doubly-black node to black and propagate the doubly-blackness one level
up to the parent node as shown in figure \ref{fig:del-case2}. Note that
there are two symmetric sub-cases. This case is described in \cite{CLRS}
as case 2.

\begin{figure}[htbp]
  \centering
  \setlength{\unitlength}{1cm}
  \begin{picture}(10, 4)
  \put(5, 2){$\Longrightarrow$}
  \subfloat[Color of $x$ can be either black or red.]{\includegraphics[scale=0.4]{img/case2-a.ps}}
  \subfloat[If $x$ was red, then it becomes black, otherwise, it becomes doubly-black.]{\includegraphics[scale=0.4]{img/case2-a1.ps}}
  \end{picture}
  \\
  \begin{picture}(10, 5)
  \put(5, 2){$\Longrightarrow$}
  \subfloat[Color of $y$ can be either black or red.]{\includegraphics[scale=0.4]{img/case2-b.ps}}
  \subfloat[If $y$ was red, then it becomes black, otherwise, it becomes doubly-black.]{\includegraphics[scale=0.4]{img/case2-b1.ps}}
  \end{picture}
  \\
  \begin{picture}(1, 0.5)\end{picture} %pad
  \caption{propagate the blackness up.} \label{fig:del-case2}
\end{figure}

We go on adding this fixing after formula \ref{eq:db-case-1}.

\be
fixBlack^2(T) = \left \{
  \begin{array}
  {r@{\quad:\quad}l}
  ... & ... \\
  mkBlk(node(\mathcal{C}, mkBlk(A), x, node(\mathcal{R}, B, y, C))) & p 2.1 \\
  mkBlk(node(\mathcal{C}, node(\mathcal{R}, A, x, B), y, mkBlk(C))) & p 2.2 \\
  ... & ...
  \end{array}
\right .
\label{eq:db-case-2}
\ee

where $p 2.1$ and $p 2.2$ are two patterns as below.

\[
p 2.1 = \left \{ \begin{array}{l}
  node(\mathcal{C}, A, x, node(\mathcal{B}, B, y, C)) \land \\
  Color(A) = \mathcal{B}^2 \land Color(B) = Color(C) = \mathcal{B}
  \end{array} \right \}
\]

\[
p 2.2 = \left \{ \begin{array}{l}
  node(\mathcal{C}, node(\mathcal{B}, A, x, B), y, C) \land \\
  Color(C) = \mathcal{B}^2 \land Color(A) = Color(B) = \mathcal{B}
  \end{array} \right \}
\]

There is a final case left, that the sibling of the doubly-black node is red.
We can do a rotation to change this case to pattern $p 1.1$ or $p 1.2$.
Figure \ref{fig:del-case3} shows about it.

\begin{figure}[htbp]
  \centering
  \includegraphics[scale=0.4]{img/del-case3.eps}
  \caption{The sibling of the doubly-black node is red.} \label{fig:del-case3}
\end{figure}

We can finish formula \ref{eq:db-case-2} with \ref{eq:db-case-3}.

\be
fixBlack^2(T) = \left \{
  \begin{array}
  {r@{\quad:\quad}l}
  ... & ... \\
  fixBlack^2(\mathcal{B}, fixBlack^2(node(\mathcal{R}, A, x, B), y, C) & p 3.1 \\
  fixBlack^2(\mathcal{B}, A, x, fixBlack^2(node(\mathcal{R}, B, y, C)) & p 3.2 \\
  T & otherwise
  \end{array}
\right .
\label{eq:db-case-3}
\ee

where $p 3.1$ and $p 3.2$ are two patterns as the following.

\[
p 3.1 = \{ Color(T) = \mathcal{B} \land Color(L) = \mathcal{B}^2 \land Color(R) = \mathcal{R} \}
\]

\[
p 3.2 = \{ Color(T) = \mathcal{B} \land Color(L) = \mathcal{R} \land Color(R) = \mathcal{B}^2 \}
\]


This two cases are described in \cite{CLRS} as case 1.

Fixing the doubly-black node with all above different cases is a recursive function.
There are two termination conditions. One contains pattern $p 1.1$ and $p 1.2$,
The doubly-black node was eliminated. The other cases may continuously propagate the
doubly-blackness from bottom to top till the root.
Finally the algorithm marks the root node as black anyway. The doubly-blackness will be
removed.

Put formula \ref{eq:db-nil}, \ref{eq:db-case-1}, \ref{eq:db-case-2}, and \ref{eq:db-case-3}
together, we can write the final Haskell program.

\begin{lstlisting}
fixDB::Color -> RBTree a -> a -> RBTree a -> RBTree a
fixDB color BBEmpty k Empty = Node BB Empty k Empty
fixDB color BBEmpty k r = Node color Empty k r
fixDB color Empty k BBEmpty = Node BB Empty k Empty
fixDB color l k BBEmpty = Node color l k Empty
-- the sibling is black, and it has one red child
fixDB color a@(Node BB _ _ _) x (Node B (Node R b y c) z d) =
      Node color (Node B (makeBlack a) x b) y (Node B c z d)
fixDB color a@(Node BB _ _ _) x (Node B b y (Node R c z d)) =
      Node color (Node B (makeBlack a) x b) y (Node B c z d)
fixDB color (Node B a x (Node R b y c)) z d@(Node BB _ _ _) =
      Node color (Node B a x b) y (Node B c z (makeBlack d))
fixDB color (Node B (Node R a x b) y c) z d@(Node BB _ _ _) =
      Node color (Node B a x b) y (Node B c z (makeBlack d))
-- the sibling and its 2 children are all black, propagate the blackness up
fixDB color a@(Node BB _ _ _) x (Node B b@(Node B _ _ _) y c@(Node B _ _ _))
    = makeBlack (Node color (makeBlack a) x (Node R b y c))
fixDB color (Node B a@(Node B _ _ _) x b@(Node B _ _ _)) y c@(Node BB _ _ _)
    = makeBlack (Node color (Node R a x b) y (makeBlack c))
-- the sibling is red
fixDB B a@(Node BB _ _ _) x (Node R b y c) = fixDB B (fixDB R a x b) y c
fixDB B (Node R a x b) y c@(Node BB _ _ _) = fixDB B a x (fixDB R b y c)
-- otherwise
fixDB color l k r = Node color l k r
\end{lstlisting}

The deletion algorithm takes $O(\lg N)$ time to delete a key from
a red-black tree with $N$ nodes.

\begin{Exercise}

\begin{itemize}
\item As we mentioned in this section, deletion can be implemented
by just marking the node as deleted without actually removing it.
Once the number of marked nodes exceeds 50\% of the total node
number, a tree re-build is performed. Try to implement this
method in your favorite programming language.
\item 为什么不需要在$mkBlk$的调用处,显示地再调用$fixBlack^2$?
\end{itemize}

\end{Exercise}

\section{Imperative red-black tree algorithm $\star$}
\index{red-black tree!imperative insertion}

We almost finished all the content in this chapter. By induction
the patterns, we can implement the red-black tree in a simple way
compare to the imperative tree rotation solution. However, we
should show the comparator for completeness.

For insertion, the basic idea is to use the similar algorithm
as described in binary search tree. And then fix the balance
problem by rotation and return the final result.

\begin{algorithmic}[1]
\Function{Insert}{$T, k$}
  \State $root \gets T$
  \State $x \gets$ \Call{Create-Leaf}{$k$}
  \State \Call{Color}{$x$} $\gets RED$
  \State $parent \gets NIL$
  \While{$T \neq NIL$}
    \State $parent \gets T$
    \If{$k <$ \Call{Key}{$T$}}
      \State $T \gets $ \Call{Left}{$T$}
    \Else
      \State $T \gets $ \Call{Right}{$T$}
    \EndIf
  \EndWhile
  \State \Call{Parent}{$x$} $\gets parent$
  \If{$parent = NIL$} \Comment{tree $T$ is empty}
    \State \Return $x$
  \ElsIf{$k <$ \Call{Key}{$parent$}}
    \State \Call{Left}{$parent$} $\gets x$
  \Else
    \State \Call{Right}{$parent$} $\gets x$
  \EndIf
  \State \Return \Call{Insert-Fix}{$root, x$}
\EndFunction
\end{algorithmic}

The only difference from the binary search tree insertion algorithm
is that we set the color of the new node as red, and perform fixing
before return. It is easy to translate the pseudo code to real
imperative programming language, for instance Python \footnote{C,
and C++ source codes are available along with this book}.

\lstset{language=Python}
\begin{lstlisting}
def rb_insert(t, key):
    root = t
    x = Node(key)
    parent = None
    while(t):
        parent = t
        if(key < t.key):
            t = t.left
        else:
            t = t.right
    if parent is None: #tree is empty
        root = x
    elif key < parent.key:
        parent.set_left(x)
    else:
        parent.set_right(x)
    return rb_insert_fix(root, x)
\end{lstlisting}

There are 3 base cases for fixing, and if we take the left-right
symmetric into consideration. there are total 6 cases.
Among them two cases can be merged together, because they all have
uncle node in red color, we can toggle the parent color and
uncle color to black and set grand parent color to red.
With this merging, the fixing algorithm can be realized as the following.

\begin{algorithmic}[1]
\Function{Insert-Fix}{$T, x$}
  \While{\Call{Parent}{$x$} $\neq NIL$ and \textproc{Color}(\Call{Parent}{$x$}) $= RED$}
    \If{\textproc{Color}(\Call{Uncle}{$x$}) $= RED$}
      \Comment{Case 1, x's uncle is red}
      \State \textproc{Color}(\Call{Parent}{$x$}) $\gets BLACK$
      \State \textproc{Color}(\Call{Grand-Parent}{$x$}) $\gets RED$
      \State \textproc{Color}(\Call{Uncle}{$x$}) $\gets BLACK$
      \State $x \gets$ \Call{Grandparent}{$x$}
    \Else
      \Comment{x's uncle is black}
      \If{\Call{Parent}{$x$} = \textproc{Left}(\Call{Grand-Parent}{$x$})}
        \If{ $x =$ \textproc{Right}(\Call{Parent}{$x$})}
          \Comment{Case 2, x is a right child}
          \State $x \gets$ \Call{Parent}{$x$}
          \State $T \gets$ \Call{Left-Rotate}{$T, x$}
        \EndIf
        \Comment{Case 3, x is a left child}
        \State \textproc{Color}(\Call{Parent}{$x$}) $\gets BLACK$
        \State \textproc{Color}(\Call{Grand-Parent}{$x$}) $\gets RED$
        \State $T \gets$ \textproc{Right-Rotate}($T$, \Call{Grand-Parent}{$x$})
      \Else
         \If{ $x =$ \textproc{Left}(\Call{Parent}{$x$})}
          \Comment{Case 2, Symmetric}
          \State $x \gets$ \Call{Parent}{$x$}
          \State $T \gets$ \Call{Right-Rotate}{$T, x$}
        \EndIf
        \Comment{Case 3, Symmetric}
        \State \textproc{Color}(\Call{Parent}{$x$}) $\gets BLACK$
        \State \textproc{Color}(\Call{Grand-Parent}{$x$}) $\gets RED$
        \State $T \gets$ \textproc{Left-Rotate}($T$, \Call{Grand-Parent}{$x$})
      \EndIf
    \EndIf
  \EndWhile
  \State \Call{Color}{$T$} $\gets BLACK$
  \State \Return $T$
\EndFunction
\end{algorithmic}

This program takes $O(\lg N)$ time to insert a new key to the red-black tree.
Compare this pseudo code and the $balance$ function we defined in previous
section, we can see the difference. They differ not only in terms of
simplicity, but also in logic. Even if we feed the same series of keys to
the two algorithms, they may build different red-black trees. There
is a bit performance overhead in the pattern matching algorithm.
Okasaki discussed about the difference in detail in his paper\cite{okasaki}.

Translate the above algorithm to Python yields the below program.

\begin{lstlisting}
# Fix the red->red violation
def rb_insert_fix(t, x):
    while(x.parent and x.parent.color==RED):
        if x.uncle().color == RED:
            #case 1: ((a:R x:R b) y:B c:R) ==> ((a:R x:B b) y:R c:B)
            set_color([x.parent, x.grandparent(), x.uncle()],
                      [BLACK, RED, BLACK])
            x = x.grandparent()
        else:
            if x.parent == x.grandparent().left:
                if x == x.parent.right:
                    #case 2: ((a x:R b:R) y:B c) ==> case 3
                    x = x.parent
                    t=left_rotate(t, x)
                # case 3: ((a:R x:R b) y:B c) ==> (a:R x:B (b y:R c))
                set_color([x.parent, x.grandparent()], [BLACK, RED])
                t=right_rotate(t, x.grandparent())
            else:
                if x == x.parent.left:
                    #case 2': (a x:B (b:R y:R c)) ==> case 3'
                    x = x.parent
                    t = right_rotate(t, x)
                # case 3': (a x:B (b y:R c:R)) ==> ((a x:R b) y:B c:R)
                set_color([x.parent, x.grandparent()], [BLACK, RED])
                t=left_rotate(t, x.grandparent())
    t.color = BLACK
    return t
\end{lstlisting}

Figure \ref{fig:imperative-insert} shows the results of feeding same
series of keys to the above python insertion program. Compare them with
figure \ref{fig:insert-example}, one can tell the difference clearly.

\begin{figure}[htbp]
   \centering
   \subfloat[]{\includegraphics[scale=0.4]{img/clrs-fig-13-4.ps}}
   \subfloat[]{\includegraphics[scale=0.4]{img/python-insert.ps}}
   \caption{Red-black trees created by imperative algorithm.}
   \label{fig:imperative-insert}
\end{figure}

We skip the red-black tree delete algorithm in imperative settings, because it
is even more complex than the insertion. The implementation of deleting is
left as an exercise of this chapter.

\begin{Exercise}

\begin{itemize}
\item Implement the red-black tree deleting algorithm in your favorite
imperative programming language. you can refer to \cite{CLRS} for algorithm
details.
\end{itemize}

\end{Exercise}

\section{More words}
Red-black tree is the most popular implementation of balanced binary search
tree. Another one is the AVL tree, which we'll introduce in next chapter.
Red-black tree can be a good start point for more data structures. If we
extend the number of children from 2 to $K$, and keep the balance as well,
it leads to B-tree, If we store the data along with edge but not inside
node, it leads to Tries. However, the multiple cases handling and the long
program tends to make new comers think red-black tree is complex.

Okasaki's work helps making the red-black tree much easily understand.
There are many implementation in other programming languages in that
manner \cite{rosetta}. It's also inspired me to find the pattern matching
solution for Splay tree and AVL tree etc.

\begin{thebibliography}{99}

\bibitem{CLRS}
Thomas H. Cormen, Charles E. Leiserson, Ronald L. Rivest and Clifford Stein.
``Introduction to Algorithms, Second Edition''. ISBN:0262032937. The MIT Press. 2001

\bibitem{okasaki}
Chris Okasaki. ``FUNCTIONAL PEARLS Red-Black Trees in a Functional Setting''. J. Functional Programming. 1998

\bibitem{okasaki-blog}
Chris Okasaki. ``Ten Years of Purely Functional Data Structures''. http://okasaki.blogspot.com/2008/02/ten-years-of-purely-functional-data.html

\bibitem{wiki}
Wikipedia. ``Red-black tree''. http://en.wikipedia.org/wiki/Red-black\_tree

\bibitem{lyn}
Lyn Turbak. ``Red-Black Trees''. cs.wellesley.edu/~cs231/fall01/red-black.pdf Nov. 2, 2001.

\bibitem{sgi-stl}
SGI STL. http://www.sgi.com/tech/stl/

\bibitem{rosetta}
Pattern matching. http://rosettacode.org/wiki/Pattern\_matching

\end{thebibliography}

\ifx\wholebook\relax\else
\end{document}
\fi
