\ifx\wholebook\relax \else
% ------------------------

\documentclass[UTF8]{article}
%------------------- Other types of document example ------------------------
%
%\documentclass[twocolumn]{IEEEtran-new}
%\documentclass[12pt,twoside,draft]{IEEEtran}
%\documentstyle[9pt,twocolumn,technote,twoside]{IEEEtran}
%
%-----------------------------------------------------------------------------
%
% loading packages
%

\RequirePackage{ifpdf}
\RequirePackage{ifxetex}

%
%
\ifpdf
  \RequirePackage[pdftex,%
       bookmarksnumbered,%
              colorlinks,%
          linkcolor=blue,%
              hyperindex,%
        plainpages=false,%
       pdfstartview=FitH]{hyperref}
\else\ifxetex
  \RequirePackage[bookmarksnumbered,%
               colorlinks,%
           linkcolor=blue,%
               hyperindex,%
         plainpages=false,%
        pdfstartview=FitH]{hyperref}
\else
  \RequirePackage[dvipdfm,%
        bookmarksnumbered,%
               colorlinks,%
           linkcolor=blue,%
               hyperindex,%
         plainpages=false,%
        pdfstartview=FitH]{hyperref}
\fi\fi
%\usepackage{hyperref}

% other packages
%--------------------------------------------------------------------------
\usepackage{graphicx, color}
\usepackage{subfig}
\usepackage{tikz}
\usetikzlibrary{matrix,positioning}

\usepackage{amsmath, amsthm, amssymb} % for math
\usepackage{exercise} % for exercise
\usepackage{import} % for nested input

%
% for programming
%
\usepackage{verbatim}
\usepackage{listings}
%\usepackage{algorithmic} %old version; we can use algorithmicx instead
%\usepackage[plain]{algorithm} %remove rule (horizontal line on top/below the algorithm
\usepackage{algorithm} %to remove rules change to \usepackage[plain]{algorithm}
%\usepackage{algorithm2e}
\usepackage[noend]{algpseudocode} %for pseudo code, include algorithmicsx automatically
\usepackage{appendix}
\usepackage{makeidx} % for index support
\usepackage{titlesec}

\usepackage[cm-default]{fontspec}
\usepackage{xunicode}
%\usepackage{fontenc}
\usepackage{textcomp}

% detect and select Chinese font
% ------------------------------
% the following cmd can list all availabe Chinese fonts in host.
% fc-list :lang=zh
\def\myfont{STSong}  % Under Mac OS X
\def\linuxfallback{WenQuanYi Micro Hei} % Under Linux
\def\winfallback{SimSun} % Under Windows
\suppressfontnotfounderror1 % Avoid setting exit code (error level) to break make process
\count255=\interactionmode
\batchmode
\font\foo="\myfont"\space at 10pt
\ifx\foo\nullfont
  \font\foo = "\linuxfallback"\space at 10pt
  \ifx\foo\nullfont
    \font\foo = "\winfallback"\space at 10pt
    \ifx\foo\nullfont
      \errorstopmode
      \errmessage{no suitable Chinese font found}
    \else
      \let\myfont=\winfallback % Windows
    \fi
  \else
    \let\myfont=\linuxfallback % Linux
  \fi
\fi
\interactionmode=\count255
\setmainfont[Mapping=tex-text]{\myfont}
\setmonofont{Monaco}   % 英文等宽字体

\XeTeXlinebreaklocale "zh"  % to solve the line breaking issue
\XeTeXlinebreakskip = 0pt plus 1pt minus 0.1pt

\titleformat{\paragraph}
{\normalfont\normalsize\bfseries}{\theparagraph}{1em}{}
\titlespacing*{\paragraph}
{0pt}{3.25ex plus 1ex minus .2ex}{1.5ex plus .2ex}

\lstdefinelanguage{Smalltalk}{
  morekeywords={self,super,true,false,nil,thisContext}, % This is overkill
  morestring=[d]',
  morecomment=[s]{"}{"},
  alsoletter={\#:},
  escapechar={!},
  literate=
    {BANG}{!}1
    {UNDERSCORE}{\_}1
    {\\st}{Smalltalk}9 % convenience -- in case \st occurs in code
    % {'}{{\textquotesingle}}1 % replaced by upquote=true in \lstset
    {_}{{$\leftarrow$}}1
    {>>>}{{\sep}}1
    {^}{{$\uparrow$}}1
    {~}{{$\sim$}}1
    {-}{{\sf -\hspace{-0.13em}-}}1  % the goal is to make - the same width as +
    %{+}{\raisebox{0.08ex}{+}}1		% and to raise + off the baseline to match -
    {-->}{{\quad$\longrightarrow$\quad}}3
	, % Don't forget the comma at the end!
  tabsize=2
}[keywords,comments,strings]

% for literate Haskell code
\lstdefinestyle{Haskell}{
  flexiblecolumns=false,
  basewidth={0.5em,0.45em},
  morecomment=[l]--,
  literate={+}{{$+$}}1 {/}{{$/$}}1 {*}{{$*$}}1 {=}{{$=$}}1
           {>}{{$>$}}1 {<}{{$<$}}1 {\\}{{$\lambda$}}1
           {\\\\}{{\char`\\\char`\\}}1
           {->}{{$\rightarrow$}}2 {>=}{{$\geq$}}2 {<-}{{$\leftarrow$}}2
           {<=}{{$\leq$}}2 {=>}{{$\Rightarrow$}}2
           {\ .}{{$\circ$}}2 {\ .\ }{{$\circ$}}2
           {>>}{{>>}}2 {>>=}{{>>=}}2
           {|}{{$\mid$}}1
}

\lstloadlanguages{C, C++, Lisp, Haskell, Python, Smalltalk}

\lstset{
  basicstyle=\small\ttfamily,
  commentstyle=\rmfamily,
  texcl=true,
  showstringspaces = false,
  upquote=true,
  flexiblecolumns=false
}

% ======================================================================

\def\BibTeX{{\rm B\kern-.05em{\sc i\kern-.025em b}\kern-.08em
    T\kern-.1667em\lower.7ex\hbox{E}\kern-.125emX}}

%
% mathematics
%
\newcommand{\be}{\begin{equation}}
\newcommand{\ee}{\end{equation}}
\newcommand{\bmat}[1]{\left( \begin{array}{#1} }
\newcommand{\emat}{\end{array} \right) }
\newcommand{\VEC}[1]{\mbox{\boldmath $#1$}}

% numbered equation array
\newcommand{\bea}{\begin{eqnarray}}
\newcommand{\eea}{\end{eqnarray}}

% equation array not numbered
\newcommand{\bean}{\begin{eqnarray*}}
\newcommand{\eean}{\end{eqnarray*}}

\newtheorem{theorem}{Theorem}[section]
\newtheorem{lemma}[theorem]{引理}
\newtheorem{proposition}[theorem]{Proposition}
\newtheorem{corollary}[theorem]{Corollary}

% 中文书籍设置
% ====================================
\renewcommand\contentsname{目\ 录}
%\renewcommand\listfigurename{插图目录}
%\renewcommand\listtablename{表格目录}
\renewcommand\figurename{图}
\renewcommand\tablename{表}
\renewcommand\proofname{证明}
\renewcommand\ExerciseName{练习}
%\renewcommand{\algorithmcfname}{算法}

\ifx\wholebook\relax
\renewcommand\bibname{参\ 考\ 文\ 献}                    %book类型
%\newtheorem{Definition}[Theorem]{定义}
\newtheorem{Theorem}{定理}[chapter]
\newtheorem{example}{例题}[chapter]
\else
\renewcommand\refname{参\ 考\ 文\ 献}
\fi

%\setcounter{secnumdepth}{4}
\titleformat{\chapter}
  {\normalfont\bfseries\Large}
  {第\arabic{chapter}章}
  {12pt}{\Large}
%% \titleformat{\subsection}
%%   {\normalfont\bfseries\large}
%%   {\CJKnumber{\arabic{subsection}}、}
%%   {12pt}{\large}
%% \titleformat{\subsubsection}
%%   {\normalfont\bfseries\normalsize}
%%   {\arabic{subsubsection}.}
%%   {12pt}{\normalsize}

%\renewcommand{\baselinestretch}{1.5}                        %文章行间距为1.5倍。

\setcounter{tocdepth}{4}
\setcounter{secnumdepth}{4}


\setcounter{page}{1}

\begin{document}

%--------------------------

% ================================================================
%                 COVER PAGE
% ================================================================

\title{B树}

\author{刘新宇
\thanks{{\bfseries 刘新宇 } \newline
  Email: liuxinyu95@gmail.com \newline}
  }

\maketitle
\fi

\markboth{B-树}{初等算法}

\ifx\wholebook\relax
\chapter{B树}
\numberwithin{Exercise}{chapter}
\fi

%{\bfseries Corresponding Author:} Larry LIU Xinyu

% ================================================================
%                 Introduction
% ================================================================
\section{简介}
\index{B-树}
\label{introduction}

B树是一个重要的数据结构。现代文件系统有许多是由B+树实现的。B+树是B树的一种扩展形式。B树还被广泛用于数据库的实现中。

B树在传统的教科书中通常被用来解释磁盘或者外部存储器中整块数据的存取\cite{CLRS}。我们也可以把平衡二叉树的概念进行抽象、推广,从而引出B树\cite{wiki-b-tree}。

图\ref{fig:btree-example}展示了一棵B树,我们可以观察它和二叉搜索树之间的异同。

\begin{figure}[htbp]
   \begin{center}
	\includegraphics[scale=0.5]{img/btree-example.ps}
   \caption{B树} \label{fig:btree-example}
   \end{center}
\end{figure}

回忆一下二叉搜索树的定义。一棵二叉搜索树
\begin{itemize}
\item 或者是一个空节点;
\item 或者包含三部份,一个键值,一棵左侧分支和一棵右侧分支。这两个子分支也都是二叉搜索树。
\end{itemize}

同时,二叉搜索树满足下面的限制条件:
\begin{itemize}
\item 任何左侧分支包含的键值都不大于节点的键值;
\item 节点的键值不大于任何右侧分支包含的键值。
\end{itemize}

对于非空的二叉树$(L, k, R)$,其中$L$、$R$和$k$分别代表左右子树和键值。若函数$Key(T)$可以获取树$T$的键值,这一限制条件可以表达为如下形式:

\begin{equation}
\forall x \in L, \forall y \in R \\
\Rightarrow Key(x) \leq k \leq Key(y)
\end{equation}

将这一定义进一步推广,使得它允许多个键值和子分支,就得到了B树的定义。

一棵B树
\begin{itemize}
\item 或者为空;
\item 或者包含$n$个键值和$n+1$棵子树,每可子树也是一棵B树。这些键值和子树分别记为$k_1, k_2, ..., k_n$和$c_1, c_2, ..., c_n, c_{n+1}$。
\end{itemize}

图\ref{fig:btree-node}描述了一个B树节点的样子。

\begin{figure}[htbp]
  \centering
	\includegraphics[scale=0.5]{img/btree-node.ps}
  \caption{一个B树节点} \label{fig:btree-node}
\end{figure}

节点的所有键值和子树都满足下面的限制条件:

\begin{itemize}
\item 所有的键值按照单调增(非递减)的顺序保存。即:$k_1 \leq k_2 \leq ... \leq k_n$;
\item 对于任意$k_i$,子树$c_i$中所有的元素都不大于$k_i$,且$k_i$不大于子树$c_{i+1}$的任意元素。
\end{itemize}

这一限制条件可以表达为下面的式(\ref{eq:btree-order})。

\begin{equation}
\forall x_i \in c_i, i=0, 1, ..., n, \Rightarrow x_1 \leq k_1 \leq
x_2 \leq k_2 \leq ... \leq x_n \leq k_n \leq x_{n+1}
\label{eq:btree-order}
\end{equation}

最后,为了保证平衡性,B树还满足一些额外的要求:

\begin{itemize}
\item 所有的叶子节点具有相同的深度;
\item 定义整数$t$,称为B树的\underline{最小度数}:
    \begin{itemize}
        \item 每个节点最多含有$2t-1$个键值;
        \item 除根节点外,每个节点最少含有$t-1$个键值。
    \end{itemize}
\end{itemize}

考虑一棵含有$n$个键值的B树。最小度数$t \geq 2$。树的高度为$h$。除根节点外的全部节点至少含有$t-1$个键值。因为根节点至少含有一个键值,所以至少由两个深度为1的子节点,至少有$2t$个深度为2的子节点,至少有$2t^2$个深度为3的子节点……最后,至少有$2t^{h-1}$个深度为$h$的叶子节点。除根节点外,将节点个树乘以$t-1$,一棵B树所存储的全部元素个数必然满足下面的不等式。

\be
\begin{array}{rl}
n & \geq 1 + (t-1)(2 + 2t + 2t^2 + ... + 2t^{h-1}) \\
  & = 1 + 2(t-1) \displaystyle \sum_{k=0}^{h-1} t^k \\
  & = 1 + 2(t-1) \displaystyle \frac{t^h-1}{t-1} \\
  & = 2t^h - 1
\end{array}
\ee

于是可以导出B树的高度和元素数满足下面的关系:

\be
h \leq \log_t \frac{n+1}{2}
\ee

这就证明了B树的平衡性。最简单的B树称为2-3-4树。它的最小度数$t=2$,除根节点外的任何节点都包含2到4个键值。任何一棵红黑树本质上都可以转换为一棵2-3-4树。

下面的Python例子代码给出了B树的定义。它根据传入的最小度数$t$创建一个节点:

\lstset{language=Python}
\begin{lstlisting}
class BTree:
    def __init__(self, t):
        self.t = t
        self.keys = []
        self.children = []
\end{lstlisting}

B树的节点通常还保存有额外的数据(卫星数据),为了简化问题,本书后面的介绍不考虑这些额外数据。

本章中,我们首先介绍如何通过插入操作构造B树。我们会介绍传统方法\cite{CLRS}:将过满的节点在插入前进行拆分;此外,我们会介绍另外一种和红黑树类似的插入方法,它使用先插入后调整的侧略\cite{okasaki-rbtree} \cite{wiki-b-tree}。然后我们会介绍B树的删除和查找算法。


% ================================================================
%                 Insertion
% ================================================================
\section{插入}
\index{B树!插入}
\label{btree-insertion}

我们可以通过不断插入key来构建B树。方法和二叉搜索树类似。当插入$x$时,从根节点开始,我们在节点中找到一个位置,这个位置左侧的所有key都小于$x$,而右侧的所有key都大于$x$\footnote{实际上,元素只要支持小于比较和等于比较。参见本章练习题。}。如果当前节点是叶子节点,并且没有满(节点中含有的key不足$2t-1$个),就可以将$x$插入到这个位置。否则,这一位置会指向一个子节点,我们需要递归向这一子节点插入$x$。

\begin{figure}[htbp]
  \centering
  \subfloat[将22插入2-3-4树:$22 > 20$,插入右子树;$22 < 26$,插入第一个子节点。]{\includegraphics[scale=0.5]{img/btree-insert-simple1.ps}} \\
  \subfloat[$21 < 22 < 25$,且叶子节点未满。]{\includegraphics[scale=0.5]{img/btree-insert-simple2.ps}}
  \caption{B树的插入和二叉搜索树相似。} \label{fig:btree-insert-simple}
\end{figure}

图\ref{fig:btree-insert-simple}描述了一个插入的例子。这里的B树为2-3-4树。当插入元素$x=22$时,由于它的比根节点保存的key大,所以接下来检查右侧节点中的26、38和45;因为$22 < 26$,所以接下来检查第一个子节点中的21和25。这是一个叶子节点,并且未满。因此22被插入到21和25中间。

但是,如果叶子节点中已经含有$2t-1$个key,它已经满了。我们就不能简单地将新key插入。对于图中的B树,插入18就会遇到这个问题。有两种方法可以解决它。

%=========================================================================
%       Splitting
%=========================================================================
\subsection{分拆}
\index{B树!分拆}
\label{split}

\subsubsection{插入前预分拆}

如果节点已满,我们可以在插入前预先对节点进行分拆。

一个含有$t-1$个key的节点可以按照图\ref{fig:node-split}所示分拆为3个部份。左侧的部份包括前$t-1$个key和$t$个子树;右侧的部份包括剩下的$t-1$个key和$t$个子树。左右两侧都是合法的B树。中间的部份时第$t$个key。我们可以把它向上推入到父节点中。如果当前节点时根节点,则第$t$个key和分拆出的两个较小的子树将组成一个新的根节点。

\begin{figure}[htbp]
  \centering
  \subfloat[分拆前]{\includegraphics[scale=0.5]{img/split-node-before.ps}} \\
  \subfloat[分拆后]{\includegraphics[scale=0.5]{img/split-node-after.ps}}
  \caption{分拆节点}
  \label{fig:node-split}
\end{figure}

给定节点$x$,记$K(x)$为key的列表,$C(x)$为全部子树的列表。第$i$个key为$k_i(x)$,第$j$个子树为$c_j(x)$。下面的算法描述了如何分拆节点node的第$i$个子树:

\begin{algorithmic}[1]
\Procedure{Split-Child}{$node, i$}
  \State $x \gets c_i(node)$
  \State $y \gets$ \Call{CREATE-NODE}{}
  \State \Call{Insert}{$K(node), i, k_t(x)$}
  \State \Call{Insert}{$C(node), i + 1, y$}
  \State $K(y) \gets \{k_{t+1}(x), k_{t+2}(x), ..., k_{2t-1}(x)\}$
  \State $K(x) \gets \{k_1(x), k_2(x), ..., k_{t-1}(x)\}$
  \If{$y$ is not leaf}
    \State $C(y) \gets \{c_{t+1}(x), c_{t+2}(x), ..., c_{2t}(x)\}$
    \State $C(x) \gets \{c_1(x), c_2(x), ..., c_t(x)\}$
  \EndIf
\EndProcedure
\end{algorithmic}

下面的Python例子程序实现了子树分拆算法。

\lstset{language=Python}
\begin{lstlisting}
def split_child(node, i):
    t = node.t
    x = node.children[i]
    y = BTree(t)
    node.keys.insert(i, x.keys[t-1])
    node.children.insert(i+1, y)
    y.keys = x.keys[t:]
    x.keys = x.keys[:t-1]
    if not is_leaf(x):
        y.children = x.children[t:]
        x.children = x.children[:t]
\end{lstlisting}

其中函数\texttt{is\_leaf}判断一个节点是否是叶子节点。

\lstset{language=Python}
\begin{lstlisting}
def is_leaf(t):
    return t.children == []
\end{lstlisting}

%=========================================================================
%       Split before insertion
%=========================================================================

分拆后,有一个key被向上推入父节点。而父节点很可能已经满了,因此推入后可能会不再满足B树的限制条件。

为了解决这一问题,我们可以从根节点开始,沿着插入的路线检查每一个节点。如果路径上的任何节点已经满了,我们就将其分拆。由于我们已经检查过此节点的父节点,因此该父节点所含有的key一定少于$2t-1$。向它推入一个key不会破坏B树的性质。这一方法只需要自顶向下处理一次而无需任何回溯。

如果根节点需要拆分,就会产生出一个新的根节点,它不含任何key,此前的根节点成为这个新节点的唯一的子节点。然后我们就可以按照上面的描述进行自顶向下地检查,并最终将新key插入。

\begin{algorithmic}[1]
\Function{Insert}{$T, k$}
  \State $r \gets T$
  \If{$r$ is full} \Comment{root is full}
    \State $s \gets$ \Call{CREATE-NODE}{}
    \State $C(s) \gets \{r\}$
    \State \Call{Split-Child}{$s, 1$}
    \State $r \gets s$
  \EndIf
  \State \Return \Call{Insert-Nonfull}{$r, k$}
\EndFunction
\end{algorithmic}

其中算法\textproc{Insert-Nonfull}假设传入的节点不满而不再做额外的检查。如果传入的节点为叶子节点,待插入的key就根据它的大小插入到节点的合适位置;否则,算法就寻找可插入的子节点。如果子节点已满,就进行拆分。

\begin{algorithmic}[1]
\Function{Insert-Nonfull}{$T, k$}
  \If{$T$ is leaf}
    \State $i \gets 1$
    \While{$i \leq |K(T)| \land k > k_i(T)$}
      \State $i \gets i+1$
    \EndWhile
    \State \Call{Insert}{$K(T), i, k$}
  \Else
    \State $i \gets |K(T)|$
    \While{$i>1 \land k < k_i(T)$}
      \State $i \gets i-1$
    \EndWhile
    \If{$c_i(T)$ is full}
      \State \Call{Split-Child}{$T, i$}
      \If{$k > k_i(T)$}
        \State $i \gets i+1$
      \EndIf
    \EndIf
    \State \Call{Insert-Nonfull}{$c_i(T), k$}
  \EndIf
  \State \Return $T$
\EndFunction
\end{algorithmic}

这一算法是递归的。B树的最小度数$t$通常根据磁盘结构来确定,即使很小的深度也能保存巨大数量的数据(例如$t=10$的时候,一棵深度为10的B树可以保存100亿数据)。在实际实现中,递归也可以被消除。这作为一道习题留给读者。

图\ref{fig:btree-insert}描述了依次向一个空树插入G, M, P, X, A, C, D, E, J, K, N, O, R, S, T, U, V, Y, Z的结果。第一个结果是一棵2-3-4树($t=2$),第二个结果中的最小度数$t=3$。我们可以看出两棵B树的异同。

\begin{figure}[htbp]
  \centering
  \subfloat[2-3-4 tree.]{\includegraphics[scale=0.5]{img/btree-insert-2-3-4.ps}}\\
  \subfloat[$t=3$]{\includegraphics[scale=0.5]{img/btree-insert-3.ps}}
  \caption{Insertion result} \label{fig:btree-insert}
\end{figure}

下面的Python例子程序实现了这一算法。

\lstset{language=Python}
\begin{lstlisting}
def insert(tr, key):
    root = tr
    if is_full(root):
        s = BTree(root.t)
        s.children.insert(0, root)
        split_child(s, 0)
        root = s
    return insert_nonfull(root, key)
\end{lstlisting}

其中向未满节点插入的实现如下所示:

\begin{lstlisting}
def insert_nonfull(tr, key):
    if is_leaf(tr):
        ordered_insert(tr.keys, key)
    else:
        i = len(tr.keys)
        while i>0 and key < tr.keys[i-1]:
            i = i-1
        if is_full(tr.children[i]):
            split_child(tr, i)
            if key>tr.keys[i]:
                i = i+1
        insert_nonfull(tr.children[i], key)
    return tr
\end{lstlisting}

这里,函数\texttt{ordered\_insert}用于将一个元素插入到已序列表中。函数\texttt{is\_full}用以检查一个节点是否含有$2t-1$个key。

\begin{lstlisting}
def ordered_insert(lst, x):
    i = len(lst)
    lst.append(x)
    while i>0 and lst[i]<lst[i-1]:
        (lst[i-1], lst[i]) = (lst[i], lst[i-1])
        i=i-1

def is_full(node):
    return len(node.keys) >= 2 * node.t - 1
\end{lstlisting}

如果容器基于数组实现,向末尾添加元素的效率要远高于向中间位置插入的效率。对于长度为$n$的数组,后者往往是线性时间$O(n)$的。函数\texttt{ordered\_insert}首先将新元素添加到当前容器的末尾,然后从最后一个元素向前检查相邻两个元素是否已序。如果大小颠倒,就进行交换操作。

% ================================================================
%               Insert and fix method
% ================================================================

\subsubsection{先插入再修复}

我们也可以利用类似红黑树插入的方法来实现纯函数式的B树插入算法。当向红黑树插入时,首先按照普通的二叉搜索树将新key插入,然后递归地进行修复以恢复平衡性。B树可以看做二叉搜索树的扩展,每个节点含有多个key和子树。插入时,我们可以暂时不考虑节点是否已满,将新key插入后,再进行修复以满足最小度数的限制条件。

\be
insert(T, k) = fix(ins(T, k))
\ee

函数$ins(T, k)$从根节点开始遍历B树,找到合适的未知将$k$插入。此后在应用函数$fix$来回复B树的性质。记B树为$T = (K, C, t)$,其中$K$代表全部的key,$C$代表子树,$t$代表最小度数。

下面的Haskell例子代码定义了B树。

\lstset{language=Haskell}
\begin{lstlisting}
data BTree a = Node{ keys :: [a]
                   , children :: [BTree a]
                   , degree :: Int} deriving (Eq)
\end{lstlisting}

根据这一B树的定义,我们可以给出如下的Haskell插入函数

\lstset{language=Haskell}
\begin{lstlisting}
insert tr x = fixRoot $ ins tr x
\end{lstlisting} %$

实现函数$ins(T, k)$时,我们要处理两种不同情况:如果$T$是叶子节点,$k$就直接被插入到节点中;否则$T$为分支节点,我们需要递归地将$k$插入到某个子节点中。

图\ref{fig:recursive-insert}给出了分支节点的情况。算法首先定位到插入位置。对于某个key $k_i$,若待插入的key $k$满足$k_{i-1}<k<k_i$,就需要递归将$k$插入到子分支$c_i$中。

待插入位置将节点分成了三个部分:左侧部分、子分支$c_i$、和右侧部分。

\begin{figure}[htbp]
  \centering
  \subfloat[定位到待插入的子分支。]{\includegraphics[scale=0.5]{img/insert-before.ps}} \\
  \subfloat[递归插入。]{\includegraphics[scale=0.5]{img/insert-after.ps}}
  \caption{向分支节点插入key。} \label{fig:recursive-insert}
\end{figure}

\be
ins(T, k) = \left \{
  \begin{array}
  {r@{\quad:\quad}l}
  (K' \cup \{k\} \cup K'', \Phi, t) & C = \Phi, (K', K'') = divide(K, k) \\
  make((K', C_1), ins(c, k), (K'', C_2')) & (C_1, C_2) = split(|K'|, C)
  \end{array}
\right.
\ee

上式中的第一行处理叶子节点的情况。函数$divide(K, k)$将所有的key分成两部分,第一部分中的key都不大于$k$,第二部分中剩余的key都不小于$k$:

\[
K = K' \cup K'' \land \forall k' \in K, k'' \in K'' \Rightarrow k' \leq k \leq k''
\]

第二行处理分支节点的情况。函数$split(n, C)$将所有的子树分成$C_1$和$C_2$两部分。其中$C_1$包含了前$n$棵子树;而$C_2$包含剩余的子树。$C_2$中的第一棵子树记为$c$,其余子树记为$C_2'$。

此后,我们需要将$k$递归地插入到子树$c$中。函数$make$接受3个参数:其中第一个和第三个分别是一对key和子树的列表;第二个参数是一棵子树。它检查用传入的key和子树构造的B树节点是否会违反最小度数限制,如果违反,就进行适当的修复。

\be
make((K', C'), c, (K'', C'')) = \left \{
  \begin{array}
  {r@{\quad:\quad}l}
  fixFull((K', C'), c, (K'', C'')) & full(c) \\
  (K' \cup K'', C' \cup \{c\} \cup C'', t) & otherwise
  \end{array}
\right.
\ee

其中函数$full(c)$检查节点$c$是否已满。如果满,函数$fixFull$将节点$c$进行分拆,并且用分拆后推上来的key来构造一个新的B树节点。

\be
fixFull((K', C'), c, (K'', C'')) = (K' \cup \{k'\} \cup K'', C' \cup \{c_1, c_2\} \cup C'', t)
\ee

这里$(c_1, k', c_2) = split(c)$。在分拆中,前$t-1$个key和前$t$个子树被抽出构造一个新节点,后$t-1$个key和后$t$个子树被用于构造另一个新节点;第$t$个key $k'$被向上推入到key中。

使用上述定义的函数,我们可以最终实现$fix(T)$以完成函数式的B树插入算法。它首先检查根节点是否含有过多的key,如果超过限制,就进行分拆。分拆的结果被用于构造一个新节点,因此树的高度会增加1。

\be
fix(T) = \left \{
  \begin{array}
  {r@{\quad:\quad}l}
  c & T = (\Phi, \{c\}, t) \\
  (\{k'\}, \{c_1, c_2\}, t) & full(T), (c_1, k', c_2) = split(T) \\
  T & otherwise
  \end{array}
\right.
\ee

下面的Haskell例子程序实现了B树的插入算法。

\lstset{language=Haskell}
\begin{lstlisting}
import qualified Data.List as L

ins (Node ks [] t) x = Node (L.insert x ks) [] t
ins (Node ks cs t) x = make (ks', cs') (ins c x) (ks'', cs'')
    where
      (ks', ks'') = L.partition (<x) ks
      (cs', (c:cs'')) = L.splitAt (length ks') cs

fixRoot (Node [] [tr] _) = tr -- shrink height
fixRoot tr = if full tr then Node [k] [c1, c2] (degree tr)
             else tr
    where
      (c1, k, c2) = split tr

make (ks', cs') c (ks'', cs'')
    | full c = fixFull (ks', cs') c (ks'', cs'')
    | otherwise = Node (ks'++ks'') (cs'++[c]++cs'') (degree c)

fixFull (ks', cs') c (ks'', cs'') = Node (ks'++[k]++ks'')
                                         (cs'++[c1,c2]++cs'') (degree c)
    where
      (c1, k, c2) = split c

full tr = (length $ keys tr) > 2*(degree tr)-1
\end{lstlisting}

图\ref{fig:btree-insert-fp}给出了不断向空树中插入“GMPXACDEJKNORSTUVYZ”的两个不同结果。

\begin{figure}[htbp]
  \centering
  \subfloat[2-3-4树的插入结果。]{\includegraphics[scale=0.5]{img/btree-insert-fp-234.ps}} \\
  \subfloat[最小度数$t = 3$的B树插入结果。]{\includegraphics[scale=0.5]{img/btree-insert-fp-3.ps}}
    \caption{线插入再修复的结果。} \label{fig:btree-insert-fp}
\end{figure}

和图\ref{fig:btree-insert-fp}所示的命令式的插入结果相比较,我们可以看到它们的不同之处。它们都是满足B树性质的合法结果。


% ================================================================
%               Deletion
% ================================================================
\section{删除}
\index{B树!删除}

Deleting a key from
B-tree may violate balance properties. Except the root, a node shouldn't
contain too few keys less than $t-1$, where $t$ is the
minimum degree.

Similar to the approaches for insertion, we can either do some preparation
so that the node from where the key being deleted contains enough
keys; or do some fixing after the deletion if the node has too few keys.


% ================================================================
%               Merge before delete method
% ================================================================
\subsection{Merge before delete method}

We start from the easiest case. If the key $k$ to be deleted
can be located in node $x$, and $x$ is a leaf node,
we can directly remove $k$ from $x$. If $x$ is the root (the only
node of the tree), we needn't worry about there are too few
keys after deletion. This case is named as case 1 later.

In most cases, we start from the root, along a path to locate
where is the node contains $k$. If $k$ can be located in the
internal node $x$, there are three sub cases.

\begin{itemize}
\item Case 2a, If the child $y$ precedes $k$ contains enough keys (more than $t$).
We replace $k$ in node $x$ with $k'$, which is
the predecessor of $k$ in child $y$. And recursively remove $k'$
from $y$.

The predecessor of $k$ can be easily located as the last key of child
$y$.

This is shown in figure \ref{fig:btree-del-case2a}.

\begin{figure}[htbp]
  \centering
    \includegraphics[scale=0.5]{img/btree-del-case2a.eps}
    \caption{Replace and delete from predecessor.} \label{fig:btree-del-case2a}
\end{figure}

\item Case 2b, If $y$ doesn't contain enough keys, while the child $z$
follows $k$ contains more than $t$ keys. We replace $k$ in node $x$
with $k''$, which is the successor of $k$ in child $z$. And recursively
remove $k''$ from $z$.

The successor of $k$ can be easily located as the first key of child $z$.

This sub-case is illustrated in figure \ref{fig:btree-del-case2b}.

\begin{figure}[htbp]
  \centering
    \includegraphics[scale=0.5]{img/btree-del-case2b.eps}
    \caption{Replace and delete from successor.} \label{fig:btree-del-case2b}
\end{figure}

\item Case 2c, Otherwise, if neither $y$, nor $z$ contains enough keys, we
can merge $y$, $k$ and $z$ into one new node, so that this new node
contains $2t-1$ keys. After that, we can then recursively do the removing.

Note that after merge, if the current node doesn't contain any keys,
which means $k$ is the only key in $x$. $y$ and $z$ are the only two
children of $x$. we need shrink the tree height by one.
\end{itemize}

Figure \ref{fig:btree-del-case2c} illustrates this sub-case.

\begin{figure}[htbp]
  \centering
    \includegraphics[scale=0.5]{img/btree-del-case2c.eps}
    \caption{Merge and delete.} \label{fig:btree-del-case2c}
\end{figure}

the last case states that, if $k$ can't be located in node $x$, the algorithm
need find a child node $c_i$ in $x$, so that the sub-tree $c_i$
contains $k$. Before the deletion is recursively applied in $c_i$, we
need make sure that there are at least $t$ keys in $c_i$. If there are
not enough keys, the following adjustment is performed.

\begin{itemize}
\item Case 3a, We check the two sibling of $c_i$, which are $c_{i-1}$ and $c_{i+1}$.
If either one contains enough keys (at least $t$ keys), we move
one key from $x$ down to $c_i$, and move one key from the sibling up to
$x$. Also we need move the relative child from the sibling to $c_i$.

This operation makes $c_i$ contains enough keys for deletion. we can
next try to delete $k$ from $c_i$ recursively.

Figure \ref{fig:btree-del-case3a} illustrates this case.

\begin{figure}[htbp]
  \centering
    \includegraphics[scale=0.5]{img/btree-del-case3a.eps}
    \caption{Borrow from the right sibling.}
    \label{fig:btree-del-case3a}
\end{figure}

\item Case 3b, In case neither one of the two siblings contains enough keys, we then
merge $c_i$, a key from $x$, and either one of the sibling into a new
node. Then do the deletion on this new node.
\end{itemize}

Figure \ref{fig:btree-del-case3b} shows this case.

\begin{figure}[htbp]
  \centering
    \includegraphics[scale=0.5]{img/btree-del-case3b.eps}
    \caption{Merge $c_i$, $k$, and $c_{i+1}$ to a new node.}
    \label{fig:btree-del-case3b}
\end{figure}

Before define the B-tree delete algorithm, we need provide some auxiliary
functions. Function \textproc{Can-Del} tests if a node contains enough keys
for deletion.

\begin{algorithmic}[1]
\Function{Can-Del}{$T$}
  \State \Return $|K(T)| \ge t$
\EndFunction
\end{algorithmic}

Procedure \textproc{Merge-Children}($T, i$) merges child $c_i(T)$, key $k_i(T)$,
and child $c_{i+1}(T)$ into one big node.

\begin{algorithmic}[1]
\Procedure{Merge-Children}{$T, i$} \Comment{Merge $c_i(T)$, $k_i(T)$, and $c_{i+1}(T)$}
  \State $x \gets c_i(T)$
  \State $y \gets c_{i+1}(T)$
  \State $K(x) \gets K(x) \cup \{k_i(T)\} \cup K(y)$
  \State $C(x) \gets C(x) \cup C(y)$
  \State \Call{Remove-At}{$K(T), i$}
  \State \Call{Remove-At}{$C(T), i+1$}
\EndProcedure
\end{algorithmic}

Procedure \textproc{Merge-Children} merges the $i$-th child, the $i$-th key,
and $i+1$-th child of node $T$ into a new child, and remove the
$i$-th key and $i+1$-th child from $T$ after merging.

With these functions defined, the B-tree deletion algorithm can be given by
realizing the above 3 cases.

\begin{algorithmic}[1]
\Function{Delete}{$T, k$}
  \State $i \gets 1$
  \While{$i \leq |K(T)|$}
    \If{$k = k_i(T)$}
      \If{$T$ is leaf} \Comment{case 1}
        \State \Call{Remove}{$K(T), k$}
      \Else \Comment{case 2}
        \If{\Call{Can-Del}{$c_i(T)$}} \Comment{case 2a}
          \State $k_i(T) \gets$ \Call{Last-Key}{$c_i(T)$}
          \State \Call{Delete}{$c_i(T), k_i(T)$}
        \ElsIf{\Call{Can-Del}{$c_{i+1}(T)$}} \Comment{case 2b}
          \State $k_i(T) \gets$ \Call{First-Key}{$c_{i+1}(T)$}
          \State \Call{Delete}{$c_{i+1}(T), k_i(T)$}
        \Else \Comment{case 2c}
          \State \Call{Merge-Children}{$T, i$}
          \State \Call{Delete}{$c_i(T), k$}
          \If{$K(T) = NIL$}
            \State $T \gets c_i(T)$ \Comment{Shrinks height}
          \EndIf
        \EndIf
      \EndIf
      \State \Return $T$
    \ElsIf{$k < k_i(T)$}
      \State Break
    \Else
      \State $i \gets i+1$
    \EndIf
  \EndWhile
  \Statex
  \If{$T$ is leaf}
    \State \Return $T$ \Comment{$k$ doesn't exist in $T$.}
  \EndIf
  \If{$\lnot$ \Call{Can-Del}{$c_i(T)$}}  \Comment{case 3}
    \If{$i>1 \land$ \Call{Can-Del}{$c_{i-1}(T)$}} \Comment{case 3a: left sibling}
      \State \Call{Insert}{$K(c_i(T)), k_{i-1}(T)$}
      \State $k_{i-1}(T) \gets$ \Call{Pop-Back}{$K(c_{i-1}(T))$}
      \If{$c_i(T)$ isn't leaf}
        \State $c \gets$ \Call{Pop-Back}{$C(c_{i-1}(T))$}
        \State \Call{Insert}{$C(c_i(T)), c$}
      \EndIf
    \ElsIf{$i \leq |C(T)| \land$ \Call{Can-Del}{$c_{i_1}(T)$}} \Comment{case 3a: right sibling}
      \State \Call{Append}{$K(c_i(T)), k_i(T)$}
      \State $k_i(T) \gets$ \Call{Pop-Front}{$K(c_{i+1}(T))$}
      \If{$c_i(T)$ isn't leaf}
        \State $c \gets$ \Call{Pop-Front}{$C(c_{i+1}(T))$}
        \State \Call{Append}{$C(c_i(T)), c$}
      \EndIf
    \Else \Comment{case 3b}
      \If{$i>1$}
        \State \Call{Merge-Children}{$T, i-1$}
      \Else
        \State \Call{Merge-Children}{$T, i$}
      \EndIf
    \EndIf
  \EndIf
  \State \Call{Delete}{$c_i(T), k$} \Comment {recursive delete}
  \If{$K(T) = NIL$} \Comment {Shrinks height}
    \State $T \gets c_1(T)$
  \EndIf
  \State \Return $T$
\EndFunction
\end{algorithmic}

Figure \ref{fig:result-del1}, \ref{fig:result-del2}, and \ref{fig:result-del3}
show the deleting process step by step. The nodes modified are shaded.

\begin{figure}[htbp]
  \centering
  \subfloat[A B-tree before deleting.]{\includegraphics[scale=0.5]{img/btree-del-before.ps}} \\
  \subfloat[After delete key 'F', case 1.]{\includegraphics[scale=0.5]{img/btree-del-F.ps}}
  \caption{Result of B-tree deleting (1).} \label{fig:result-del1}
\end{figure}

\begin{figure}[htbp]
  \centering
  \subfloat[After delete key 'M', case 2a.]{\includegraphics[scale=0.5]{img/btree-del-M.ps}} \\
  \subfloat[After delete key 'G', case 2c.]{\includegraphics[scale=0.5]{img/btree-del-G.ps}}
  \caption{Result of B-tree deleting program (2)} \label{fig:result-del2}
\end{figure}

\begin{figure}[htbp]
  \centering
  \subfloat[After delete key 'D', case 3b, and height is shrunk.]{\includegraphics[scale=0.5]{img/btree-del-D.ps}} \\
  \subfloat[After delete key 'B', case 3a, borrow from right sibling.]{\includegraphics[scale=0.5]{img/btree-del-B.ps}} \\
  \subfloat[After delete key 'U', case 3a, borrow from left sibling.]{\includegraphics[scale=0.5]{img/btree-del-U.ps}}
  \caption{Result of B-tree deleting program (3)} \label{fig:result-del3}
\end{figure}

The following example Python program implements the B-tree deletion algorithm.

\lstset{language=Python}
\begin{lstlisting}
def can_remove(tr):
    return len(tr.keys) >= tr.t

def replace_key(tr, i, k):
    tr.keys[i] = k
    return k

def merge_children(tr, i):
    tr.children[i].keys += [tr.keys[i]] + tr.children[i+1].keys
    tr.children[i].children += tr.children[i+1].children
    tr.keys.pop(i)
    tr.children.pop(i+1)

def B_tree_delete(tr, key):
    i = len(tr.keys)
    while i>0:
        if key == tr.keys[i-1]:
            if tr.leaf:  # case 1 in CLRS
                tr.keys.remove(key)
            else: # case 2 in CLRS
                if tr.children[i-1].can_remove(): # case 2a
                    key = tr.replace_key(i-1, tr.children[i-1].keys[-1])
                    B_tree_delete(tr.children[i-1], key)
                elif tr.children[i].can_remove(): # case 2b
                    key = tr.replace_key(i-1, tr.children[i].keys[0])
                    B_tree_delete(tr.children[i], key)
                else: # case 2c
                    tr.merge_children(i-1)
                    B_tree_delete(tr.children[i-1], key)
                    if tr.keys==[]: # tree shrinks in height
                        tr = tr.children[i-1]
            return tr
        elif key > tr.keys[i-1]:
            break
        else:
            i = i-1
    # case 3
    if tr.leaf:
        return tr #key doesn't exist at all
    if not tr.children[i].can_remove():
        if i>0 and tr.children[i-1].can_remove(): #left sibling
            tr.children[i].keys.insert(0, tr.keys[i-1])
            tr.keys[i-1] = tr.children[i-1].keys.pop()
            if not tr.children[i].leaf:
                tr.children[i].children.insert(0, tr.children[i-1].children.pop())
        elif i<len(tr.children) and tr.children[i+1].can_remove(): #right sibling
            tr.children[i].keys.append(tr.keys[i])
            tr.keys[i]=tr.children[i+1].keys.pop(0)
            if not tr.children[i].leaf:
                tr.children[i].children.append(tr.children[i+1].children.pop(0))
        else: # case 3b
            if i>0:
                tr.merge_children(i-1)
            else:
                tr.merge_children(i)
    B_tree_delete(tr.children[i], key)
    if tr.keys==[]: # tree shrinks in height
        tr = tr.children[0]
    return tr
\end{lstlisting}

% ================================================================
%               Delete and fix method
% ================================================================

\subsection{Delete and fix method}

The merge and delete algorithm is a bit complex.
There are several cases, and in each case, there are sub cases to deal.

Another approach to design the deleting algorithm is to perform fixing
after deletion. It is similar to the insert-then-fix strategy.

\be
delete(T, k) = fix(del(T, k))
\ee

When delete a key from B-tree, we firstly locate
which node this key is contained. We traverse
from the root to the leaves till find this key in some node.

If this node is a leaf, we can remove the key, and then
examine if the deletion makes the node contains too few keys to
satisfy the B-tree balance properties.

If it is a branch node, removing the key breaks the node into
two parts. We need merge them together. The merging is a recursive
process which is shown in figure \ref{fig:del-fp-merge}.

\begin{figure}[htbp]
  \centering
  \includegraphics[scale=0.5]{img/btree-del-fp-merge.eps}
  \caption{Delete a key from a branch node. Removing $k_i$ breaks
the node into 2 parts. Merging these 2 parts
is a recursive process. When the two parts are leaves, the merging
terminates.} \label{fig:del-fp-merge}
\end{figure}

When do merging, if the two nodes are not leaves, we merge the keys
together, and recursively merge the last child of the left part
and the first child of the right part to one new node. Otherwise,
if they are leaves, we merely put all keys together.

Till now, the deleting in performed in straightforward way. However, deleting
decreases the number of keys of a node, and it may result in
violating the B-tree balance properties. The solution is to perform
fixing along the path traversed from root.

\begin{figure}[htbp]
  \centering
  \includegraphics[scale=0.5]{img/btree-del-fp-make.eps}
  \caption{After delete key $k$ from node $c_i$, denote the result
as $c_i'$. The fixing makes a new node from the
left part, $c_i'$ and the right part.}
  \label{fig:del-fp-make}
\end{figure}

During the recursive deletion, the branch node is broken into 3 parts.
The left part contains all keys less than $k$, includes $k_1, k_2, ..., k_{i-1}$,
and children $c_1, c_2, ..., c_{i-1}$, the right part contains all keys
greater than $k$, say $k_i, k_{i+1}, ..., k_{n+1}$, and children
$c_{i+1}, c_{i+2}, ..., c_{n+1}$. Then key $k$ is recursively deleted
from child $c_i$. Denote the result becomes $c_i'$ after that.
We need make a new node from these 3 parts,
as shown in figure \ref{fig:del-fp-make}.

At this time point, we need examine if $c_i'$ contains enough keys.
If there are to less keys (less than $t-1$, but not $t$ in
contrast to the merge-and-delete approach), we can either borrow a key-child
pair from the left or the right part, and do inverse operation of
splitting. Figure \ref{fig:del-fp-fixlow} shows example of borrowing from the left part.

\begin{figure}[htbp]
  \centering
  \includegraphics[scale=0.5]{img/btree-del-fp-fixlow.eps}
  \caption{Borrow a key-child pair from left part and
un-split to a new child.} \label{fig:del-fp-fixlow}
\end{figure}

If both left part and right part are empty, we can simply
push $c_i'$ up.

Denote the B-tree as $T=(K, C, t)$, where $K$ and $C$ are keys and children.
The $del(T, k)$ function deletes key $k$
from the tree.

\be
del(T, k) = \left \{
  \begin{array}
  {r@{\quad:\quad}l}
  (delete(K, k), \Phi, t) & C = \Phi \\
  merge((K_1, C_1, t), (K_2, C_2, t)) & k_i = k \\
  make((K_1', C_1'), del(c, k), (K_2', C_2')) & k \notin K
  \end{array}
\right.
\ee

If children $C = \Phi$ is empty, $T$ is leaf. $k$ is deleted from keys directly.
Otherwise, $T$ is internal node. If $k \in K$, removing it separates the keys
and children in two parts $(K_1, C_1)$ and $(K_2, C_2)$. They will be recursively
merged.

\[
\begin{array}{l}
K_1 = \{k_1, k_2, ..., k_{i-1}\} \\
K_2 = \{k_{i+1}, k_{i+2}, ..., k_m\} \\
C_1 = \{c_1, c_2, ..., c_i\} \\
C_2 = \{c_{i+1}, c_{i+2}, ..., c_{m+1}\}
\end{array}
\]

If $k \notin K$, we need locate a child $c$, and further delete $k$ from
it.

\[
\begin{array}{l}
(K_1', K_2') = (\{k' | k' \in K, k' < k \}, \{k' | k' \in K, k < k' \}) \\
(C_1', \{c\} \cup C_2') = splitAt(|K_1'|, C)
\end{array}
\]

The recursive merge function is defined as the following. When merge two
trees $T_1 = (K_1, C_1, t)$ and $T_2 = (K_2, C_2, t)$, if both are leaves,
we create a new leave by concatenating the keys. Otherwise, the last
child in $C_1$, and the first child in $C_2$ are recursively merged.
And we call $make$ function to form the new tree. When $C_1$ and $C_2$
are not empty, denote the last child of $C_1$ as $c_{1, m}$, the rest
as $C_1'$; the first child of $C_2$ as $C_{2, 1}$, the rest as $C_2'$.
Below equation defines the merge function.

\be
merge(T_1, T_2) = \left \{
  \begin{array}
  {r@{\quad:\quad}l}
  (K_1 \cup K_2, \Phi, t) & C_1 = C_2 = \Phi \\
  make((K_1, C_1'), merge(c_{1,m}, c_{2, 1}), (K_2, C_2')) & otherwise
  \end{array}
\right.
\ee

The $make$ function defined above only handles the case that a node
contains too many keys due to insertion. When delete key, it may
cause a node contains too few keys. We need test and fix this
situation as well.

\be
make((K', C'), c, (K'', C'')) = \left \{
  \begin{array}
  {r@{\quad:\quad}l}
  fixFull((K', C'), c, (K'', C'')) & full(c) \\
  fixLow((K', C'), c, (K'', C'')) & low(c) \\
  (K' \cup K'', C' \cup \{c\} \cup C'', t) & otherwise
  \end{array}
\right.
\ee

Where $low(T)$ checks if there are too few keys less than $t-1$.
Function $fixLow(P_l, c, P_r)$ takes three arguments, the left pair of keys and children,
a child node, and the right pair of keys and children.
If the left part isn't empty, we borrow a pair of key-child, and
do un-splitting to make the child contain enough keys, then recursively
call $make$; If the right part isn't empty, we borrow a pair
from the right; and if both sides are empty, we
return the child node as result. In this case, the height of the tree
shrinks.

Denote the left part $P_l = (K_l, C_l)$. If $K_l$ isn't empty, the last
key and child are represented as $k_{l, m}$ and $c_{l, m}$ respectively.
The rest keys and children become $K_l'$ and $C_l'$; Similarly, the
right part is denoted as $P_r = (K_r, C_r)$. If $K_r$ isn't empty, the
first key and child are represented as $k_{r, 1}$, and $c_{r, 1}$. The
rest keys and children are $K_r'$ and $C_r'$. Below equation gives
the definition of $fixLow$.

\be
fixLow(P_l, c, P_r) = \left \{
  \begin{array}
  {r@{\quad:\quad}l}
  make((K_l', C_l'), unsplit(c_{l, m}, k_{l, m}, c), (K_r, C_r)) & K_l \neq \Phi \\
  make((K_r, C_r), unsplit(c, k_{r, 1}, c_{r, 1}), (K_r', C_r')) & K_r \neq \Phi \\
  c & otherwise
  \end{array}
\right.
\ee

Function $unsplit(T_1, k, T_2)$ is the inverse operation to splitting.
It forms a new B-tree nodes from two small nodes and a key.

\be
unsplit(T_1, k, T_2) = (K_1 \cup \{k\} \cup K_2, C_1 \cup C_2, t)
\ee

The following example Haskell program implements the B-tree deletion
algorithm.

\lstset{language=Haskell}
\begin{lstlisting}
import qualified Data.List as L

delete tr x = fixRoot $ del tr x

del:: (Ord a) => BTree a -> a -> BTree a
del (Node ks [] t) x = Node (L.delete x ks) [] t
del (Node ks cs t) x =
    case L.elemIndex x ks of
      Just i -> merge (Node (take i ks) (take (i+1) cs) t)
                      (Node (drop (i+1) ks) (drop (i+1) cs) t)
      Nothing -> make (ks', cs') (del c x) (ks'', cs'')
    where
      (ks', ks'') = L.partition (<x) ks
      (cs', (c:cs'')) = L.splitAt (length ks') cs

merge (Node ks [] t) (Node ks' [] _) = Node (ks++ks') [] t
merge (Node ks cs t) (Node ks' cs' _) = make (ks, init cs)
                                             (merge (last cs) (head cs'))
                                             (ks', tail cs')

make (ks', cs') c (ks'', cs'')
    | full c = fixFull (ks', cs') c (ks'', cs'')
    | low c  = fixLow  (ks', cs') c (ks'', cs'')
    | otherwise = Node (ks'++ks'') (cs'++[c]++cs'') (degree c)

low tr = (length $ keys tr) < (degree tr)-1

fixLow (ks'@(_:_), cs') c (ks'', cs'') = make (init ks', init cs')
                                              (unsplit (last cs') (last ks') c)
                                              (ks'', cs'')
fixLow (ks', cs') c (ks''@(_:_), cs'') = make (ks', cs')
                                              (unsplit c (head ks'') (head cs''))
                                              (tail ks'', tail cs'')
fixLow _ c _ = c

unsplit c1 k c2 = Node ((keys c1)++[k]++(keys c2))
                       ((children c1)++(children c2)) (degree c1)
\end{lstlisting}

When delete the same keys from the B-tree as in merge and fixing
approach, the results are different. However, both satisfy
the B-tree properties, so they are all valid.

\begin{figure}[htbp]
  \centering
  \subfloat[B-tree before deleting]{\includegraphics[scale=0.5]{img/btree-del-fp-before.ps}}\\
  \subfloat[After delete key 'E'.]{\includegraphics[scale=0.5]{img/btree-del-fp-E.ps}}
  \caption{Result of delete-then-fixing (1)} \label{fig:result-del-fp1}
\end{figure}

\begin{figure}[htbp]
  \centering
  \subfloat[After delete key 'G',]{\includegraphics[scale=0.5]{img/btree-del-fp-G.ps}} \\
  \subfloat[After delete key 'A'.]{\includegraphics[scale=0.5]{img/btree-del-fp-A.ps}}
  \caption{Result of delete-then-fixing (2)} \label{fig:result-del-fp2}
\end{figure}

\begin{figure}[htbp]
  \centering
  \subfloat[After delete key 'M'.]{\includegraphics[scale=0.5]{img/btree-del-fp-M.ps}} \\
  \subfloat[After delete key 'U'.]{\includegraphics[scale=0.5]{img/btree-del-fp-U.ps}}
  \caption{Result of delete-then-fixing (3)} \label{fig:result-del-fp3}
\end{figure}


% ================================================================
%               Searching
% ================================================================
\section{Searching}
\index{B-tree!search}

Searching in B-tree can be considered as the generalized
tree search extended from binary search tree.

When searching in the binary tree, there are only 2 different directions,
the left and the right. However, there are multiple directions in B-tree.

\begin{algorithmic}[1]
\Function{Search}{$T, k$}
  \Loop
    \State $i \gets 1$
    \While{$i \leq |K(T)| \land k > k_i(T)$}
      \State $i \gets i+1$
    \EndWhile
    \If{$i \leq |K(T)| \land k = k_i(T)$}
      \State \Return $(T, i)$
    \EndIf
    \If{$T$ is leaf}
      \State \Return $NIL$ \Comment{$k$ doesn't exist}
    \Else
      \State $T \gets c_i(T)$
    \EndIf
  \EndLoop
\EndFunction
\end{algorithmic}

Starts from the root, this program examines each key one by one
from the smallest to the biggest. In case it finds the
matched key, it returns the current node and the index of this
key. Otherwise, if it finds the position $i$ that $k_i < k < k_{i+1}$,
the program will next search the child node
$c_{i+1}$ for the key. If it traverses to some leaf node, and
fails to find the key, the empty value is returned to
indicate that this key doesn't exist in the tree.

The following example Python program implements the search algorithm.

\lstset{language=Python}
\begin{lstlisting}
def B_tree_search(tr, key):
    while True:
        for i in range(len(tr.keys)):
            if key <= tr.keys[i]:
                break
        if key == tr.keys[i]:
            return (tr, i)
        if tr.leaf:
            return None
        else:
            if key > tr.keys[-1]:
                i=i+1
            tr = tr.children[i]
\end{lstlisting}

The search algorithm can also be realized by recursion. When search
key $k$ in B-tree $T = (K, C, t)$, we partition the keys with $k$.

\[
\begin{array}{l}
K_1 = \{ k' | k' < k \} \\
K_2 = \{ k' | k \leq k' \}
\end{array}
\]

Thus $K_1$ contains all the keys less than $k$, and $K_2$ holds the
rest. If the first element in $K_2$ is equal to $k$, we find the key.
Otherwise, we recursively search the key in child $c_{|K_1|+1}$.

\be
search(T, k) = \left \{
  \begin{array}
  {r@{\quad:\quad}l}
  (T, |K_1|+1) & k \in K_2 \\
  \Phi & C = \Phi \\
  search(c_{|K_1|+1}, k) & otherwise
  \end{array}
\right.
\ee

Below example Haskell program implements this algorithm.

\lstset{language=Haskell}
\begin{lstlisting}
search :: (Ord a)=> BTree a -> a -> Maybe (BTree a, Int)
search tr@(Node ks cs _) k
    | matchFirst k $ drop len ks = Just (tr, len)
    | otherwise = if null cs then Nothing
                  else search (cs !! len) k
    where
      matchFirst x (y:_) = x==y
      matchFirst x _ = False
      len = length $ filter (<k) ks
\end{lstlisting}


% ================================================================
%                 Short summary
% ================================================================
\section{Notes and short summary}
In this chapter, we explained the B-tree data structure as a kind of
extension from binary search tree. The background knowledge of
magnetic disk access is skipped, user can refer to \cite{CLRS}
for detail. For the three main operations, insertion, deletion,
and searching, both imperative and functional algorithms are
given. They traverse from the root to the leaf. All the three
operations perform in time proportion to the height of the tree.
Because B-tree always maintains the balance properties. The
performance is ensured to bound to $O(\lg n)$ time,
where $n$ is the number of the keys in B-tree.

\begin{Exercise}
\begin{itemize}
\item When insert a key, we need find a position, where all keys on the left are less than it, while all the others on the right are greater than it. Actually, we can loose the constraint, that all the keys on the left are not greater than, while all the others on the right are not less than the key. Modify the algorithm so that the elements stored in B-tree only need support less-than and equality test.
\item We assume the element being inserted doesn't exist in the tree. Modify the algorithm so that duplicated elements can be stored in a linked-list.
\item Eliminate the recursion in imperative B-tree insertion algorithm.
\end{itemize}
\end{Exercise}

\begin{thebibliography}{99}

\bibitem{CLRS}
Thomas H. Cormen, Charles E. Leiserson, Ronald L. Rivest and Clifford Stein. ``Introduction to Algorithms, Second Edition''. The MIT Press, 2001. ISBN: 0262032937.

\bibitem{wiki-b-tree}
B-tree, Wikipedia. http://en.wikipedia.org/wiki/B-tree

\bibitem{okasaki-rbtree}
Chris Okasaki. ``FUNCTIONAL PEARLS Red-Black Trees in a Functional Setting''. J. Functional Programming. 1998

\end{thebibliography}

\ifx\wholebook\relax \else
\end{document}
\fi
