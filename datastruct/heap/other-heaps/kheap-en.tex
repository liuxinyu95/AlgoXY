\ifx\wholebook\relax \else
\documentclass[b5paper]{article}
\usepackage[nomarginpar
  %, margin=.5in
]{geometry}

\addtolength{\oddsidemargin}{-0.05in}
\addtolength{\evensidemargin}{-0.05in}
\addtolength{\textwidth}{0.1in}
\usepackage[en]{../../../prelude}

\setcounter{page}{1}

\begin{document}

\title{Binomial heap, Fibonacci heap, and pairing heap}

\author{Xinyu~LIU
\thanks{{\bfseries Xinyu LIU} \newline
  Email: liuxinyu95@gmail.com \newline}
  }

\maketitle
\fi

\markboth{Binomial heap, Fibonacci heap, and pairing heap}{Elementary Algorithms}

\ifx\wholebook\relax
\chapter{Binomial heap, Fibonacci heap, and pairing heap}
\numberwithin{Exercise}{chapter}
\fi

\section{Introduction}
\label{introduction}

Binary heap stores elements in a binary tree, we can extend it to $k$-ary tree\cite{K-ary-tree} ($k > 2$ multi-ways tree), or multiple trees. This chapter introduces binomial heap, which consists of forest of $k$-ary trees. When delay some operations to a Binomial heap, we obtained Fibonacci heap. It improves the heap merge performance from $O(\lg n)$ time bound to amortized constant time. This is critical for graph algorithm design. We give pairing heap as a simplified heap implementation with good overall performance.

\section{Binomial Heaps}
\label{sec:binomial-heap} \index{Binomial heap}

Binomial heap is named after Newton's binomial theorem. It consists of a set of $k$-ary trees (also called a forest). Every tree has the size equal to a binomial coefficient. Newton proved that $(a + b)^n$ expands to:

\be
(a + b)^n = a^n + \binom{n}{1} a^{n-1}b + ... + \binom{n}{n-1} a b^{n-1} + b
\ee

When $n$ is a natural number, the coefficients is some row in Pascal's triangle\footnote{Also know as the {\em Jia Xian}'s triangle named after ancient Chinese mathematician Jia Xian (1010-1070). Newton generalized $n$ to rational numbers, later Euler expand it to real exponents.}\cite{wiki-pascal-triangle}.

\begin{verbatim}
    1
   1 1
  1 2 1
 1 3 3 1
1 4 6 4 1
...
\end{verbatim}

The first row is 1, all the first and last numbers are 1 for every row. Any other number is the sum of the top-left and top-right numbers in the previous row. There are many methods to generate pascal triangles, like recursion.

\subsubsection{Binomial tree}
\label{Binomial tree} \index{Binomial tree}

A binomial tree is a multi-ways tree with an integer rank. Denoted as $B_0$ if the rank is 0, and $B_n$ for rank $n$.

\begin{enumerate}
\item $B_0$ has only one node;
\item $B_n$ is formed by two $B_{n-1}$ trees, the one with the greater root element is the left most sub-tree of the other, as shown in figure \ref{fig:link-bitree}.
\end{enumerate}

\begin{figure}[htbp]
  \centering
  \includegraphics[scale=0.5]{img/btrees}
  \caption{Binomial tree}
  \label{fig:link-bitree}
\end{figure}

Figure \ref{fig:bitree-forms} gives examples of $B_0$ to $B_4$.

\begin{figure}[htbp]
  \centering
  \subcaptionbox{$B_0$}{\hspace{0.05\textwidth}\includegraphics[scale=0.5]{img/b0tree}\hspace{0.05\textwidth}}
  \subcaptionbox{$B_1$}{\hspace{0.05\textwidth}\includegraphics[scale=0.5]{img/b1tree}\hspace{0.05\textwidth}}
  \subcaptionbox{$B_2$}{\includegraphics[scale=0.5]{img/b2tree}}
  \subcaptionbox{$B_3$}{\includegraphics[scale=0.5]{img/b3tree}} \\
  \subcaptionbox{$B_4$}{\includegraphics[scale=0.5]{img/b4tree}...}
  \caption{Binomial trees of rank 0, 1, 2, 3, 4, ...}
  \label{fig:bitree-forms}
\end{figure}

We can find the number of nodes in every row in $B_n$ is a binomial coefficient. For example in $B_4$, there is a node (root) in level 0, 4 nodes in level 1, 6 nodes in level 2, 4 nodes in level 3, and a node in level 4. They are exactly same as the 4th row (start from 0) of Pascal's triangle: 1, 4, 6, 4, 1. This is the reason why we name it binomial tree. We can further know there are $2^n$ elements in a $B_n$ tree.

\label{Binomial heap} \index{Binomial heap!definition}

A binomial heap is a set of binomial trees (a forest) that satisfies the following two rules:

\begin{enumerate}
\item Every tree satisfies the {\em heap property}, i.e. for min heap, the element in every node is not less than ($\geq$) its parent;
\item Every tree has unique rank. i.e. any two trees have different ranks.
\end{enumerate}

From the 2nd rule, for a binomial heap of $n$ elements, convert $n$ to its binary format $(a_m ... a_1, a_0)_2$, where $a_0$ is the least significant bit (LSB) and $a_m$ is the most significant bit (MSB). If if $a_i=0$, there is no tree of rank $i$; if $a_i = 1$, there is a tree of rank $i$. For example, consider a binomial heap of 5 elements. As 5 is 101 in binary, there are 2 binomial trees, one is $B_0$, the other is $B_2$. The binomial heap in figure \ref{fig:bheap2} has 19 elements, 19 is $(10011)_2$. There is a $B_0$, a $B_1$, and a $B_4$.

\begin{figure}[htbp]
  \centering
  \includegraphics[scale=0.5]{img/bheap2}
  \caption{A binomial heap with 19 elements}
  \label{fig:bheap2}
\end{figure}

We define the binomial tree as $(r, k, ts)$, where $r$ is the rank, $k$ is the element in the root, and $ts$ is the list of sub-trees ordered by rank.

\lstset{frame=single}
\begin{Haskell}
data BiTree a = Node Int a [BiTree a]

type BiHeap a = [BiTree a]
\end{Haskell}

\index{left child, right sibling}
There is a method called `left-child, right-sibling'\cite{CLRS}, that can reuse the binary tree data structure to define multi-ways tree. Every node has the left and right part. the left references to the first sub-tree; the right references to its sibling. All siblings form a list as shown in figure \ref{fig:lcrs}. Alternatively, we can use an array or a list to represent the sub-trees.

\begin{figure}[htbp]
  \centering
  \includegraphics[scale=0.5]{img/left-child-right-sibling}
  \caption{$R$ is the root, $T_1, T_2, ..., T_m$ are sub-trees of $R$. The left of $R$ is $T_1$, the right is NIL. $T_{11}, ..., T_{1p}$ are sub-trees of $T_1$. The left of $T_1$ is $T_{11}$, the right is its sibling $T_2$. The left of $T_2$ is $T_{21}$, the left is sibling.}
  \label{fig:lcrs}
\end{figure}

\subsection{Link}
\index{Binomial Heap!Link}

To link two $B_n$ trees to a $B_{n+1}$ tree, we compare the two root elements, choose the smaller one as the root, and put the other tree ahead of other sub-trees as shown in figure \ref{fig:link-xy}.

\be
link\ (r, x, ts)\ (r, y, ts') = \begin{cases}
  x < y: & (r + 1, x, (r, t, ts') : ts) \\
  \text{otherwise}: & (r + 1, y, (r, x, ts): ts') \\
  \end{cases}
\label{eq:link}
\ee

\begin{figure}[htbp]
  \centering
  \includegraphics[scale=0.5]{img/link-bitree-xy}
  \caption{If $x < y$, link $y$ as the first sub-tree of $x$.}
  \label{fig:link-xy}
\end{figure}

We can implement link with `left child, right sibling' method as below. Link operation is bound to constant time.

\begin{algorithmic}[1]
\Function{Link}{$x, y$}
  \If{\Call{Key}{$y$} $<$ \Call{Key}{$x$}}
    \State Exchange $x \leftrightarrow y$
  \EndIf
  \State \Call{Sibling}{$y$} $\gets$ \Call{Sub-Trees}{$T_1$}
  \State \Call{Sub-Trees}{$x$} $\gets y$
  \State \Call{Parent}{$y$} $\gets x$
  \State \Call{Rank}{$x$} $\gets$ \Call{Rank}{$y$} + 1
  \State \Return $x$
\EndFunction
\end{algorithmic}

\begin{Exercise}
\Question{Write a program to generate Pascal's triangle.}
\Question{Prove that the $i$-th row in tree $B_n$ has $\binom{n}{i}$ nodes.}
\Question{Prove there are $2^n$ elements in $B_n$ tree.}
\Question{Use a container to store sub-trees, how to implement link? How to secure the operation is in constant time?}
\end{Exercise}

\begin{Answer}
\Question{Write a program to generate Pascal's triangle.
\begin{Haskell}[frame=single]
pascal = gen [1] where
  gen cs (x:y:xs) = gen ((x + y) : cs) (y:xs)
  gen cs _ = 1 : cs
\end{Haskell}
}
\Question{Prove that the $i$-th row in tree $B_n$ has $\binom{n}{i}$ nodes.
\begin{proof}
Use induction. There is only a root node for $B_0$. Assume every row in $B_n$ is binomial number. Tree $B_{n+1}$ is composed from two $B_n$ trees. The 0-th row contains root: $1 = \binom{n+1}{0}$. The $i$-th row has two parts: one from the $(i-1)$-th row of the left most sub-tree $B_n$, the other from the $i$-th row of the other $B_n$ tree. In total:

\[
\begin{array}{rcl}
\binom{n}{i-1} + \binom{n}{i} & = & \dfrac{n!}{(i-1)!(n-i+1)!} + \dfrac{n!}{i!(n-i)!} \\
 & = & \dfrac{n!}{(i-1)!(n-i)!}(\dfrac{1}{i} - \dfrac{1}{n-i+1}) \\
 & = & \dfrac{n!}{(i-1)!(n-i)!}\dfrac{n+1}{i(n-i+1)} \\
 & = & \dfrac{(n+1)!}{i!(n-i+1)!} \\
 & = & \binom{n+1}{i} \\
\end{array}
\]
\end{proof}
}
\Question{Prove there are $2^n$ elements in $B_n$ tree.
\begin{proof}
From previous exercise, sum all rows of $B_n$ tree:
\[
\begin{array}{cll}
  & \binom{n}{0} + \binom{n}{1} + ... + \binom{n}{n} & \text{各行相加} \\
= & (1 + 1)^n & \text{令二项式} (a + b)^n \text{中}a = b = 1 \\
= & 2^n & \\
\end{array}
\]
\end{proof}
}
\Question{Use a container to store sub-trees, how to implement link? How to secure the operation is in constant time?
If store all sub-trees in an array, we need linear time to insert a new tree ahead of all sub-trees:

\begin{algorithmic}[1]
\Function{Link'}{$T_1, T_2$}
  \If{\Call{Key}{$T_2$} $<$ \Call{Key}{$T_1$}}
    \State Exchange $T_1 \leftrightarrow T_2$
  \EndIf
  \State \Call{Parent}{$T_2$} $\gets T_1$
  \State \textproc{Insert}(\Call{Sub-Trees}{$T_1$}, 1, $T_2$)
  \State \Call{Rank}{$T_1$} $\gets$ \Call{Rank}{$T_2$} + 1
  \State \Return $T_1$
\EndFunction
\end{algorithmic}

We can store the sub-trees in reversed order, it's need constant time to append the new tree on tail.
}
\end{Answer}

\subsubsection{Insert}
\index{Binomial heap!insert} \index{Binomial heap!push}

When insert a new tree, we keep the trees in binomial heap ordered by rank (ascending):

\be
\begin{array}{rcl}
ins\ t\ [\ ] & = & [t] \\
ins\ t\ (t':ts) & = & \begin{cases}
  rank\ t < rank\ t': & t:t':ts \\
  rank\ t' < rank\ t: & t' : ins\ t\ ts \\
  \text{otherwise}: & ins\ (link\ t\ t')\ ts  \\
\end{cases}
\end{array}
\ee

Where $rank\ (r, k, ts) = r$ gives the rank of a tree. For empty heap $[\ ]$, it becomes a single list of the new tree $t$; otherwise, we compare the rank of $t$ with the first tree $t'$, if $t$ has less rank, then it becomes the new first one; if $t'$ has less rank, we recursively insert $t$ to the rest trees; if they have the same rank, then link $t$ and $t'$ to a bigger tree, and recursively insert to the rest. For $n$ elements, there are at most $O(\lg n)$ binomial trees in the heap. $ins$ links $O(\lg n)$ time at most, as linking is bound to constant time, the overall performance is bound to $O(\lg n)$\footnote{It's similar to adding two binary numbers. A more generic topic is {\em numeric representation}\cite{okasaki-book}.}. We can define insert for binomial heap with $ins$. First wrap the new element $x$ in a singleton tree, then insert the tree to the heap:

\be
insert\ x = ins\ (0, x, [\ ])
\ee

This is a Curried definition, we can further insert a list of elements to the heap by using fold:

\be
\textit{fromList} = foldr\ insert\ [\ ]
\ee

Below is the implementation with 'left child, right sibling' method: \label{alg:insert-tree}

\begin{algorithmic}[1]
\Function{Insert-Tree}{$T, H$}
  \State $\perp \gets p \gets$ \Call{Node}{$0$, NIL, NIL}
  \While{$H \neq$ NIL 且 \Call{Rank}{$H$} $\leq$ \Call{Rank}{$T$}}
    \State $T_1 \gets H$
    \State $H \gets $ \Call{Sibling}{$H$}
    \If{\Call{Rank}{$T$} = \Call{Rank}{$T_1$}}
      \State $T \gets$ \Call{Link}{$T, T_1$}
    \Else
      \State \Call{Sibling}{$p$} $\gets T_1$
      \State $p \gets T_1$
    \EndIf
  \EndWhile
  \State \Call{Sibling}{$p$} $\gets T$
  \State \Call{Sibling}{$T$} $\gets H$
  \State \Return \Call{Remove-First}{$\perp$}
\EndFunction
\Statex
\Function{Remove-First}{$H$}
  \State $n \gets$ \Call{Sibling}{$H$}
  \State \Call{Sibling}{$H$} $\gets$ NIL
  \State \Return $n$
\EndFunction
\end{algorithmic}

\subsection{Merge}
\index{Binomial tree!merge}

When merge two binomial heaps, we actually merge two lists of binomial trees. Every tree has unique rank in merged result, and the ranks are in ascending order. The tree merge process is similar to merge sort. Every time, we pick the first tree from each heap, compare their ranks, put the smaller one to the result. If the two trees have the same rank, we link them to a bigger one, and recursively insert to the merge result.

\be
\begin{array}{rcl}
merge\ ts_1\ [\ ] & = & ts_1 \\
merge\ [\ ]\ ts_2 & = & ts_2 \\
merge\ (t_1:ts_1)\ (t_2:ts_2) & = & \begin{cases}
  rank\ t_1 < rank\ t_2: & t_1 : (merge\ ts_1\ (t_2:ts_2)) \\
  rank\ t_2 < rank\ t_1: & t_2 : (merge\ (t_1:ts_1)\ ts_2) \\
  \text{otherwise}: & ins\ (link\ t_1\ t_2)\ (merge\ ts_1\ ts_2) \\
  \end{cases}
\end{array}
\ee

Alternatively, when $t_1$ and $t_2$ have the same rank, we can insert the linked tree back to either heap, and recursively merge:

\[
merge\ (ins\ (link\ t_1\ t_2)\ ts_1)\ ts_2
\]

We can also eliminate recursion, and implement iterative merge:

\begin{algorithmic}[1]
\Function{Merge}{$H_1, H_2$}
  \State $H \gets p \gets$ \Call{Node}{0, NIL, NIL}
  \While{$H_1 \neq$ NIL and $H_2 \neq$ NIL}
    \If{\Call{Rank}{$H_1$} $<$ \Call{Rank}{$H_2$}}
      \State \Call{Sibling}{$p$} $\gets H_1$
      \State $p \gets$ \Call{Sibling}{$p$}
      \State $H_1 \gets$ \Call{Sibling}{$H_1$}
    \ElsIf{\Call{Rank}{$H_2$} $<$ \Call{Rank}{$H_1$}}
      \State \Call{Sibling}{$p$} $\gets H_2$
      \State $p \gets$ \Call{Sibling}{$p$}
      \State $H_2 \gets$ \Call{Sibling}{$H_2$}
    \Else \Comment{same rank}
      \State $T_1 \gets H_1, T_2 \gets H_2$
      \State $H_1 \gets$ \Call{Sibling}{$H_1$}, $H_2 \gets$ \Call{Sibling}{$H_2$}
      \State $H_1 \gets $ \textproc{Insert-Tree}(\Call{Link}{$T_1, T_2$}, $H_1$)
    \EndIf
  \EndWhile
  \If{$H_1 \neq$ NIL}
    \State \Call{Sibling}{$p$} $\gets H_1$
  \EndIf
  \If{$H_2 \neq$ NIL}
    \State \Call{Sibling}{$p$} $\gets H_2$
  \EndIf
  \State \Return \Call{Remove-First}{$H$}
\EndFunction
\end{algorithmic}

If there are $m_1$ trees in $H_1$, $m_2$ trees in $H_2$. There are at most $m_1 + m_2$ trees after merge. The merge is bound to $O(m_1 + m_2)$ time if all trees have different ranks. If there exist trees of the same rank, we call $ins$ up to $O(m_1 + m_2)$ times. Consider $m_1 = 1 + \lfloor \lg n_1 \rfloor$ and $m_2 = 1 + \lfloor \lg n_2 \rfloor$, where $n_1$, $n_2$ are the numbers of elements in each heap, and $\lfloor \lg n_1 \rfloor + \lfloor \lg n_2 \rfloor \leq 2 \lfloor \lg n \rfloor$, where $n = n_1 + n_2$. The final performance of merge is $O(\lg n)$.

\subsubsection{Pop}
\index{Binomial heap!pop}

Although every tree has the minimal element in its root, we don't know which tree holds the overall minimum in the heap. We need locate it from all trees. As there are $O(\lg n)$ trees, it takes $O(\lg n)$ time to find the top element. For pop, we need further remove the top element and maintain heap property. Let the trees be $B_i, B_j, ..., B_p, ..., B_m$ in the heap, and the minimum is in the root of $B_p$. After remove the top, there leave $p$ sub binomial trees with ranks of $p-1, p-2, ..., 0$. We can reverse them to form a new binomial heap $H_p$. The other trees without $B_p$ also form a binomial heap $H' = H - [B_p]$. We merge $H_p$ and $H'$ to get the final result as shown in figure \ref{fig:bheap-del-min}. Below is the definition to access the minimal element in the heap.

\begin{figure}[htbp]
  \centering
  \includegraphics[scale=0.4]{img/bheap-pop}
  \caption{Binomial heap pop.}
  \label{fig:bheap-del-min}
\end{figure}

\be
top\ (t:ts) = foldr\ f\ (key\ t)\ ts
\ee

\[
f\ (r, x, ts)\ y = min\ x\ y
\]

It's means to traverse all trees and find the which root has the minimum.

\begin{algorithmic}[1]
\Function{Top}{$H$}
  \State $m \gets \infty$
  \While{$H \neq$ NIL}
    \State $m \gets$ \textproc{Min}($m$, \Call{Key}{$H$})
    \State $H \gets $ \Call{Sibling}{$H$}
  \EndWhile
  \State \Return $m$
\EndFunction
\end{algorithmic}

To support pop, we need extract the tree containing the minimum out:

\be
\begin{array}{rcl}
min'\ [t] & = & (t, [\ ]) \\
min'\ (t:ts) & = & \begin{cases}
  key\ t < key\ t': & (t, ts), \text{其中}: (t', ts') = min'\ ts \\
  \text{否则}: & (t', t:ts')
  \end{cases}
\end{array}
\label{eq:extract-min-bitree}
\ee

Where $key\ (r, k, ts) = k$ accesses the root element, the result of $min'$ is a pair: the tree containing the minimum and the remaining trees. We next define $pop$ with it:

\be
pop\ H = (k, merge\ (reverse\ ts)\ H'), \text{其中}: ((r, k, ts), H') = min'\ H
\ee

The iterative implementation is as below:

\begin{algorithmic}[1]
\Function{Pop}{$H$}
  \State $(T_m, H) \gets$ \Call{Extract-Min}{$H$}
  \State $H \gets$ \textproc{Merge}($H$, \textproc{Reverse}(\Call{Sub-Trees}{$T_m$}))
  \State \Call{Sub-Trees}{$T_m$}
  \State \Return (\Call{Key}{$T_m$}, $H$)
\EndFunction
\end{algorithmic}

Where the list reverse is defined in chapter 1, \textproc{Extract-Min} is implemented as below:

\begin{algorithmic}[1]
\Function{Extract-Min}{$H$}
  \State $H' \gets H, p \gets$ NIL
  \State $T_m \gets T_p \gets$ NIL
  \While{$H \neq$ NIL}
    \If{$T_m =$ NIL or \Call{Key}{$H$} $<$ \Call{Key}{$T_m$}}
      \State $T_m \gets H$
      \State $T_p \gets p$
    \EndIf
    \State $p \gets H$
    \State $H \gets $ \Call{Sibling}{$H$}
  \EndWhile
  \If{$T_p \neq$ NIL}
    \State \Call{Sibling}{$T_p$} $\gets$ \Call{Sibling}{$T_m$}
  \Else
    \State $H' \gets$ \Call{Sibling}{$T_m$}
  \EndIf
  \State \Call{Sibling}{$T_m$} $\gets$ NIL
  \State \Return $(T_m, H')$
\EndFunction
\end{algorithmic}

We can implement heap sort with $pop$. First build a binomial heap from a list of elements, then repeatedly pop the smallest element.

\be
sort  = heapSort \circ fromList
\ee

Where $heapSort$ is defined as:

\be
\begin{array}{rcl}
  heapSort\ [\ ] & = & [\ ] \\
  heapSort\ H & = & k : (heapSort\ H'), \text{where}: (k, H') = pop\ H
\end{array}
\ee

Binomial heap insert and merge are bound to $O(\lg n)$ time in worst case, their amortized performance are constant time, we skip the proof.

\section{Fibonacci heap}
\label{fib-heap} \index{Fibonacci heap}

It's interesting that why the name is given as `Fibonacci heap'.
In fact, there is no direct connection from the structure design
to Fibonacci series. The inventors of `Fibonacci heap', Michael L.
Fredman and Robert E. Tarjan, utilized the property of Fibonacci series
to prove the performance time bound, so they decided to use Fibonacci
to name this data structure.\cite{CLRS}

% ================================================================
%                 Definition
% ================================================================
\subsection{Definition}

Fibonacci heap is essentially a lazy evaluated binomial heap. Note
that, it doesn't mean implementing binomial heap in lazy evaluation
settings, for instance Haskell, brings Fibonacci heap automatically.
However, lazy evaluation setting does help in realization. For example
in \cite{hackage-fibq}, presents a elegant implementation.

Fibonacci heap has excellent performance theoretically. All operations
except for pop are bound to amortized $O(1)$ time. In this section,
we'll give an algorithm different from some popular textbook\cite{CLRS}.
Most of the ideas present here are based on Okasaki's work\cite{okasaki-fibh}.

Let's review and compare the performance of binomial heap and Fibonacci
heap (more precisely, the performance goal of Fibonacci heap).

% \begin{table}
% \caption{Performance goal of Fibonacci heap}
\begin{tabular}{l | c | r}
  \hline
  operation & Binomial heap & Fibonacci heap \\
  \hline
  insertion & $O(\lg n)$ & $O(1)$ \\
  merge & $O(\lg n)$ & $O(1)$ \\
  top & $O(\lg n)$ & $O(1)$ \\
  pop & $O(\lg n)$ & amortized $O(\lg n)$ \\
  \hline
\end{tabular}
% \end{table}

Consider where is the bottleneck of inserting a new element $x$ to
binomial heap. We actually wrap $x$ as a singleton leaf and insert
this tree into the heap which is actually a forest.

During this operation, we inserted the tree in
monotonically increasing order of rank, and once the rank is equal,
recursively linking and inserting will happen, which lead to the
$O(\lg n)$ time.

As the lazy strategy, we can postpone the ordered-rank insertion and
merging operations. On the contrary, we just put the singleton
leaf to the forest. The problem is that when we try to find the
minimum element, for example the top operation, the performance
will be bad, because we need check all trees in the forest, and
there aren't only $O(\lg n)$ trees.

In order to locate the top element in constant time, we must remember
where is the tree contains the minimum element as root.

Based on this idea, we can reuse the definition of binomial tree
and give the definition of Fibonacci heap as the following Haskell
program for example.

\lstset{language=Haskell}
\begin{lstlisting}
data BiTree a = Node { rank :: Int
                     , root :: a
                     , children :: [BiTree a]}
\end{lstlisting}

The Fibonacci heap is either empty or a forest of binomial trees with
the minimum element stored in a special one explicitly.

\begin{lstlisting}
data FibHeap a = E | FH { size :: Int
                        , minTree :: BiTree a
                        , trees :: [BiTree a]}
\end{lstlisting}

For convenient purpose, we also add a size field to record how many
elements are there in a heap.

The data layout can also be defined in imperative way as the following
ANSI C code.

\lstset{language=C}
\begin{lstlisting}
struct node{
  Key key;
  struct node *next, *prev, *parent, *children;
  int degree; /* As known as rank */
  int mark;
};

struct FibHeap{
  struct node *roots;
  struct node *minTr;
  int n; /* number of nodes */
};
\end{lstlisting}

For generality, Key can be a customized type, we use integer for illustration
purpose.

\lstset{language=C}
\begin{lstlisting}
typedef int Key;
\end{lstlisting}

In this chapter, we use the circular doubly linked-list for imperative
settings to realize the Fibonacci Heap as described in \cite{CLRS}.
It makes many operations easy and fast. Note that, there are two extra
fields added. A $degree$, also known as $rank$ for a node is the number
of children of this node; Flag $mark$ is used only in decreasing key
operation. It will be explained in detail in later section.


% ================================================================
%          Basic Heap operations
% ================================================================
\subsection{Basic heap operations}
As we mentioned that Fibonacci heap is essentially binomial heap
implemented in a lazy evaluation strategy, we'll reuse many algorithms
defined for binomial heap.

\subsubsection{Insert a new element to the heap}
\index{Fibonacci Heap!insert}
Recall the insertion algorithm of binomial tree. It can be treated
as a special case of merge operation, that one heap contains only
a singleton tree.

\be
insert(H, x) = merge(H, singleton(x))
\label{eq:fib-insert}
\ee

where singleton is an auxiliary function to wrap an element to a
one-leaf-tree.

\[
singleton(x) = FibHeap(1, node(1, x, \phi), \phi)
\]

Note that function $FibHeap()$ accepts three parameters, a
size value, which is 1 for this one-leaf-tree, a special tree
which contains the minimum element as root, and a list of other
binomial trees in the forest. The meaning of function $node()$ is
as same as before, that it creates a binomial tree from a rank,
an element, and a list of children.

Insertion can also be realized directly by appending the new node
to the forest and updated the record of the tree which contains the
minimum element.

\begin{algorithmic}[1]
\Function{Insert}{$H, k$}
  \State $x \gets$ \Call{Singleton}{$k$} \Comment{Wrap $x$ to a node}
  \State append $x$ to root list of $H$
  \If{$T_{min}(H) = NIL \lor k <$ \Call{Key}{$T_{min}(H)$} }
    \State $T_{min}(H) \gets x$
  \EndIf
  \State \Call{n}{$H$} $\gets$ \Call{n}{$H$}+1
\EndFunction
\end{algorithmic}

Where function $T_{min}()$ returns the tree which contains the minimum
element at root.

The following C source snippet is a translation for this algorithm.

\lstset{language=C}
\begin{lstlisting}
struct FibHeap* insert_node(struct FibHeap* h, struct node* x){
  h = add_tree(h, x);
  if(h->minTr == NULL || x->key < h->minTr->key)
    h->minTr = x;
  h->n++;
  return h;
}
\end{lstlisting}

\begin{Exercise}
Implement the insert algorithm in your favorite imperative programming
language completely. This is also an exercise to circular doubly linked list
manipulation.
\end{Exercise}

\subsubsection{Merge two heaps}
\index{Fibonacci Heap!merge}
Different with the merging algorithm of binomial heap, we post-pone
the linking operations later. The idea is to just put all binomial
trees from each heap together, and choose one special tree which
record the minimum element for the result heap.

\be
merge(H_1, H_2) = \left \{
  \begin{array}
  {r@{\quad:\quad}l}
  H_1 & H_2 = \phi \\
  H_2 & H_1 = \phi \\
  FibHeap(s_1 + s_2, {T_1}_{min}, \{ {T_2}_{min} \} \cup \mathbb{T}_1 \cup \mathbb{T}_2) & root({T_1}_{min}) < root({T_2}_{min}) \\
  FibHeap(s_1 + s_2, {T_2}_{min}, \{ {T_1}_{min} \} \cup \mathbb{T}_1 \cup \mathbb{T}_2) & otherwise \\
  \end{array}
\right .
\ee

where $s_1$ and $s_2$ are the size of $H_1$ and $H_2$; ${T_1}_{min}$ and
${T_2}_{min}$ are the special trees with minimum element as root in $H_1$
and $H_2$ respectively; $\mathbb{T}_1 = \{{T_1}_1, {T_1}_2, ...\}$ is
a forest contains all other binomial trees in $H_1$; while $\mathbb{T}_2$
has the same meaning as $\mathbb{T}_1$ except that it represents the
forest in $H_2$. Function $root(T)$ return the root element of a binomial
tree.

Note that as long as the $\cup$ operation takes constant time, these
$merge$ algorithm is bound to $O(1)$. The following Haskell program
is the translation of this algorithm.

\lstset{language=Haskell}
\begin{lstlisting}
merge h E = h
merge E h = h
merge h1@(FH sz1 minTr1 ts1) h2@(FH sz2 minTr2 ts2)
    | root minTr1 < root minTr2 = FH (sz1+sz2) minTr1 (minTr2:ts2++ts1)
    | otherwise = FH (sz1+sz2) minTr2 (minTr1:ts1++ts2)
\end{lstlisting}

Merge algorithm can also be realized imperatively by concatenating
the root lists of the two heaps.

\begin{algorithmic}[1]
\Function{Merge}{$H_1, H_2$}
  \State $H \gets \Phi$
  \State \Call{Root}{$H$} $\gets$ \textproc{Concat}(\Call{Root}{$H_1$}, \Call{Root}{$H_2$})
  \If{\Call{Key}{$T_{min}(H_1)$} $<$ \Call{Key}{$T_{min}(H_2)$}}
    \State $T_{min}(H) \gets T_{min}(H_1)$
  \Else
    \State $T_{min}(H) \gets T_{min}(H_2)$
  \EndIf
  \Call{n}{$H$} = \Call{n}{$H_1$} + \Call{n}{$H_2$}
  \State \Return $H$
\EndFunction
\end{algorithmic}

This function assumes neither $H_1$, nor $H_2$ is empty. And it's easy
to add handling to these special cases as the following ANSI C program.

\lstset{language=C}
\begin{lstlisting}
struct FibHeap* merge(struct FibHeap* h1, struct FibHeap* h2){
  struct FibHeap* h;
  if(is_empty(h1))
    return h2;
  if(is_empty(h2))
    return h1;
  h = empty();
  h->roots = concat(h1->roots, h2->roots);
  if(h1->minTr->key < h2->minTr->key)
    h->minTr = h1->minTr;
  else
    h->minTr = h2->minTr;
  h->n = h1->n + h2->n;
  free(h1);
  free(h2);
  return h;
}
\end{lstlisting}

With $merge$ function defined, the $O(1)$ insertion algorithm is realized
as well. And we can also give the $O(1)$ time top function as below.

\be
top(H) = root(T_{min})
\ee

\begin{Exercise}
Implement the circular doubly linked list concatenation function in
your favorite imperative programming language.
\end{Exercise}

\subsubsection{Extract the minimum element from the heap (pop)}
\index{Fibonacci Heap!pop} \index{Fibonacci Heap!delete min}

The pop operation is the most complex
one in Fibonacci heap. Since we postpone the tree consolidation
in merge algorithm. We have to compensate it somewhere. Pop is
the only place left as we have defined, insert, merge, top already.

There is an elegant procedural algorithm to do the tree consolidation
by using an auxiliary array\cite{CLRS}. We'll show it later in imperative
approach section.

In order to realize the purely functional consolidation algorithm,
let's first consider a similar number puzzle.

Given a list of numbers, such as $\{2, 1, 1, 4, 8, 1, 1, 2, 4\}$, we want
to add any two values if they are same. And repeat this procedure till
all numbers are unique. The result of the example list should be
$\{8, 16\}$ for instance.

One solution to this problem will as the following.

\be
consolidate(L) = fold(meld, \phi, L)
\ee

Where $fold()$ function is defined to iterate all elements from a list,
applying a specified function to the intermediate result and each
element. it is sometimes called as {\em reducing}. Please refer to Appendix A and the
chapter of binary search tree for it.

$L=\{x_1, x_2, ..., x_n\}$, denotes a list of numbers; and we'll use
$L'=\{x_2, x_3, ..., x_n\}$ to represent the rest of the list with the
first element removed. Function $meld()$ is defined as below.

\be
meld(L, x) = \left \{
  \begin{array}
  {r@{\quad:\quad}l}
  \{ x \} & L = \phi \\
  meld(L', x+x_1) & x = x_1 \\
  \{ x \} \cup L & x < x_1 \\
  \{ x_1 \} \cup meld(L', x) & otherwise
  \end{array}
\right .
\ee

The $consolidate()$ function works as the follows. It maintains an
ordered result list $L$, contains only unique numbers, which is
initialized from an empty list $\phi$. Each time it process an
element $x$, it firstly check if the first element in $L$ is equal
to $x$, if so, it will add them together (which yields $2x$),
and repeatedly check if $2x$ is equal to the next element in $L$.
This process won't stop until either the element to be melt is
not equal to the head element in the rest of the list, or the
list becomes empty. Table \ref{tb:num-consolidate} illustrates
the process of consolidating number sequence $\{2, 1, 1, 4, 8, 1, 1, 2, 4\}$.
Column one lists the number 'scanned' one by one; Column two
shows the intermediate result, typically the new scanned number
is compared with the first number in result list. If they
are equal, they are enclosed in a pair of parentheses; The
last column is the result of meld, and it will be used as the
input to next step processing.

The Haskell program can be give accordingly.

\lstset{language=Haskell}
\begin{lstlisting}
consolidate = foldl meld [] where
    meld [] x = [x]
    meld (x':xs) x | x == x' = meld xs (x+x')
                   | x < x'  = x:x':xs
                   | otherwise = x': meld xs x
\end{lstlisting}

We'll analyze the performance of consolidation as a part of
pop operation in later section.

\begin{table}
\caption{Steps of consolidate numbers} \label{tb:num-consolidate}
\centering
\begin{tabular}{r | l | l }
  \hline
  number & intermediate result & result \\
  \hline
  2 & 2 & 2 \\
  1 & 1, 2 & 1, 2 \\
  1 & (1+1), 2 & 4 \\
  4 & (4+4) & 8 \\
  8 & (8+8) & 16 \\
  1 & 1, 16 & 1, 16 \\
  1 & (1+1), 16 & 2, 16 \\
  2 & (2+2), 16 & 4, 16 \\
  4 & (4+4), 16 & 8, 16 \\
  \hline
\end{tabular}
\end{table}

The tree consolidation is very similar to this algorithm except
it performs based on rank. The only thing we need to do is to
modify $meld()$ function a bit, so that it compare on ranks and
do linking instead of adding.

\be
meld(L, x) = \left \{
  \begin{array}
  {r@{\quad:\quad}l}
  \{ x \} & L = \phi \\
  meld(L', link(x, x_1)) & rank(x) = rank(x_1) \\
  \{ x \} \cup L & rank(x) < rank(x_1) \\
  \{ x_1 \} \cup meld(L', x) & otherwise
  \end{array}
\right .
\ee

The final consolidate Haskell program changes to the below version.

\lstset{language=Haskell}
\begin{lstlisting}
consolidate = foldl meld [] where
    meld [] t = [t]
    meld (t':ts) t | rank t == rank t' = meld ts (link t t')
                   | rank t <  rank t' = t:t':ts
                   | otherwise = t' : meld ts t
\end{lstlisting}

Figure \ref{fig:fib-meld-a} and \ref{fig:fib-meld-b} show the steps of
consolidation when processing a Fibonacci Heap contains different ranks
of trees. Comparing with table \ref{tb:num-consolidate} reveals the similarity.

\begin{figure}[htbp]
  \centering
  \subcaptionbox{Before consolidation}{\includegraphics[scale=0.5]{img/fib-meld-01}} \\
  \subcaptionbox{Step 1, 2}{\includegraphics[scale=0.5]{img/fib-meld-02}\hspace{0.1\textwidth}}
  \subcaptionbox{Step 3, 'd' is firstly linked to 'c', then repeatedly linked to 'a'.}{ \hspace{0.1\textwidth} \includegraphics[scale=0.5]{img/fib-meld-03} \hspace{0.1\textwidth}}
  \subcaptionbox{Step 4}{\includegraphics[scale=0.5]{img/fib-meld-04}}
  \caption{Steps of consolidation} \label{fig:fib-meld-a}
\end{figure}

\begin{figure}[htbp]
  \centering
  \subcaptionbox{Step 5}{\includegraphics[scale=0.5]{img/fib-meld-05}}
  \subcaptionbox{Step 6}{\includegraphics[scale=0.5]{img/fib-meld-06}} \\
  \subcaptionbox{Step 7, 8, 'r' is firstly linked to 'q', then 's' is linked to 'q'.}{\includegraphics[scale=0.5]{img/fib-meld-07}}
  \caption{Steps of consolidation} \label{fig:fib-meld-b}
\end{figure}

After we merge all binomial trees, including the special tree
record for the minimum element in root, in a Fibonacci heap, the heap
becomes a Binomial heap. And we lost the special tree, which gives
us the ability to return the top element in $O(1)$ time.

It's necessary to perform a $O(\lg n)$ time search to resume the
special tree. We can reuse the function $extractMin()$ defined for
Binomial heap.

It's time to give the final pop function for Fibonacci heap as all
the sub problems have been solved. Let $T_{min}$ denote the special
tree in the heap to record the minimum element in root; $\mathbb{T}$
denote the forest contains all the other trees except for the
special tree, $s$ represents the size of a heap, and function
$children()$ returns all sub trees except the root of a binomial
tree.

\be
deleteMin(H) =  \left \{
  \begin{array}
  {r@{\quad:\quad}l}
  \phi & \mathbb{T} = \phi \land children(T_{min})=\phi \\
  FibHeap(s-1, T'_{min}, \mathbb{T}') & otherwise
  \end{array}
\right .
\ee

Where

\[
  (T'_{min}, \mathbb{T}') = extractMin(consolidate(children(T_{min}) \cup \mathbb{T}))
\]

Translate to Haskell yields the below program.

\lstset{language=Haskell}
\begin{lstlisting}
deleteMin (FH _ (Node _ x []) []) = E
deleteMin h@(FH sz minTr ts) = FH (sz-1) minTr' ts' where
    (minTr', ts') = extractMin $ consolidate (children minTr ++ ts)
\end{lstlisting} %$

The main part of the imperative realization is similar. We cut all children of
$T_{min}$ and append them to root list, then perform consolidation to merge
all trees with the same rank until all trees are unique in term of rank.

\begin{algorithmic}[1]
\Function{Delete-Min}{$H$}
  \State $x \gets T_{min}(H)$
  \If{$x \neq NIL$}
    \For{each $y \in $ \Call{Children}{$x$}}
      \State append $y$ to root list of $H$
      \State \Call{Parent}{$y$} $\gets NIL$
    \EndFor
    \State remove $x$ from root list of $H$
    \State \Call{n}{$H$} $\gets$ \Call{n}{$H$} - 1
    \State \Call{Consolidate}{$H$}
  \EndIf
  \State \Return $x$
\EndFunction
\end{algorithmic}

Algorithm \textproc{Consolidate} utilizes an auxiliary array $A$ to do the
merge job. Array $A[i]$ is defined to store the tree with rank (degree) $i$.
During the traverse of root list, if we meet another tree of rank $i$, we
link them together to get a new tree of rank $i+1$. Next we clean $A[i]$,
and check if $A[i+1]$ is empty and perform further linking if necessary.
After we finish traversing all roots, array $A$ stores all result trees
and we can re-construct the heap from it.

\begin{algorithmic}[1]
\Function{Consolidate}{$H$}
  \State $D \gets $ \textproc{Max-Degree}(\Call{n}{$H$})
  \For{$i \gets 0$ to $D$}
    \State $A[i] \gets NIL$
  \EndFor
  \For{each $x \in$ root list of $H$}
    \State remove $x$ from root list of $H$
    \State $d \gets $ \Call{Degree}{$x$}
    \While{$A[d] \neq NIL$}\
      \State $y \gets A[d]$
      \State $x \gets $ \Call{Link}{$x, y$}
      \State $A[d] \gets NIL$
      \State $d \gets d + 1$
    \EndWhile
    \State $A[d] \gets x$
  \EndFor
  \State $T_{min}(H) \gets NIL$ \Comment{root list is NIL at the time}
  \For{$i \gets 0$ to $D$}
    \If{$A[i] \neq NIL$}
      \State append $A[i]$ to root list of $H$.
      \If{$T_{min} = NIL \lor$ \Call{Key}{$A[i]$} $<$ \Call{Key}{$T_{min}(H)$}}
        \State $T_{min}(H) \gets A[i]$
      \EndIf
    \EndIf
  \EndFor
\EndFunction
\end{algorithmic}

The only unclear sub algorithm is \textproc{Max-Degree}, which can determine
the upper bound of the degree of any node in a Fibonacci Heap. We'll delay
the realization of it to the last sub section.

Feed a Fibonacci Heap shown in Figure \ref{fig:fib-meld-a} to the above algorithm,
Figure \ref{fig:fib-cons-a}, \ref{fig:fib-cons-b} and \ref{fig:fib-cons-c}
show the result trees stored in auxiliary array $A$ in every steps.

\begin{figure}[htbp]
  \centering
  \subcaptionbox{Step 1, 2}{\includegraphics[scale=0.5]{img/fib-cons-02}}
  \subcaptionbox{Step 3, Since $A_0 \neq NIL$, 'd' is firstly linked to 'c', and clear $A_0$ to $NIL$. Again, as $A_1 \neq NIL$, 'c' is linked to 'a' and the new tree is stored in $A_2$.}{\includegraphics[scale=0.5]{img/fib-cons-03}}
  \subcaptionbox{Step 4}{\includegraphics[scale=0.5]{img/fib-cons-04}}
  \caption{Steps of consolidation} \label{fig:fib-cons-a}
\end{figure}

\begin{figure}[htbp]
  \centering
  \subcaptionbox{Step 5}{\includegraphics[scale=0.5]{img/fib-cons-05}} \\
  \subcaptionbox{Step 6}{\includegraphics[scale=0.5]{img/fib-cons-06}}
  \caption{Steps of consolidation} \label{fig:fib-cons-b}
\end{figure}

\begin{figure}[htbp]
  \centering
  \subcaptionbox{Step 7, 8, Since $A_0 \neq NIL$, 'r' is firstly linked to 'q', and the new tree is stored in $A_1$ ($A_0$ is cleared); then 's' is linked to 'q', and stored in $A_2$ ($A_1$ is cleared).}{\includegraphics[scale=0.5]{img/fib-cons-07}}
  \caption{Steps of consolidation} \label{fig:fib-cons-c}
\end{figure}


Translate the above algorithm to ANSI C yields the below program.

\lstset{language = C}
\begin{lstlisting}
void consolidate(struct FibHeap* h){
  if(!h->roots)
    return;
  int D = max_degree(h->n)+1;
  struct node *x, *y;
  struct node** a = (struct node**)malloc(sizeof(struct node*)*(D+1));
  int i, d;
  for(i=0; i<=D; ++i)
    a[i] = NULL;
  while(h->roots){
    x = h->roots;
    h->roots = remove_node(h->roots, x);
    d= x->degree;
    while(a[d]){
      y = a[d];  /* Another node has the same degree as x */
      x = link(x, y);
      a[d++] = NULL;
    }
    a[d] = x;
  }
  h->minTr = h->roots = NULL;
  for(i=0; i<=D; ++i)
    if(a[i]){
      h->roots = append(h->roots, a[i]);
      if(h->minTr == NULL || a[i]->key < h->minTr->key)
	h->minTr = a[i];
    }
  free(a);
}
\end{lstlisting}

\begin{Exercise}
Implement the remove function for circular doubly linked list in your favorite
imperative programming language.
\end{Exercise}

\subsection{Running time of pop}

In order to analyze the amortize performance of pop,
we adopt potential method. Reader can refer to \cite{CLRS} for a formal
definition. In this chapter, we only give a intuitive illustration.

Remind the gravity potential energy, which is defined as
\[
E = M \cdot g \cdot h
\]

Suppose there is a complex process, which moves the object with mass $M$
up and down, and finally the object stop at height $h'$. And if there
exists friction resistance $W_f$, We say
the process works the following power.

\[
W = M \cdot g \cdot (h' - h) + W_f
\]

\begin{figure}[htbp]
  \centering
  \includegraphics[scale=0.5]{img/potential-energy}
  \caption{Gravity potential energy.}
  \label{fig:potential-energy}
\end{figure}

Figure \ref{fig:potential-energy} illustrated this concept.

We treat the Fibonacci heap pop operation in a similar
way, in order to evaluate the cost, we firstly define the potential
$\Phi(H)$ before extract the minimum element. This potential is
accumulated by insertion and merge operations executed so far.
And after tree consolidation and
we get the result $H'$, we then calculate the new potential $\Phi(H')$.
The difference between $\Phi(H')$ and $\Phi(H)$ plus the contribution
of consolidate algorithm indicates the amortized
performance of pop.

For pop operation analysis, the potential can be defined as

\be
\Phi(H) = t(H)
\ee

Where $t(H)$ is the number of trees in Fibonacci heap forest.
We have $t(H) = 1 + length(\mathbb{T})$ for any non-empty heap.

For the $n$-nodes Fibonacci heap, suppose there is an upper bound
of ranks for all trees as $D(n)$. After consolidation, it ensures
that the number of trees in the heap forest is at most $D(n)+1$.

Before consolidation, we actually did another important thing, which
also contribute to running time, we removed the root of the minimum
tree, and concatenate all children left to the forest. So consolidate
operation at most processes $D(n)+t(H)-1$ trees.

Summarize all the above factors, we deduce the amortized cost
as below.

\be
\begin{array}{lll}
T & = & T_{consolidation} + \Phi(H') -\Phi(H) \\
  & = & O(D(n)+t(H)-1) + (D(n) + 1) - t(H) \\
  & = & O(D(n))
\end{array}
\ee

If only insertion, merge, and pop function are applied to Fibonacci
heap. We ensure that all trees are binomial trees. It is easy to
estimate the upper limit $D(n)$ is $O(\lg n)$. (Suppose the extreme
case, that all nodes are in only one Binomial tree).

However, we'll show in next sub section that, there is operation can
violate the binomial tree assumption.

\begin{Exercise}
Why the tree consolidation time is proportion to the number of trees
it processed?
\end{Exercise}

\subsection{Decreasing key}
\index{Fibonacci Heap!decrease key}
There is a special
heap operation left. It only makes sense for imperative settings.
It's about decreasing key of a certain node. Decreasing key plays
important role in some Graphic algorithms such as Minimum Spanning
tree algorithm and Dijkstra's algorithm \cite{CLRS}. In that case
we hope the decreasing key takes $O(1)$ amortized time.

However, we can't define a function like $Decrease(H, k, k')$, which
first locates a node with key $k$, then decrease $k$ to $k'$ by replacement,
and then resume the heap properties. This is because the time for
locating phase is bound to $O(n)$ time, since we don't have a pointer
to the target node.

In imperative setting, we can define the algorithm as
\textproc{Decrease-Key}($H, x, k$). Here $x$ is a node in heap $H$, which
we want to decrease its key to $k$. We needn't perform a search, as
we have $x$ at hand. It's possible to give an amortized $O(1)$ solution.

When we decreased the key of a node, if it's not a root, this operation
may violate the property Binomial tree that the key of parent is
less than all keys of children. So we need to compare the decreased key
with the parent node, and if this case happens, we can cut this node
and append it to the root list. (Remind the recursive swapping solution
for binary heap which leads to $O(\lg n)$)

\begin{figure}[htbp]
  \centering
  \setlength{\unitlength}{1cm}
  \begin{picture}(12, 7)
    \put(0, 0){\includegraphics[scale=0.7]{img/cut-fib-tree}}
    \put(6.7, 3){\line(1, 1){0.5}}
    \put(6.7, 3.5){\line(1, -1){0.5}}
  \end{picture}
  \caption{$x<y$, cut tree $x$ from its parent, and add $x$ to root list.} \label{fig:cut-fib-tree}
\end{figure}

Figure \ref{fig:cut-fib-tree} illustrates this situation. After decreasing
key of node $x$, it is less than $y$, we cut $x$ off its parent $y$, and
'past' the whole tree rooted at $x$ to root list.

Although we recover the property of that parent is less than all children,
the tree isn't any longer a Binomial tree after it losses some sub tree.
If a tree losses too many of its children because of cutting, we can't ensure
the performance of merge-able heap operations. Fibonacci Heap adds another
constraints to avoid such problem:

{\em If a node losses its second child, it is immediately cut from parent,
and added to root list}

The final \textproc{Decrease-Key} algorithm is given as below.

\begin{algorithmic}[1]
\Function{Decrease-Key}{$H, x, k$}
  \State \Call{Key}{$x$} $\gets k$
  \State $p \gets $ \Call{Parent}{$x$}
  \If{$p \neq NIL \land k < $ \Call{Key}{$p$}}
    \State \Call{Cut}{$H, x$}
    \State \Call{Cascading-Cut}{$H, p$}
  \EndIf
  \If{$k <$ \Call{Key}{$T_{min}(H)$}}
    \State $T_{min}(H) \gets x$
  \EndIf
\EndFunction
\end{algorithmic}

Where function \textproc{Cascading-Cut} uses the mark to determine
if the node is losing the second child. the node is marked after
it losses the first child. And the mark is cleared in \textproc{Cut}
function.

\begin{algorithmic}[1]
\Function{Cut}{$H, x$}
  \State $p \gets $ \Call{Parent}{$x$}
  \State remove $x$ from $p$
  \State \Call{Degree}{$p$} $\gets$ \Call{Degree}{$p$} - 1
  \State add $x$ to root list of $H$
  \State \Call{Parent}{$x$} $\gets NIL$
  \State \Call{Mark}{$x$} $\gets FALSE$
\EndFunction
\end{algorithmic}

During cascading cut process, if $x$ is marked, which means it has
already lost one child. We recursively performs cut and cascading cut
on its parent till reach to root.

\begin{algorithmic}[1]
\Function{Cascading-Cut}{$H, x$}
  \State $p \gets $ \Call{Parent}{$x$}
  \If{$p \neq NIL$}
    \If{\Call{Mark}{$x$} $= FALSE$}
      \State \Call{Mark}{$x$} $\gets TRUE$
    \Else
      \State \Call{Cut}{$H, x$}
      \State \Call{Cascading-Cut}{$H, p$}
    \EndIf
  \EndIf
\EndFunction
\end{algorithmic}

The relevant ANSI C decreasing key program is given as the following.

\lstset{language=C}
\begin{lstlisting}
void decrease_key(struct FibHeap* h, struct node* x, Key k){
  struct node* p = x->parent;
  x->key = k;
  if(p && k < p->key){
    cut(h, x);
    cascading_cut(h, p);
  }
  if(k < h->minTr->key)
    h->minTr = x;
}

void cut(struct FibHeap* h, struct node* x){
  struct node* p = x->parent;
  p->children = remove_node(p->children, x);
  p->degree--;
  h->roots = append(h->roots, x);
  x->parent = NULL;
  x->mark = 0;
}

void cascading_cut(struct FibHeap* h, struct node* x){
  struct node* p = x->parent;
  if(p){
    if(!x->mark)
      x->mark = 1;
    else{
      cut(h, x);
      cascading_cut(h, p);
    }
  }
}
\end{lstlisting}

\begin{Exercise}
Prove that \textproc{Decrease-Key} algorithm is amortized $O(1)$ time.
\end{Exercise}

\subsection{The name of Fibonacci Heap}
It's time to reveal the reason why the data structure is named
as 'Fibonacci Heap'.

There is only one undefined algorithm so far, \textproc{Max-Degree}($n$).
Which can determine the upper bound of degree for any node in a $n$ nodes
Fibonacci Heap. We'll give the proof by using Fibonacci series and
finally realize \textproc{Max-Degree} algorithm.

\begin{lemma}
\label{lemma:Fib-degree}
For any node $x$ in a Fibonacci Heap, denote $k = degree(x)$, and
$|x| = size(x)$, then
\be
  |x| \geq F_{k+2}
\ee

Where $F_k$ is Fibonacci series defined as the following.
\[
F_k = \left \{
  \begin{array}
  {r@{\quad:\quad}l}
  0 & k = 0 \\
  1 & k = 1 \\
  F_{k-1} + F_{k-2} & k \geq 2
  \end{array}
\right.
\]
\end{lemma}

\begin{proof}
Consider all $k$ children of node $x$, we denote them as $y_1, y_2, ..., y_k$
in the order of time when they were linked to $x$. Where $y_1$ is the
oldest, and $y_k$ is the youngest.

Obviously, $|y_i| \geq 0$. When we link $y_i$ to $x$, children $y_1, y_2, ..., y_{i-1}$ have already been there. And algorithm \textproc{Link} only links
nodes with the same degree. Which indicates at that time, we have

\[
  degree(y_i) = degree(x) = i - 1
\]

After that, node $y_i$ can at most
lost 1 child, (due to the decreasing key operation) otherwise, if it
will be immediately cut off and append to root list after the second
child loss. Thus we conclude

\[
degree(y_i) \geq i-2
\]

For any $i = 2, 3, ..., k$.

Let $s_k$ be the {\em minimum possible size} of node $x$, where
$degree(x) = k$. For trivial cases, $s_0 = 1$, $s_1 = 2$, and we have

\bea*{rcl}
|x| & \geq & s_k \\
    & =   & 2 + \sum_{i=2}^{k} s_{degree(y_i)} \qquad \\
    & \geq & 2 + \sum_{i=2}^{k} s_{i-2}
\eea*

We next show that $s_k > F_{k+2}$. This can be proved by induction.
For trivial cases, we have $s_0 = 1 \geq F_2 = 1$, and $s_1 = 2 \geq F_3 = 2$.
For induction case $k \geq 2$. We have

\bea*{rcl}
|x| & \geq & s_k \\
    & \geq & 2 + \sum_{i=2}^{k} s_{i-2} \\
    & \geq & 2 + \sum_{i=2}^{k} F_i \\
    & =    & 1 +  \sum_{i=0}^{k} F_i \\
\eea*

At this point, we need prove that

\be
F_{k+2} = 1 +  \sum_{i=0}^{k} F_i
\ee

This can also be proved by using induction:
\begin{itemize}
\item Trivial case, $F_2 = 1 + F_0 = 2$
\item Induction case,
\bea*{rcl}
  F_{k+2} & = & F_{k+1} + F_k \\
         & = & 1 + \sum_{i=0}^{k-1}F_i + F_k \\
         & = & 1 + \sum_{i=0}^{k} F_i
\eea*
\end{itemize}

Summarize all above we have the final result.
\be
n \geq |x| \geq F_k+2
\ee
\end{proof}

Recall the result of AVL tree, that $F_k \geq \phi^k$, where
$\phi = \frac{1+\sqrt{5}}{2}$ is the golden ratio. We also proved
that pop operation is amortized $O(\lg n)$ algorithm.

Based on this result. We can define Function $MaxDegree$ as the following.

\be
  MaxDegree(n) = 1 + \lfloor \log_{\phi} n \rfloor
\ee

The imperative \textproc{Max-Degree} algorithm can also be realized by
using Fibonacci sequences.

\begin{algorithmic}[1]
\Function{Max-Degree}{$n$}
  \State $F_0 \gets 0$
  \State $F_1 \gets 1$
  \State $k \gets 2$
  \Repeat
    \State $F_k \gets F_{k_1} + F_{k_2}$
    \State $k \gets k+1$
  \Until{$F_k < n$}
  \State \Return $k-2$
\EndFunction
\end{algorithmic}

Translate the algorithm to ANSI C given the following program.

\lstset{language=C}
\begin{lstlisting}
int max_degree(int n){
  int k, F;
  int F2 = 0;
  int F1 = 1;
  for(F=F1+F2, k=2; F<n; ++k){
    F2 = F1;
    F1 = F;
    F = F1 + F2;
  }
  return k-2;
}
\end{lstlisting}

% ================================================================
%                 Pairing Heaps
% ================================================================

\section{Pairing Heaps}
\label{pairing-heap} \index{Pairing heap}
Although Fibonacci Heaps provide excellent performance theoretically,
it is complex to realize. People find that the constant behind the
big-O is big. Actually, Fibonacci Heap is more significant in theory
than in practice.

In this section, we'll introduce another solution, Pairing heap,
which is one of the best heaps ever known in terms of performance.
Most operations including insertion, finding minimum element (top),
merging are all bounds to $O(1)$ time, while deleting minimum element (pop)
is conjectured to amortized $O(\lg n)$ time \cite{pairing-heap}
\cite{okasaki-book}. Note that this is still
a conjecture for 15 years by the time I write this chapter. Nobody has been
proven it although there are much experimental data support the
$O(\lg n)$ amortized result.

Besides that, pairing heap is simple. There exist both elegant
imperative and functional implementations.

% ================================================================
%                 Definition
% ================================================================
\subsection{Definition}
\index{Pairing heap!definition}

Both Binomial Heaps and Fibonacci Heaps are realized with forest.
While a pairing heaps is essentially a K-ary tree. The minimum element
is stored at root. All other elements are stored in sub trees.

The following Haskell program defines pairing heap.

\lstset{language=Haskell}
\begin{lstlisting}
data PHeap a = E | Node a [PHeap a]
\end{lstlisting}

This is a recursive definition, that a pairing heap is either empty
or a K-ary tree, which is consist of a root node, and a list of sub trees.

Pairing heap can also be defined in procedural languages, for example
ANSI C as below. For illustration purpose, all heaps we mentioned later
are minimum-heap, and we assume the type of key is integer \footnote{We
can parametrize the key type with C++ template, but this is beyond
our scope, please refer to the example programs along with
this book}. We use same linked-list based left-child, right-sibling
approach (aka, binary tree representation\cite{CLRS}).

\lstset{language=C}
\begin{lstlisting}
typedef int Key;

struct node{
  Key key;
  struct node *next, *children, *parent;
};
\end{lstlisting}

Note that the parent field does only make sense for decreasing key
operation, which will be explained later on. we can omit it for the
time being.


% ================================================================
%          Basic Heap operations
% ================================================================
\subsection{Basic heap operations}
In this section, we first give the merging operation for pairing
heap, which can be used to realize insertion. Merging, insertion,
and finding the minimum element are relative trivial compare to
the extracting minimum element operation.

\subsubsection{Merge, insert, and find the minimum element (top)}
\index{Pairing heap!insert} \index{Pairing heap!top}
\index{Pairing heap!find min}
The idea of merging is similar to the linking algorithm we shown
previously for Binomial heap. When we merge two pairing heaps, there
are two cases.

\begin{itemize}
\item Trivial case, one heap is empty, we simply return the other
heap as the result;

\item Otherwise, we compare the root element of the two heaps, make
the heap with bigger root element as a new children of the other.
\end{itemize}

Let $H_1$, and $H_2$ denote the two heaps, $x$ and $y$ be the root
element of $H_1$ and $H_2$ respectively. Function $Children()$
returns the children of a K-ary tree. Function $Node()$ can
construct a K-ary tree from a root element and a list of children.

\be
merge(H_1, H_2) = \left \{
  \begin{array}
  {r@{\quad:\quad}l}
  H_1 & H_2 = \phi \\
  H_2 & H_1 = \phi \\
  Node(x, \{H_2\} \cup Children(H_1)) & x < y \\
  Node(y, \{H_1\} \cup Children(H_2)) & otherwise
  \end{array}
\right .
\ee

Where
\[
\begin{array}{l}
x = Root(H_1) \\
y = Root(H_2)
\end{array}
\]

It's obviously that merging algorithm is bound to $O(1)$ time
\footnote{Assume $\cup$ is constant time operation, this is true
for linked-list settings, including 'cons' like operation in
functional programming languages.}.
The $merge$ equation can be translated to the following Haskell program.

\lstset{language=Haskell}
\begin{lstlisting}
merge h E = h
merge E h = h
merge h1@(Node x hs1) h2@(Node y hs2) =
    if x < y then Node x (h2:hs1) else Node y (h1:hs2)
\end{lstlisting}

Merge can also be realized imperatively. With left-child, right
sibling approach, we can just link the heap, which is in fact a
K-ary tree, with larger key as the first new child of the other.
This is constant time operation as described below.

\begin{algorithmic}[1]
\Function{Merge}{$H_1, H_2$}
  \If{$H_1 = $ NIL}
    \State \Return $H_2$
  \EndIf
  \If{$H_2 = $ NIL}
    \State \Return $H_1$
  \EndIf
  \If{\Call{Key}{$H_2$} $<$ \Call{Key}{$H_1$}}
    \State \Call{Exchange}{$H_1 \leftrightarrow H_2$}
  \EndIf
  \State Insert $H_2$ in front of \Call{Children}{$H_1$}
  \State \Call{Parent}{$H_2$} $\gets H_1$
  \State \Return $H_1$
\EndFunction
\end{algorithmic}

Note that we also update the parent field accordingly. The ANSI C
example program is given as the following.

\lstset{language=C}
\begin{lstlisting}
struct node* merge(struct node* h1, struct node* h2) {
  if (h1 == NULL)
    return h2;
  if (h2 == NULL)
    return h1;
  if (h2->key < h1->key)
    swap(&h1, &h2);
  h2->next = h1->children;
  h1->children = h2;
  h2->parent = h1;
  h1->next = NULL; /*Break previous link if any*/
  return h1;
}
\end{lstlisting}

Where function swap() is defined in a similar way as Fibonacci Heap.

With merge defined, insertion can be realized as same as Fibonacci Heap
in Equation \ref{eq:fib-insert}. Definitely it's $O(1)$ time operation.
As the minimum element is always stored in root, finding it is trivial.

\be
top(H) = Root(H)
\ee

Same as the other two above operations, it's bound to $O(1)$ time.

\subsubsection{Decrease key of a node}
\index{Pairing heap!decrease key}
There is another operation, to decrease key of a given node,
which only makes sense in imperative settings as we explained in Fibonacci
Heap section.

The solution is simple, that we can cut the node with the new smaller
key from it's parent along with all its children. Then merge it again
to the heap. The only special case is that if the given node is the
root, then we can directly set the new key without doing anything else.

The following algorithm describes this procedure for a given node $x$, with
new key $k$.

\begin{algorithmic}[1]
\Function{Decrease-Key}{$H, x, k$}
  \State \Call{Key}{$x$} $\gets k$
  \If{\Call{Parent}{$x$} $\neq$ NIL}
    \State Remove $x$ from \textproc{Children}(\Call{Parent}{$x$})
    \Call{Parent}{$x$} $\gets$ NIL
    \State \Return \Call{Merge}{$H, x$}
  \EndIf
  \State \Return $H$
\EndFunction
\end{algorithmic}

The following ANSI C program translates this algorithm.

\lstset{language=C}
\begin{lstlisting}
struct node* decrease_key(struct node* h, struct node* x, Key key) {
  x->key = key; /* Assume key <= x->key */
  if(x->parent) {
    x->parent->children = remove_node(x->parent->children, x);
    x->parent = NULL;
    return merge(h, x);
  }
  return h;
}
\end{lstlisting}

\begin{Exercise}
Implement the program of removing a node from the children of its
parent in your favorite imperative programming language. Consider
how can we ensure the overall performance of decreasing key is
O(1) time? Is left-child, right sibling approach enough?
\end{Exercise}

\subsubsection{Delete the minimum element from the heap (pop)}
\index{Pairing heap!pop} \index{Pairing heap!delete min}
Since the minimum element is always stored at root, after delete it
during popping, the rest things left are all sub-trees. These trees
can be merged to one big tree.

\be
  pop(H) = mergePairs(Children(H))
\ee

Pairing Heap uses a special approach that it merges every two sub-trees
from left to right in pair. Then
merge these paired results from right to left which forms a final
result tree. The name of `Pairing Heap' comes from the characteristic
of this pair-merging.

Figure \ref{fig:merge-pairs} and \ref{fig:merge-right} illustrate the procedure of pair-merging.

\begin{figure}[htbp]
  \centering
  \subcaptionbox{A pairing heap before pop.}{\includegraphics[scale=0.5]{img/pairing-hp}} \\
  \subcaptionbox{After root element 2 being removed, there are 9 sub-trees left.}{\includegraphics[scale=0.5]{img/pairs}} \\
  \subcaptionbox{Merge every two trees in pair, note that there are odd number trees, so the last one needn't merge.}{\includegraphics[scale=0.5]{img/pairs-merge}} \\
  \caption{Remove the root element, and merge children in pairs.} \label{fig:merge-pairs}
\end{figure}

\begin{figure}[htbp]
  \centering
  \subcaptionbox{Merge tree with 9, and tree with root 6.}{\hspace{0.2\textwidth}\includegraphics[scale=0.5]{img/right-merge-1}\hspace{0.2\textwidth}}
  \subcaptionbox{Merge tree with root 7 to the result.}{\hspace{0.1\textwidth}\includegraphics[scale=0.5]{img/right-merge-2}\hspace{0.1\textwidth}} \\
  \subcaptionbox{Merge tree with root 3 to the result.}{\hspace{0.1\textwidth}\includegraphics[scale=0.5]{img/right-merge-3}\hspace{0.1\textwidth}}
  \subcaptionbox{Merge tree with root 4 to the result.}{\hspace{0.1\textwidth}\includegraphics[scale=0.5]{img/right-merge-4}\hspace{0.1\textwidth}}
  \caption{Steps of merge from right to left.} \label{fig:merge-right}
\end{figure}

The recursive pair-merging solution is quite similar to the bottom up
merge sort\cite{okasaki-book}. Denote the children of a pairing
heap as $A$, which is a list of trees of $\{ T_1, T_2, T_3, ..., T_m\}$
for example. The $mergePairs()$ function can be given as below.

\be
mergePairs(A) = \left \{
  \begin{array}
  {r@{\quad:\quad}l}
  \Phi & A = \Phi \\
  T_1 & A = \{ T_1 \} \\
  merge(merge(T_1, T_2), mergePairs(A')) & otherwise
  \end{array}
\right .
\ee

where

\[
A' = \{ T_3, T_4, ..., T_m\}
\]

is the rest of the children without the first two trees.

The relative Haskell program of popping is given as the following.

\lstset{language=Haskell}
\begin{lstlisting}
deleteMin (Node _ hs) = mergePairs hs where
    mergePairs [] = E
    mergePairs [h] = h
    mergePairs (h1:h2:hs) = merge (merge h1 h2) (mergePairs hs)
\end{lstlisting}

The popping operation can also be explained in the following
procedural algorithm.

\begin{algorithmic}[1]
\Function{Pop}{$H$}
  \State $L \gets NIL$
  \For{every 2 trees $T_x$, $T_y \in$ \Call{Children}{$H$} from left to right}
    \State Extract $x$, and $y$ from \Call{Children}{$H$}
    \State $T \gets $ \Call{Merge}{$T_x, T_y$}
    \State Insert $T$ at the beginning of $L$
  \EndFor
  \State $H \gets $ \Call{Children}{$H$} \Comment{$H$ is either $NIL$ or one tree.}
  \For{$\forall T \in L$ from left to right}
    \State $H \gets $ \Call{Merge}{$H, T$}
  \EndFor
  \State \Return $H$
\EndFunction
\end{algorithmic}

Note that $L$ is initialized as an empty linked-list, then the algorithm
iterates every two trees in pair in the children of the K-ary tree, from
left to right, and performs merging, the result is inserted at the beginning
of $L$. Because we insert to front end, so when we traverse $L$ later on,
we actually process from right to left. There may be odd number of sub-trees
in $H$, in that case, it will leave one tree after pair-merging. We
handle it by start the right to left merging from this left tree.

Below is the ANSI C program to this algorithm.

\lstset{language=C}
\begin{lstlisting}
struct node* pop(struct node* h) {
  struct node *x, *y, *lst = NULL;
  while ((x = h->children) != NULL) {
    if ((h->children = y = x->next) != NULL)
      h->children = h->children->next;
    lst = push_front(lst, merge(x, y));
  }
  x = NULL;
  while((y = lst) != NULL) {
    lst = lst->next;
    x = merge(x, y);
  }
  free(h);
  return x;
}
\end{lstlisting}

The pairing heap pop operation is conjectured to be amortized $O(\lg n)$
time \cite{pairing-heap}.

\begin{Exercise}
Write a program to insert a tree at the beginning of a linked-list
in your favorite imperative programming language.
\end{Exercise}

\subsubsection{Delete a node}
\index{Pairing heap!delete}
We didn't mention delete in Binomial heap or Fibonacci Heap. Deletion
can be realized by first decreasing key to minus infinity ($-\infty$), then
performing pop. In this section, we present another solution for
delete node.

The algorithm is to define the function $delete(H, x)$, where $x$ is
a node in a pairing heap $H$ \footnote{Here the semantic of $x$ is a
reference to a node.}.

If $x$ is root, we can just perform a pop operation. Otherwise, we
can cut $x$ from $H$, perform a pop on $x$, and then merge the pop
result back to $H$. This can be described as the following.

\be
delete(H, x) = \left \{
  \begin{array}
  {r@{\quad:\quad}l}
  pop(H) & x \quad \text{is root of} \quad H \\
  merge(cut(H, x), pop(x)) & otherwise
  \end{array}
\right .
\ee

As delete algorithm uses pop, the performance is conjectured to be
amortized $O(\lg n)$ time.

\begin{Exercise}
\begin{itemize}
\item Write procedural pseudo code for delete algorithm.

\item Write the delete operation in your favorite imperative programming
language

\item Consider how to realize delete in purely functional setting.
\end{itemize}
\end{Exercise}

% ================================================================
%                 Short summary
% ================================================================
\section{Notes and short summary}

In this chapter, we extend the heap implementation from binary tree to
more generic approach. Binomial heap and Fibonacci heap use Forest of
K-ary trees as under ground data structure, while Pairing heap use
a K-ary tree to represent heap. It's a good point to post pone some
expensive operation, so that the over all amortized performance is
ensured. Although Fibonacci Heap gives good performance in theory, the
implementation is a bit complex. It was removed in some latest textbooks.
We also present pairing heap, which is easy to realize and have good
performance in practice.

The elementary tree based data structures are all introduced in this
book. There are still many tree based data structures which we can't
covers them all and skip here. We encourage the reader to refer to
other textbooks about them. From next chapter, we'll introduce generic
sequence data structures, array and queue.

% ================================================================
%                 Appendix
% ================================================================

\begin{thebibliography}{99}

\bibitem{K-ary-tree}
K-ary tree, Wikipedia. \url{https://en.wikipedia.org/wiki/K-ary_tree}

\bibitem{CLRS}
Thomas H. Cormen, Charles E. Leiserson, Ronald L. Rivest and Clifford Stein. ``Introduction to Algorithms, Second Edition''. The MIT Press, 2001. ISBN: 0262032937.

\bibitem{okasaki-book}
Chris Okasaki. ``Purely Functional Data Structures''. Cambridge university press, (July 1, 1999), ISBN-13: 978-0521663502

\bibitem{wiki-pascal-triangle}
Wikipedia, ``Pascal's triangle''. \url{https://en.wikipedia.org/wiki/Pascal's_triangle}

\bibitem{hackage-fibq}
Hackage. ``An alternate implementation of a priority queue based on a Fibonacci heap.'', \url{http://hackage.haskell.org/packages/archive/pqueue-mtl/1.0.7/doc/html/src/Data-Queue-FibQueue.html}

\bibitem{okasaki-fibh}
Chris Okasaki. ``Fibonacci Heaps.'' \url{http://darcs.haskell.org/nofib/gc/fibheaps/orig}

\bibitem{pairing-heap}
Michael L. Fredman, Robert Sedgewick, Daniel D. Sleator, and Robert E. Tarjan. ``The Pairing Heap: A New Form of Self-Adjusting Heap'' Algorithmica (1986) 1: 111-129.

\end{thebibliography}

\ifx\wholebook\relax \else
\end{document}
\fi
