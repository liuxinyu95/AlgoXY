\ifx\wholebook\relax \else
% ------------------------

\documentclass[UTF8]{article}
%------------------- Other types of document example ------------------------
%
%\documentclass[twocolumn]{IEEEtran-new}
%\documentclass[12pt,twoside,draft]{IEEEtran}
%\documentstyle[9pt,twocolumn,technote,twoside]{IEEEtran}
%
%-----------------------------------------------------------------------------
%
% loading packages
%

\RequirePackage{ifpdf}
\RequirePackage{ifxetex}

%
%
\ifpdf
  \RequirePackage[pdftex,%
       bookmarksnumbered,%
              colorlinks,%
          linkcolor=blue,%
              hyperindex,%
        plainpages=false,%
       pdfstartview=FitH]{hyperref}
\else\ifxetex
  \RequirePackage[bookmarksnumbered,%
               colorlinks,%
           linkcolor=blue,%
               hyperindex,%
         plainpages=false,%
        pdfstartview=FitH]{hyperref}
\else
  \RequirePackage[dvipdfm,%
        bookmarksnumbered,%
               colorlinks,%
           linkcolor=blue,%
               hyperindex,%
         plainpages=false,%
        pdfstartview=FitH]{hyperref}
\fi\fi
%\usepackage{hyperref}

% other packages
%--------------------------------------------------------------------------
\usepackage{graphicx, color}
\usepackage{subfig}
\usepackage{tikz}
\usetikzlibrary{matrix,positioning}

\usepackage{amsmath, amsthm, amssymb} % for math
\usepackage{exercise} % for exercise
\usepackage{import} % for nested input

%
% for programming
%
\usepackage{verbatim}
\usepackage{listings}
%\usepackage{algorithmic} %old version; we can use algorithmicx instead
%\usepackage[plain]{algorithm} %remove rule (horizontal line on top/below the algorithm
\usepackage{algorithm} %to remove rules change to \usepackage[plain]{algorithm}
%\usepackage{algorithm2e}
\usepackage[noend]{algpseudocode} %for pseudo code, include algorithmicsx automatically
\usepackage{appendix}
\usepackage{makeidx} % for index support
\usepackage{titlesec}

\usepackage[cm-default]{fontspec}
\usepackage{xunicode}
%\usepackage{fontenc}
\usepackage{textcomp}

% detect and select Chinese font
% ------------------------------
% the following cmd can list all availabe Chinese fonts in host.
% fc-list :lang=zh
\def\myfont{STSong}  % Under Mac OS X
\def\linuxfallback{WenQuanYi Micro Hei} % Under Linux
\def\winfallback{SimSun} % Under Windows
\suppressfontnotfounderror1 % Avoid setting exit code (error level) to break make process
\count255=\interactionmode
\batchmode
\font\foo="\myfont"\space at 10pt
\ifx\foo\nullfont
  \font\foo = "\linuxfallback"\space at 10pt
  \ifx\foo\nullfont
    \font\foo = "\winfallback"\space at 10pt
    \ifx\foo\nullfont
      \errorstopmode
      \errmessage{no suitable Chinese font found}
    \else
      \let\myfont=\winfallback % Windows
    \fi
  \else
    \let\myfont=\linuxfallback % Linux
  \fi
\fi
\interactionmode=\count255
\setmainfont[Mapping=tex-text]{\myfont}
\setmonofont{Monaco}   % 英文等宽字体

\XeTeXlinebreaklocale "zh"  % to solve the line breaking issue
\XeTeXlinebreakskip = 0pt plus 1pt minus 0.1pt

\titleformat{\paragraph}
{\normalfont\normalsize\bfseries}{\theparagraph}{1em}{}
\titlespacing*{\paragraph}
{0pt}{3.25ex plus 1ex minus .2ex}{1.5ex plus .2ex}

\lstdefinelanguage{Smalltalk}{
  morekeywords={self,super,true,false,nil,thisContext}, % This is overkill
  morestring=[d]',
  morecomment=[s]{"}{"},
  alsoletter={\#:},
  escapechar={!},
  literate=
    {BANG}{!}1
    {UNDERSCORE}{\_}1
    {\\st}{Smalltalk}9 % convenience -- in case \st occurs in code
    % {'}{{\textquotesingle}}1 % replaced by upquote=true in \lstset
    {_}{{$\leftarrow$}}1
    {>>>}{{\sep}}1
    {^}{{$\uparrow$}}1
    {~}{{$\sim$}}1
    {-}{{\sf -\hspace{-0.13em}-}}1  % the goal is to make - the same width as +
    %{+}{\raisebox{0.08ex}{+}}1		% and to raise + off the baseline to match -
    {-->}{{\quad$\longrightarrow$\quad}}3
	, % Don't forget the comma at the end!
  tabsize=2
}[keywords,comments,strings]

% for literate Haskell code
\lstdefinestyle{Haskell}{
  flexiblecolumns=false,
  basewidth={0.5em,0.45em},
  morecomment=[l]--,
  literate={+}{{$+$}}1 {/}{{$/$}}1 {*}{{$*$}}1 {=}{{$=$}}1
           {>}{{$>$}}1 {<}{{$<$}}1 {\\}{{$\lambda$}}1
           {\\\\}{{\char`\\\char`\\}}1
           {->}{{$\rightarrow$}}2 {>=}{{$\geq$}}2 {<-}{{$\leftarrow$}}2
           {<=}{{$\leq$}}2 {=>}{{$\Rightarrow$}}2
           {\ .}{{$\circ$}}2 {\ .\ }{{$\circ$}}2
           {>>}{{>>}}2 {>>=}{{>>=}}2
           {|}{{$\mid$}}1
}

\lstloadlanguages{C, C++, Lisp, Haskell, Python, Smalltalk}

\lstset{
  basicstyle=\small\ttfamily,
  commentstyle=\rmfamily,
  texcl=true,
  showstringspaces = false,
  upquote=true,
  flexiblecolumns=false
}

% ======================================================================

\def\BibTeX{{\rm B\kern-.05em{\sc i\kern-.025em b}\kern-.08em
    T\kern-.1667em\lower.7ex\hbox{E}\kern-.125emX}}

%
% mathematics
%
\newcommand{\be}{\begin{equation}}
\newcommand{\ee}{\end{equation}}
\newcommand{\bmat}[1]{\left( \begin{array}{#1} }
\newcommand{\emat}{\end{array} \right) }
\newcommand{\VEC}[1]{\mbox{\boldmath $#1$}}

% numbered equation array
\newcommand{\bea}{\begin{eqnarray}}
\newcommand{\eea}{\end{eqnarray}}

% equation array not numbered
\newcommand{\bean}{\begin{eqnarray*}}
\newcommand{\eean}{\end{eqnarray*}}

\newtheorem{theorem}{Theorem}[section]
\newtheorem{lemma}[theorem]{引理}
\newtheorem{proposition}[theorem]{Proposition}
\newtheorem{corollary}[theorem]{Corollary}

% 中文书籍设置
% ====================================
\renewcommand\contentsname{目\ 录}
%\renewcommand\listfigurename{插图目录}
%\renewcommand\listtablename{表格目录}
\renewcommand\figurename{图}
\renewcommand\tablename{表}
\renewcommand\proofname{证明}
\renewcommand\ExerciseName{练习}
%\renewcommand{\algorithmcfname}{算法}

\ifx\wholebook\relax
\renewcommand\bibname{参\ 考\ 文\ 献}                    %book类型
%\newtheorem{Definition}[Theorem]{定义}
\newtheorem{Theorem}{定理}[chapter]
\newtheorem{example}{例题}[chapter]
\else
\renewcommand\refname{参\ 考\ 文\ 献}
\fi

%\setcounter{secnumdepth}{4}
\titleformat{\chapter}
  {\normalfont\bfseries\Large}
  {第\arabic{chapter}章}
  {12pt}{\Large}
%% \titleformat{\subsection}
%%   {\normalfont\bfseries\large}
%%   {\CJKnumber{\arabic{subsection}}、}
%%   {12pt}{\large}
%% \titleformat{\subsubsection}
%%   {\normalfont\bfseries\normalsize}
%%   {\arabic{subsubsection}.}
%%   {12pt}{\normalsize}

%\renewcommand{\baselinestretch}{1.5}                        %文章行间距为1.5倍。

\setcounter{tocdepth}{4}
\setcounter{secnumdepth}{4}


\setcounter{page}{1}

\begin{document}

%--------------------------

% ================================================================
%                 COVER PAGE
% ================================================================

\title{二项式堆,斐波那契堆和pairing堆}

\author{刘新宇
\thanks{{\bfseries 刘新宇 } \newline
  Email: liuxinyu95@gmail.com \newline}
  }

\maketitle
\fi

\markboth{二项式堆、斐波那契堆Binomial heap, Fibonacci heap, and pairing heap}{Elementary Algorithms}

\ifx\wholebook\relax
\chapter{二项式堆,斐波那契堆和pairing堆}
\numberwithin{Exercise}{chapter}
\fi

% ================================================================
%                 Introduction
% ================================================================
\section{简介}
\label{introduction}

在此前的章节中,我们看到通用的堆可以由各种不同的数据结构来实现。我们介绍了用二叉树实现的各种堆。

将二叉树进行扩展可以得到$K$叉树\cite{K-ary-tree}。本章中,我们首先介绍二项式堆,它由$K$叉树的森林组成。二项式堆除了用常数时间$O(1)$获取最小值外,其他所有操作的性能都达到了$O(\lg n)$。

如果延迟某些二项式堆的操作,就可以得到斐波那契堆。我们介绍的binary堆都最少需要$O(\lg n)$时间来实现合并,而斐波那契堆可以将合并操作提高到常数时间$O(1)$。这对于图算法很重要。实际上,斐波那契堆的大部份操作的分摊性能都是常数时间$O(1)$的,只有弹出操作为$O(\lg n)$。

最后,我们将介绍pairing堆。它在实际中拥有最好的性能。但是迄今为止,它的性能还是个猜想,没有得到最终的证明。


% ================================================================
%                 Binomial heap
% ================================================================
\section{二项式堆}
\label{binomail-heap} \index{二项式堆}


% ================================================================
%                 Definition
% ================================================================
\subsection{定义}

二项式堆比大部份的binary堆都复杂,但是合并操作的性能很好,可以达到$O(\lg n)$。一个二项式堆包含一组二项式树。

\subsubsection{二项式树}
\label{Binomial tree} \index{二项式树}

为了了解“二项式树”名字的由来,我们先看一下著名的帕斯卡三角形(中国称为“贾宪”三角形以纪念古代中国的数学家贾宪(1010-1070))\cite{wiki-pascal-triangle}。

\begin{verbatim}
    1
   1 1
  1 2 1
 1 3 3 1
1 4 6 4 1
...
\end{verbatim}

每行的数字都是二项式系数。由很多方法可以获得一系列二项式系数,其中一种是使用递归组合。同样,二项式数也可以用类似的方法定义:

\begin{itemize}
\item rank为0的二项式树只有一个根节点;
\item rank为$n$的二项式树包含两棵rank为$n-1$的二项式树,两棵树中,根节点元素较大的一棵被链接为另一棵最左侧的子树。
\end{itemize}

我们记rank为0的二项式树为$B_0$,rank为$n$的二项式树为$B_n$。

图\ref{fig:link-bitree}描述了$B_0$,以及如何将两棵$B_{n-1}$链接成$B_n$。

\begin{figure}[htbp]
  \centering
  \subfloat[一棵$B_0$树]{\hspace{0.1\textwidth}\includegraphics[scale=0.5]{img/b0tree.ps}\hspace{0.1\textwidth}} \\
  \subfloat[将两棵$B_{n-1}$树链接成一棵$B_n$树]{\includegraphics[scale=0.5]{img/link-bitree.ps}}
  \caption{二项式树的递归定义} \label{fig:link-bitree}
\end{figure}

使用这一递归定义,我们可以画出rank分别为0、1、2……的各个二项式树形式,如图\ref{fig:bitree-forms}所示。

\begin{figure}[htbp]
  \centering
  \subfloat[$B_0$树]{\hspace{0.05\textwidth}\includegraphics[scale=0.5]{img/b0tree.ps}\hspace{0.05\textwidth}}
  \subfloat[$B_1$树]{\hspace{0.05\textwidth}\includegraphics[scale=0.5]{img/b1tree.ps}\hspace{0.05\textwidth}}
  \subfloat[$B_2$树]{\includegraphics[scale=0.5]{img/b2tree.ps}}
  \subfloat[$B_3$树]{\includegraphics[scale=0.5]{img/b3tree.ps}} \\
  \subfloat[$B_4$树]{\includegraphics[scale=0.5]{img/b4tree.ps}...}
  \caption{rank为0、1、2、3、4……的二项式树} \label{fig:bitree-forms}
\end{figure}

观察这些二项式树可以发现一些有趣的性质。对于任意rank为$n$的二项式树,每行的节点数目恰好是二项式系数。

例如rank为4的二项式树,第一层根有一个节点;第二层有4个节点;第三层有6个节点;第四层有4个节点;第五层有1个节点。它们恰好是帕斯卡三角形的第5行:1、4、6、4、1。这就是二项式树名字的由来。

另外一个有趣的性质是,一棵rank为$n$的二项式树中的总节点数为$2^n$。我们可以直接用二项式定理或者用递归定义来证明它。

\subsubsection{二项式堆}
\label{Binomial heap} \index{二项式堆!定义}

利用二项式树,我们可以给出二项式堆的定义。一个二项式堆是一组二项式树(或称为一个二项式树森林),它满足如下性质:

\begin{itemize}
\item 堆中的每棵树都满足\underline{堆性质},即任意节点的key都大于等于父节点。这里使用了最小堆。也可以使用最大堆,但需要将条件变为“小于等于”。简单起见,本章仅讨论最小堆。所有内容都可以相应地通过改变比较条件,变为最大堆。
\item 堆中最多有一棵二项式树的rank为$r$。换言之,堆中任何两棵二项式树的rank都不同。
\end{itemize}

这一定义直接导致了一个重要的结果。对于含有$n$个元素的二项式堆,如果将$n$转换为二进制数$a_0, a_1, a_2, ..., a_m$,其中$a_0$是最低位(LSB),$a_m$是最高位(MSB)。对于任意$0 \leq i \leq m$,若$a_i=0$,则堆中不存在rank为$i$的二项式树;若$a_i = 1$,则堆中一定含有一棵rank为$i$的二项式树。

例如,某棵二项式堆含有5个元素,因为5的二进制为“(LSB)101(MSB)”,因此这个堆中含有两棵二项式树,一棵的rank为0,另外一棵的rank为2。

图\ref{fig:bheap2}所示的二项式堆含有19个节点,因为19的二进制为“(LSB)11001(MSB)”,因此含有一棵$B_0$树、一棵$B_1$树和一棵$B_4$树。

\begin{figure}[htbp]
  \centering
  \includegraphics[scale=0.5]{img/bheap2.ps}
  \caption{含有19个元素的二项式堆} \label{fig:bheap2}
\end{figure}

\subsubsection{数据布局}
\index{left child, right sibling}

在命令式环境,有两种方法可以定义$K$叉树。一种是使用“左侧孩子,右侧兄弟”(left-child, right-sibling)的方法\cite{CLRS}。好处是这种定义和典型的二叉树结构一致。每个节点包含两个field,左侧field和右侧field。我们使用左侧的field指向节点的第一棵子树,用右侧field指向此节点的兄弟节点。所有的兄弟节点被串联成一个单向链表。如图\ref{fig:lcrs}所示。

\begin{figure}[htbp]
  \centering
  \includegraphics[scale=0.5]{img/lcrs.ps}
  \caption{“左侧孩子,右侧兄弟”的例子。$R$为根节点,它没有兄弟节点,因此它的右侧指向空$NIL$,$C_1, C_2, ..., C_n$为$R$的子节点。$R$的左侧链接到$C_1$,其他$C_1$的兄弟节点依次向右链接起来。$C_2', ..., C_m'$为$C_1$的子节点。} \label{fig:lcrs}
\end{figure}

另外一种方法是使用容器类数据结构,例如使用数组或者链表来存储一个节点的所有子节点。

因为二项式树的rank很重要,我们将其也定义为一个field。

如果“左侧孩子,右侧兄弟”方法,对应的二项式树定义如下\footnote{相关的C例子程序也可以随本书一起获得}:

\lstset{language=Python}
\begin{lstlisting}
class BinomialTree:
    def __init__(self, x = None):
        self.rank = 0
        self.key = x
        self.parent = None
        self.child = None
        self.sibling = None
\end{lstlisting}

当使用一个元素key来构造树时,我们创建一个叶子节点,其rank为0,其他field都为空。

下面的例子程序使用列表来存储子节点,相应的定义如下:

\begin{lstlisting}
class BinomialTree:
    def __init__(self, x = None):
        self.rank = 0
        self.key = x
        self.parent = None
        self.children = []
\end{lstlisting}

在纯函数式环境中,例如Haskell,我们可以将二项式树定义如下:

\lstset{language=Haskell}
\begin{lstlisting}
data BiTree a = Node { rank :: Int
                     , root :: a
                     , children :: [BiTree a]}
\end{lstlisting}

而二项式堆被定义为二项式树的列表(森林),其中的树按照rank单调递增排序。并且符合一个额外的限制:没有任何两个树的rank相等。

\begin{lstlisting}
type BiHeap a = [BiTree a]
\end{lstlisting}

% ================================================================
%                 Basic heap operation
% ================================================================
\subsection{基本的堆操作}

\subsubsection{树的链接}
\index{二项式堆!链接}

为了实现基本的堆操作如弹出、插入,我们需要先实现如何将两棵rank一样的树链接成一棵较大的树。根据二项式树的定义,以及根必须保存最小值的堆性质,需要先比较两棵树的根节点,选取较小的一个作为新的根,然后将另一棵树插入到其他子树的前面。设函数$Key(T)$、$Children(T)$和$Rank(T)$分别访问树的key,子树和rank。

\be
link(T_1, T_2) = \left \{
  \begin{array}
  {r@{\quad:\quad}l}
  node(r+1, x, \{ T_2 \} \cup C_1) & x < y \\
  node(r+1, y, \{ T_1 \} \cup C_2) & otherwise
  \end{array}
\right .
\label{eq:link}
\ee

其中

\[
  \begin{array}{l}
  x = Key(T_1) \\
  y = Key(T_2) \\
  r = Rank(T_1) = Rank(T_2) \\
  C_1 = Children(T_1) \\
  C_2 = Children(T_2)
  \end{array}
\]

\begin{figure}[htbp]
  \centering
  \includegraphics[scale=0.5]{img/link-bitree-xy.ps}
  \caption{如果$x < y$,将$y$插入作为$x$的第一个子树} \label{fig:link-xy}
\end{figure}

如果$\cup$是一个常数时间操作,则链接操作的性能也为$O(1)$。下面的Haskell例子程序实现了链接。

\lstset{language=Haskell}
\begin{lstlisting}
link :: (Ord a) => BiTree a -> BiTree a -> BiTree a
link t1@(Node r x c1) t2@(Node _ y c2) =
    if x<y then Node (r+1) x (t2:c1)
    else Node (r+1) y (t1:c2)
\end{lstlisting}

链接操作也可以用命令式的方式实现。如果使用“左侧孩子,右侧兄弟”的布局,只需要将key较大的树链接到另一棵树的左侧field,然后将子树链接到key较大的树的右侧field。如图\ref{fig:link-lcrs}所示。

\begin{algorithmic}[1]
\Function{Link}{$T_1, T_2$}
  \If{\Call{Key}{$T_2$} $<$ \Call{Key}{$T_1$}}
    \State Exchange $T_1 \leftrightarrow T_2$
  \EndIf
  \State \Call{Sibling}{$T_2$} $\gets$ \Call{Child}{$T_1$}
  \State \Call{Child}{$T_1$} $\gets T_2$
  \State \Call{Parent}{$T_2$} $\gets T_1$
  \State \Call{Rank}{$T_1$} $\gets$ \Call{Rank}{$T_1$} + 1
  \State \Return $T_1$
\EndFunction
\end{algorithmic}

\begin{figure}[htbp]
  \centering
  \includegraphics[scale=0.5]{img/link-bitree-lcrs.ps}
  \caption{若$x < y$,将$y$链接到$x$的左侧,将$x$的子树链接到$y$的右侧。} \label{fig:link-lcrs}
\end{figure}

如果使用容器来存储节点的子树,相应的算法如下:

\begin{algorithmic}[1]
\Function{Link'}{$T_1, T_2$}
  \If{\Call{Key}{$T_2$} $<$ \Call{Key}{$T_1$}}
    \State Exchange $T_1 \leftrightarrow T_2$
  \EndIf
  \State \Call{Parent}{$T_2$} $\gets T_1$
  \State \textproc{Insert-Before}(\Call{Children}{$T_1$}, $T_2$)
  \State \Call{Rank}{$T_1$} $\gets$ \Call{Rank}{$T_1$} + 1
  \State \Return $T_1$
\EndFunction
\end{algorithmic}

上述两个算法都易于实现。这里我们给出了\textproc{Link'}的Python例子程序\footnote{相应的C和C++代码也可以随本书一起获得}。

\lstset{language=Python}
\begin{lstlisting}
def link(t1, t2):
    if t2.key < t1.key:
        (t1, t2) = (t2, t1)
    t2.parent = t1
    t1.children.insert(0, t2)
    t1.rank = t1.rank + 1
    return t1
\end{lstlisting}

\begin{Exercise}
选择一门语言,用“左侧孩子,右侧兄弟”的布局实现二项式树的链接程序。
\end{Exercise}

\subsubsection*{}

如果使用“左侧孩子,右侧兄弟”的实现,链接可以在常数时间完成,但是如果使用容器来存储子树,则性能依赖于容器的具体实现。如果容器是基于数组的,链接操作的时间和子树的数目成正比;而如果使用链表,则为常数时间。本章中,我们认为这一时间是常数的。

\subsubsection{插入新元素(push)}
\index{二项式堆!插入}

如果森林中的二项式树的rank是单调增的,通过使用$link$函数,我们可以定义一个辅助函数用于向堆中插入一棵新树。
这棵新树的rank不大于森林中的任何树。

记非空的堆为$H = \{T_1, T_2, ..., T_n\}$,我们定义:

\be
insertT(H, T) = \left \{
  \begin{array}
  {r@{\quad:\quad}l}
  \{ T \} & H = \phi \\
  \{ T \} \cup H & Rank(T) < Rank(T_1) \\
  insertT(H', link(T, T_1)) & otherwise
  \end{array}
\right .
\ee

其中

\[
  H' = \{ T_2, T_3, ..., T_n\}
\]

如果堆为空,则新树为森林中唯一的一棵;否则,我们比较新树和森林中第一棵树的rank,如果相等,就将它们链接成一棵更大的树(rank加1),然后递归地插入到森林中;如果不等,根据限制条件,新树的rank必然是最小的,我们将它插入到森林中所有树的前面。

根据此前给出的二项式堆的性质,如果元素总数为$n$,森林中最多有$O(\lg n)$棵二项式树。函数$insertT$最多执行$O(\lg n)$次常数时间的链接操作。因此$insertT$的性能为$O(\lg n)$\footnote{观察这一操作和两个二进制数的加法,它们有很多相似性。可以引出一个有趣的题目:“numeric representation”\cite{okasaki-book}。}。

相应的Haskell例子程序如下:

\lstset{language=Haskell}
\begin{lstlisting}
insertTree :: (Ord a) => BiHeap a -> BiTree a -> BiHeap a
insertTree [] t = [t]
insertTree ts@(t':ts') t = if rank t < rank t' then t:ts
                           else insertTree ts' (link t t')
\end{lstlisting}

使用这一辅助函数,我们可以实现堆的插入算法。先将待插入元素装入一个叶子节点,然后将它插入到二项式堆中。

\be
insert(H, x) = insertT(H, node(0, x, \phi))
\ee

我们可以连续将若干元素通过folding插入到堆中,下面的Haskell例子程序定义了一个辅助函数\texttt{fromList}。

\begin{lstlisting}
fromList = foldl insert []
\end{lstlisting}

因为将元素放入叶子节点只需要常数时间,主要的工作由$insertT$完成,因此二项式堆的插入操作的性能为$O(\lg n)$。

插入算法也可以用命令式的方式定义。

\begin{algorithm}
\caption{使用“左侧孩子,右侧兄弟”的实现插入一棵新树。}
\label{alg:insert-tree}
\begin{algorithmic}[1]
\Function{Insert-Tree}{$H, T$}
  \While{$H \neq \phi \land $ \textproc{Rank}(\Call{Head}{$H$}) = \Call{Rank}{$T$}}
    \State $(T_1, H) \gets$ \Call{Extract-Head}{$H$}
    \State $T \gets $ \Call{Link}{$T, T_1$}
  \EndWhile
  \State \Call{Sibling}{$T$} $\gets H$
  \State \Return $T$
\EndFunction
\end{algorithmic}
\end{algorithm}

如果rank相等,算法\ref{alg:insert-tree}不断将堆中第一棵树和待插入的树链接到一起。此后,它将剩余的树作为兄弟链接到末尾,然后将新的链表返回。

如果使用容器来存储子树,则算法定义为Algorithm \ref{alg:insert-tree1}。

\begin{algorithm}
\caption{插入一棵新树,使用容器来存储子树。}
\label{alg:insert-tree1}
\begin{algorithmic}[1]
\Function{Insert-Tree'}{$H, T$}
  \While{$H \neq \phi \land $ \Call{Rank}{$H[0]$} = \Call{Rank}{$T$}}
    \State $T_1 \gets$ \Call{Pop}{$H$}
    \State $T \gets $ \Call{Link}{$T, T_1$}
  \EndWhile
  \State \Call{Head-Insert}{$H, T$}
  \State \Return $H$
\EndFunction
\end{algorithmic}
\end{algorithm}

其中函数\textproc{Pop}将森林中的第一棵树$T_1 = H[0]$取出。函数\textproc{Head-Insert}将新树添加到堆中所有树的前面。

使用\textproc{Insert-Tree}或\textproc{Insert-Tree'}中的任何一个,都可以实现二项式堆的插入算法。

\begin{algorithm}
\caption{命令式插入算法}
\label{alg:bheap-insert}
\begin{algorithmic}[1]
\Function{Insert}{$H, x$}
  \State \Return \textproc{Insert-Tree}($H$, \Call{Node}{$0, x, \phi$})
\EndFunction
\end{algorithmic}
\end{algorithm}

下面的Python程序使用内置的列表实现了上述算法,使用“左侧孩子,右侧兄弟”的实现留给读者作为练习。

\lstset{language=Python}
\begin{lstlisting}
def insert_tree(ts, t):
    while ts !=[] and t.rank == ts[0].rank:
        t = link(t, ts.pop(0))
    ts.insert(0, t)
    return ts

def insert(h, x):
    return insert_tree(h, BinomialTree(x))
\end{lstlisting}

\begin{Exercise}
选择一门命令式编程语言,利用”左侧孩子,右侧兄弟“的布局实现二项式堆的插入算法。
\end{Exercise}

% =================================
%     Binomial heap merge
% =================================

\subsubsection{堆合并}
\index{二项式树!合并}

合并两个二项式堆等价于合并两个二项式树的森林。根据定义,合并后的森林中没有rank相同的树,并且树按照rank单调递增的顺序排列。合并过程的思想和归并排序类似。在每次迭代中,我们从两个森林中各取出第一棵树,比较它们的rank,将较小的一棵树放入结果堆中;如果rank相等,我们将它们链接起来成为一棵新树,然后递归将它插入到剩余树的合并结果中。

图\ref{fig:merge-bheaps}描述了这一算法。它和\cite{CLRS}中介绍的实现并不相同。

\begin{figure}[htbp]
  \centering
  \subfloat[选择rank较小的树放入结果中。]{\includegraphics[scale=0.7]{img/bheap-merge-1.ps}} \\
  \subfloat[若两棵树的rank相同,将它们链接成一棵新树,然后递归地将它插入的剩余树的合并结果中。]{\includegraphics[scale=0.7]{img/bheap-merge-2.ps}}
  \caption{堆合并} \label{fig:merge-bheaps}
\end{figure}

可以将合并算法形式化为一个函数。若两个堆不为空,分别记它们为:$H_1 = \{ T_1, T_2, ... \}$、$H_2 = \{ T'_1, T'_2, ...\}$。并且令$H'_1 = \{ T_2, T_3, ... \}$、$H'_2 = \{ T'_2, T'_3, ... \}$。

\be
merge(H_1, H_2) = \left \{
  \begin{array}
  {r@{\quad:\quad}l}
  H_1 & H_2 = \phi \\
  H_2 & H_1 = \phi \\
  \{ T_1 \} \cup merge(H'_1, H_2) & Rank(T_1) < Rank(T'_1) \\
  \{ T'_1 \} \cup merge(H_1, H'_2) & Rank(T_1) > Rank(T'_1) \\
  insertT(merge(H'_1, H'_2), link(T_1, T'_1)) & otherwise
  \end{array}
\right .
\ee

为了分析合并操作的性能,设堆$H_1$中有$m_1$棵树,堆$H_2$中有$m_2$棵树。合并后的结果中最多有$m_1 + m_2$棵树。如果没有rank相同的树,则合并操作的时间为$O(m_1 + m_2)$。如果存在rank相同的树需要链接,最多需要调用$insertT$的次数为$O(m_1 + m_2)$。考虑$m_1 = 1 + \lfloor \lg n_1 \rfloor$,$m_2 = 1 + \lfloor \lg N_2 \rfloor$,其中$n_1$和$n_2$是两个堆各自的节点数目,且$\lfloor \lg n_1 \rfloor + \lfloor \lg n_2 \rfloor \leq 2 \lfloor \lg n \rfloor$,其中$n = n_1 + n_2$为总节点数。最终的合并性能为$O(\lg n)$。

下面的Haskell例子程序实现了合并算法。

\lstset{language=Haskell}
\begin{lstlisting}
merge:: (Ord a) => BiHeap a -> BiHeap a -> BiHeap a
merge ts1 [] = ts1
merge [] ts2 = ts2
merge ts1@(t1:ts1') ts2@(t2:ts2')
    | rank t1 < rank t2 = t1:(merge ts1' ts2)
    | rank t1 > rank t2 = t2:(merge ts1 ts2')
    | otherwise = insertTree (merge ts1' ts2') (link t1 t2)
\end{lstlisting}

合并算法也可以用命令式的方式实现,如Algorithm \ref{alg:bheap-merge}。

\begin{algorithm}
\caption{命令式合并两个堆}
\label{alg:bheap-merge}
\begin{algorithmic}[1]
\Function{Merge}{$H_1, H_2$}
  \If{$H_1 = \phi$}
    \State \Return $H_2$
  \EndIf
  \If{$H_2 = \phi$}
    \State \Return $H_1$
  \EndIf
  \State $H \gets \phi$
  \While{$H_1 \neq \phi \land H_2 \neq \phi$}
    \State $T \gets \phi$
    \If{\Call{Rank}{$H_1$} $<$ \Call{Rank}{$H_2$}}
      \State $(T, H_1) \gets $ \Call{Extract-Head}{$H_1$}
    \ElsIf{\Call{Rank}{$H_2$} $<$ \Call{Rank}{$H_1$}}
      \State $(T, H_2) \gets $ \Call{Extract-Head}{$H_2$}
    \Else \Comment{Equal rank}
      \State $(T_1, H_1) \gets $ \Call{Extract-Head}{$H_1$}
      \State $(T_2, H_2) \gets $ \Call{Extract-Head}{$H_2$}
      \State $T \gets $ \Call{Link}{$T_1, T_2$}
    \EndIf
    \State \Call{Append-Tree}{$H, T$}
  \EndWhile
  \If{$H_1 \neq \phi$}
    \State \Call{Append-Trees}{$H, H_1$}
  \EndIf
  \If{$H_2 \neq \phi$}
    \State \Call{Append-Trees}{$H, H_2$}
  \EndIf
  \State \Return $H$
\EndFunction
\end{algorithmic}
\end{algorithm}

两个堆都含有rank单调递增的二项式树。每次迭代,我们选出rank最小的树并追加到结果堆中。如果两个堆中的第一棵树的rank相等,我们先把它们链接成一棵新树。考虑\textproc{Append-Tree}过程。根据我们的合并策略,新树的rank不能比结果堆中的任何一棵小,但是它有可能和结果树中的最后一棵相等。链接操作可能会引起这样的情况,因为它会将树的rank增加1。此时我们需要把新树和结果堆中的最后一棵树链接起来。下面的算法中,函数\textproc{Last}($H$)给出堆中最后一棵树,函数\textproc{Append}($H, T$)仅仅将一棵新树追加到森林的末尾。

\begin{algorithmic}[1]
\Function{Append-Tree}{$H, T$}
  \If{$H \neq \phi \land$ \Call{Rank}{$T$} $=$ \textproc{Rank}(\Call{Last}{$H$})}
    \State \Call{Last}{$H$} $\gets$ \textproc{Link}($T$, \Call{Last}{$H$})
  \Else
    \State \Call{Append}{$H, T$}
  \EndIf
\EndFunction
\end{algorithmic}

\textproc{Append-Trees}不断调用上述函数,逐一将一个堆中的树追加到另一堆中。

\begin{algorithmic}[1]
\Function{Append-Trees}{$H_1, H_2$}
  \For{each $T \in H_2$}
    \State $H_1 \gets $ \Call{Append-Tree}{$H_1, T$}
  \EndFor
\EndFunction
\end{algorithmic}

下面的Python例子程序实现了合并算法。

\lstset{language=Python}
\begin{lstlisting}
def append_tree(ts, t):
    if ts != [] and ts[-1].rank == t.rank:
        ts[-1] = link(ts[-1], t)
    else:
        ts.append(t)
    return ts

def append_trees(ts1, ts2):
    return reduce(append_tree, ts2, ts1)

def merge(ts1, ts2):
    if ts1 == []:
        return ts2
    if ts2 == []:
        return ts1
    ts = []
    while ts1 != [] and ts2 != []:
        t = None
        if ts1[0].rank < ts2[0].rank:
            t = ts1.pop(0)
        elif ts2[0].rank < ts1[0].rank:
            t = ts2.pop(0)
        else:
            t = link(ts1.pop(0), ts2.pop(0))
        ts = append_tree(ts, t)
    ts = append_trees(ts, ts1)
    ts = append_trees(ts, ts2)
    return ts
\end{lstlisting}

\begin{Exercise}

例子程序使用了容器来存储子树。选择一门语言,实现”左侧孩子,右侧兄弟“方式堆的合并。

\end{Exercise}

\subsubsection{弹出}
\index{二项式堆!弹出}

在二项式堆的森林中,每棵二项式树都符合堆性质,根节点保存了树中的最小元素。但是这些根节点元素间的大小关系是任意的。为了获取堆中的最小元素,我们需要从全部树根中找到最小元素。因为堆中有$\lg n$棵树,所以获取最小值的复杂度为$O(\lg n)$。

但是弹出操作要求不仅仅找到最小元素(即top),还需要将其删除并保持堆性质。设构成堆的各个二项式树为$B_i, B_j, ..., B_p, ..., B_m$,其中$B_k$为rank为$k$的二项式树。即堆中最小元素保存在树$B_p$的根节点。将其删除后,会产生$p$棵子树,它们都是二项式树,rank分别为$p-1, p-2, ..., 0$。

此前我们已经定义了性能为$O(\lg n)$的合并函数。一个思路是将$p$棵子树逆序,这样它们的rank就变为单调递增的,形成一个二项式堆$H_p$。剩余的树也构成一个二项式堆,可以表示为$H' = H - B_p$。将$H_p$和$H'$合并就可以得到弹出操作的最终结果。图\ref{fig:bheap-del-min}描述了这一思路。

\begin{figure}[htbp]
  \centering
  \includegraphics[scale=0.5]{img/bheap-pop.eps}
  \caption{二项式堆的弹出操作}
  \label{fig:bheap-del-min}
\end{figure}

为了实现弹出算法,我们需要先定义一个函数,可以从森林中取出根节点最小的树。

\be
extractMin(H) = \left \{
  \begin{array}
  {r@{\quad:\quad}l}
  (T, \phi) & \text{H只含有一个元素,形如} \{ T \} \\
  (T_1, H') & Root(T_1) < Root(T') \\
  (T', \{T_1\} \cup H'') & otherwise
  \end{array}
\right .
\ee

其中

\[
  \begin{array}{lr}
  H = \{ T_1, T_2, ...\} & \text{非空的森林;} \\
  H' = \{ T_2, T_3, ...\} & \text{去除第一棵树的森林;} \\
  (T', H'') = extractMin(H')
  \end{array}
\]

此函数的结果为一对值(tuple),第一部份是根节点最小的树,第二部份是森林中剩余的其他树。
函数逐一检查并比较森林中的每棵树,因此它的性能为$O(\lg n)$。

相应的Haskell例子程序如下:

\lstset{language=Haskell}
\begin{lstlisting}
extractMin :: (Ord a) => BiHeap a -> (BiTree a, BiHeap a)
extractMin [t] = (t, [])
extractMin (t:ts) = if root t < root t' then (t, ts)
                    else (t', t:ts')
    where
      (t', ts') = extractMin ts
\end{lstlisting}

With this function defined, to return the minimum element is trivial.

\begin{lstlisting}
findMin :: (Ord a) => BiHeap a -> a
findMin = root . fst. extractMin
\end{lstlisting}

当然,也可以仅遍历森林中的所有树,找出最小的根节点而不将树删除。下面的命令式算法使用“左侧孩子,右侧分支”的布局实现了最小值的查找。

\begin{algorithmic}[1]
\Function{Find-Minimum}{$H$}
  \State $T \gets $ \Call{Head}{$H$}
  \State $min \gets \infty$
  \While{$T \ne \phi$}
    \If{\Call{Key}{$T$}$ < min$}
      \State $min \gets $ \Call{Key}{$T$}
    \EndIf
    \State $T \gets $ \Call{Sibling}{$T$}
  \EndWhile
  \State \Return $min$
\EndFunction
\end{algorithmic}

如果使用容器来存储子树,就需要在二项式树的列表中寻找根节点最小的一棵。下面的Python例子程序给出了这种情况的实现。

\lstset{language=Python}
\begin{lstlisting}
def find_min(ts):
    min_t = min(ts, key=lambda t: t.key)
    return min_t.key
\end{lstlisting}

接下来需要使用$extractMin$来定义从堆中删除最小元素的函数。

\be
delteMin(H) = merge(reverse(Children(T)), H')
\ee

其中

\[
  (T, H') = extractMin(H)
\]

我们在此略过了相应的Haskell例子代码。

为了给出命令式的实现,我们需要额外实现列表反转等操作。我们将其留给读者作为练习。下面的伪代码描述了命令式的弹出算法。

\begin{algorithmic}[1]
\Function{Extract-Min}{$H$}
  \State $(T_{min}, H) \gets$ \Call{Extract-Min-Tree}{$H$}
  \State $H \gets$ \textproc{Merge}($H$, \textproc{Reverse}(\Call{Children}{$T_{min}$}))
  \State \Return (\Call{Key}{$T_{min}$}, $H$)
\EndFunction
\end{algorithmic}

使用弹出操作可以实现堆排序。首先从待排序元素构建一个二项式堆,然后不断从中弹出最小元素直到堆变为空。

\be
sort(xs) = heapSort(fromList(xs))
\ee

其中的$heapSort$实现如下:

\be
heapSort(H) = \left \{
  \begin{array}
  {r@{\quad:\quad}l}
  \phi & H = \phi \\
  \{ findMin(H)  \} \cup heapSort(deleteMin(H)) & otherwise
  \end{array}
\right .
\ee

下面的Haskell例子程序实现了堆排序。

\lstset{language=Haskell}
\begin{lstlisting}
heapSort :: (Ord a) => [a] -> [a]
heapSort = hsort . fromList where
    hsort [] = []
    hsort h = (findMin h):(hsort $ deleteMin h)
\end{lstlisting} %$

其中$fromList$函数可以通过folding来定义。也可以用命令式的方法实现二项式堆排序,读者可以参考前面binarh堆的相关章节。

\begin{Exercise}
\begin{itemize}
\item 选择一门编程语言,用“左侧孩子,右侧兄弟”的方法实现从二项式堆中取得最小元素的操作。

\item 实现命令式的\textproc{Extract-Min-Tree}()算法。

\item 使用“左侧孩子,右侧兄弟”的方法,将一棵树的所有子树逆序相当于实现单向链表的反转。选择一门编程语言,实现单向链表的反转。
\end{itemize}
\end{Exercise}

\subsubsection{其他}
此前我们给出的二项式堆的插入、合并的性能都是$O(\lg n)$。这一结论是针对在\underline{最坏}情况的。这些操作的分摊复杂度为$O(1)$。我们这里略去了分摊复杂度的证明。

% ================================================================
%                 Fibonacci heaps
% ================================================================
\section{斐波那契堆}
\label{fib-heap} \index{斐波那契堆}

“斐波那契堆”的命名很有趣,实际上斐波那契堆的结构和斐波那契数列无关。斐波那契堆的作者Michael L. Fredman和Robert E. Tarjan在证明这种堆的时间性能时使用了斐波那契数列的性质,于是他们决定给这种堆命名为“斐波那契堆”\cite{CLRS}。

% ================================================================
%                 Definition
% ================================================================
\subsection{定义}

斐波那契堆本质上是一个惰性二项式堆。但是这并不意味着二项式堆在支持惰性求值的环境下(例如Haskell)自动就成为斐波那契堆。惰性环境仅仅对于实现提供了便利。例如\cite{hackage-fibq}中给出了一个简洁的实现。

斐波那契堆在理论上具有良好的性能。除弹出之外所有的操作分摊性能都达到了常数时间$O(1)$。本节中,我们给出的实现和常见的实现\cite{CLRS}有所不同。主要思想来自于Okasaki的工作\cite{okasaki-fibh}。

首先我们对比一下二项式堆和斐波那契堆的性能(确切的说,是我们希望斐波那契堆达到的性能目标)。

% \begin{table}
% \caption{Performance goal of Fibonacci heap}
\begin{tabular}{l | c | r}
  \hline
  操作 & 二项式堆 & 斐波那契堆 \\
  \hline
  插入 & $O(\lg n)$ & $O(1)$ \\
  合并 & $O(\lg n)$ & $O(1)$ \\
  top & $O(\lg n)$ & $O(1)$ \\
  弹出 & $O(\lg n)$ & 分摊 $O(\lg n)$ \\
  \hline
\end{tabular}
% \end{table}

在二项式堆中,插入一个新元素时,哪里是瓶颈呢?新元素$x$被放入只有一个叶子节点的树中,然后这棵树被插入到森林中。

在此期间,树按照rank的单调递增顺序插入,如果rank相等,则进行链接,然后再递归,因此性能为$O(\lg n)$。

使用惰性策略,我们可以将按照rank的顺序插入和链接等操作推迟进行。仅仅将只有一个叶子节点的树放入森林中。这样带来的问题是,当获取最小元素时,性能会变得很差。这是因为我们需要检查森林中的所有树,而树的总数不只是$O(\lg n)$。

为了在常数时间获得堆顶元素,我们需要记录哪一棵树的根节点保存了最小元素。

根据这一思路,我们可以复用二项式树的定义。下面的Haskell例子程序给出了斐波那契树的定义。

\lstset{language=Haskell}
\begin{lstlisting}
data BiTree a = Node { rank :: Int
                     , root :: a
                     , children :: [BiTree a]}
\end{lstlisting}

The Fibonacci heap is either empty or a forest of binomial trees with
the minimum element stored in a special one explicitly.

\begin{lstlisting}
data FibHeap a = E | FH { size :: Int
                        , minTree :: BiTree a
                        , trees :: [BiTree a]}
\end{lstlisting}

For convenient purpose, we also add a size field to record how many
elements are there in a heap.

The data layout can also be defined in imperative way as the following
ANSI C code.

\lstset{language=C}
\begin{lstlisting}
struct node{
  Key key;
  struct node *next, *prev, *parent, *children;
  int degree; /* As known as rank */
  int mark;
};

struct FibHeap{
  struct node *roots;
  struct node *minTr;
  int n; /* number of nodes */
};
\end{lstlisting}

For generality, Key can be a customized type, we use integer for illustration
purpose.

\lstset{language=C}
\begin{lstlisting}
typedef int Key;
\end{lstlisting}

In this chapter, we use the circular doubly linked-list for imperative
settings to realize the Fibonacci Heap as described in \cite{CLRS}.
It makes many operations easy and fast. Note that, there are two extra
fields added. A $degree$ also known as $rank$ for a node is the number
of children of this node; Flag $mark$ is used only in decreasing key
operation. It will be explained in detail in later section.


% ================================================================
%          Basic Heap operations
% ================================================================
\subsection{Basic heap operations}
As we mentioned that Fibonacci heap is essentially binomial heap
implemented in a lazy evaluation strategy, we'll reuse many algorithms
defined for binomial heap.

\subsubsection{Insert a new element to the heap}
\index{Fibonacci Heap!insert}
Recall the insertion algorithm of binomial  tree. It can be treated
as a special case of merge operation, that one heap contains only
a singleton tree. So that the inserting algorithm can be defined
by means of merging.

\be
insert(H, x) = merge(H, singleton(x))
\label{eq:fib-insert}
\ee

where singleton is an auxiliary function to wrap an element to a
one-leaf-tree.

\[
singleton(x) = FibHeap(1, node(1, x, \phi), \phi)
\]

Note that function $FibHeap()$ accepts three parameters, a
size value, which is 1 for this one-leaf-tree, a special tree
which contains the minimum element as root, and a list of other
binomial trees in the forest. The meaning of function $node()$ is
as same as before, that it creates a binomial tree from a rank,
an element, and a list of children.

Insertion can also be realized directly by appending the new node
to the forest and updated the record of the tree which contains the
minimum element.

\begin{algorithmic}[1]
\Function{Insert}{$H, k$}
  \State $x \gets$ \Call{Singleton}{$k$} \Comment{Wrap $x$ to a node}
  \State append $x$ to root list of $H$
  \If{$T_{min}(H) = NIL \lor k <$ \Call{Key}{$T_{min}(H)$} }
    \State $T_{min}(H) \gets x$
  \EndIf
  \State \Call{n}{$H$} $\gets$ \Call{n}{$H$}+1
\EndFunction
\end{algorithmic}

Where function $T_{min}()$ returns the tree which contains the minimum
element at root.

The following C source snippet is a translation for this algorithm.

\lstset{language=C}
\begin{lstlisting}
struct FibHeap* insert_node(struct FibHeap* h, struct node* x){
  h = add_tree(h, x);
  if(h->minTr == NULL || x->key < h->minTr->key)
    h->minTr = x;
  h->n++;
  return h;
}
\end{lstlisting}

\begin{Exercise}
Implement the insert algorithm in your favorite imperative programming
language completely. This is also an exercise to circular doubly linked list
manipulation.
\end{Exercise}

\subsubsection{Merge two heaps}
\index{Fibonacci Heap!merge}
Different with the merging algorithm of binomial heap, we post-pone
the linking operations later. The idea is to just put all binomial
trees from each heap together, and choose one special tree which
record the minimum element for the result heap.

\be
merge(H_1, H_2) = \left \{
  \begin{array}
  {r@{\quad:\quad}l}
  H_1 & H_2 = \phi \\
  H_2 & H_1 = \phi \\
  FibHeap(s_1 + s_2, {T_1}_{min}, \{ {T_2}_{min} \} \cup \mathbb{T}_1 \cup \mathbb{T}_2) & root({T_1}_{min}) < root({T_2}_{min}) \\
  FibHeap(s_1 + s_2, {T_2}_{min}, \{ {T_1}_{min} \} \cup \mathbb{T}_1 \cup \mathbb{T}_2) & otherwise \\
  \end{array}
\right .
\ee

where $s_1$ and $s_2$ are the size of $H_1$ and $H_2$; ${T_1}_{min}$ and
${T_2}_{min}$ are the special trees with minimum element as root in $H_1$
and $H_2$ respectively; $\mathbb{T}_1 = \{{T_1}_1, {T_1}_2, ...\}$ is
a forest contains all other binomial trees in $H_1$; while $\mathbb{T}_2$
has the same meaning as $\mathbb{T}_1$ except that it represents the
forest in $H_2$. Function $root(T)$ return the root element of a binomial
tree.

Note that as long as the $\cup$ operation takes constant time, these
$merge$ algorithm is bound to $O(1)$. The following Haskell program
is the translation of this algorithm.

\lstset{language=Haskell}
\begin{lstlisting}
merge:: (Ord a) => FibHeap a -> FibHeap a -> FibHeap a
merge h E = h
merge E h = h
merge h1@(FH sz1 minTr1 ts1) h2@(FH sz2 minTr2 ts2)
    | root minTr1 < root minTr2 = FH (sz1+sz2) minTr1 (minTr2:ts2++ts1)
    | otherwise = FH (sz1+sz2) minTr2 (minTr1:ts1++ts2)
\end{lstlisting}

Merge algorithm can also be realized imperatively by concatenating
the root lists of the two heaps.

\begin{algorithmic}[1]
\Function{Merge}{$H_1, H_2$}
  \State $H \gets \Phi$
  \State \Call{Root}{$H$} $\gets$ \textproc{Concat}(\Call{Root}{$H_1$}, \Call{Root}{$H_2$})
  \If{\Call{Key}{$T_{min}(H_1)$} $<$ \Call{Key}{$T_{min}(H_2)$}}
    \State $T_{min}(H) \gets T_{min}(H_1)$
  \Else
    \State $T_{min}(H) \gets T_{min}(H_2)$
  \EndIf
  \Call{n}{$H$} = \Call{n}{$H_1$} + \Call{n}{$H_2$}
  \State release $H_1$ and $H_2$
  \State \Return $H$
\EndFunction
\end{algorithmic}

This function assumes neither $H_1$, nor $H_2$ is empty. And it's easy
to add handling to these special cases as the following ANSI C program.

\lstset{language=C}
\begin{lstlisting}
struct FibHeap* merge(struct FibHeap* h1, struct FibHeap* h2){
  struct FibHeap* h;
  if(is_empty(h1))
    return h2;
  if(is_empty(h2))
    return h1;
  h = empty();
  h->roots = concat(h1->roots, h2->roots);
  if(h1->minTr->key < h2->minTr->key)
    h->minTr = h1->minTr;
  else
    h->minTr = h2->minTr;
  h->n = h1->n + h2->n;
  free(h1);
  free(h2);
  return h;
}
\end{lstlisting}

With $merge$ function defined, the $O(1)$ insertion algorithm is realized
as well. And we can also give the $O(1)$ time top function as below.

\be
top(H) = root(T_{min})
\ee

\begin{Exercise}
Implement the circular doubly linked list concatenation function in
your favorite imperative programming language.
\end{Exercise}

\subsubsection{Extract the minimum element from the heap (pop)}
\index{Fibonacci Heap!pop} \index{Fibonacci Heap!delete min}

The pop (delete the minimum element) operation is the most complex
one in Fibonacci heap. Since we postpone the tree consolidation
in merge algorithm. We have to compensate it somewhere. Pop is
the only place left as we have defined, insert, merge, top already.

There is an elegant procedural algorithm to do the tree consolidation
by using an auxiliary array\cite{CLRS}. We'll show it later in imperative
approach section.

In order to realize the purely functional consolidation algorithm,
let's first consider a similar number puzzle.

Given a list of numbers, such as $\{2, 1, 1, 4, 8, 1, 1, 2, 4\}$, we want
to add any two values if they are same. And repeat this procedure till
all numbers are unique. The result of the example list should be
$\{8, 16\}$ for instance.

One solution to this problem will as the following.

\be
consolidate(L) = fold(meld, \phi, L)
\ee

Where $fold()$ function is defined to iterate all elements from a list,
applying a specified function to the intermediate result and each
element. it is sometimes called as {\em reducing}. Please refer to the
chapter of binary search tree for it.

$L=\{x_1, x_2, ..., x_n\}$, denotes a list of numbers; and we'll use
$L'=\{x_2, x_3, ..., x_n\}$ to represent the rest of the list with the
first element removed. Function $meld()$ is defined as below.

\be
meld(L, x) = \left \{
  \begin{array}
  {r@{\quad:\quad}l}
  \{ x \} & L = \phi \\
  meld(L', x+x_1) & x = x_1 \\
  \{ x \} \cup L & x < x_1 \\
  \{ x_1 \} \cup meld(L', x) & otherwise
  \end{array}
\right .
\ee

The $consolidate()$ function works as the follows. It maintains an
ordered result list $L$, contains only unique numbers, which is
initialized from an empty list $\phi$. Each time it process an
element $x$, it firstly check if the first element in $L$ is equal
to $x$, if so, it will add them together (which yields $2x$),
and repeatedly check if $2x$ is equal to the next element in $L$.
This process won't stop until either the element to be melt is
not equal to the head element in the rest of the list, or the
list becomes empty. Table \ref{tb:num-consolidate} illustrates
the process of consolidating number sequence $\{2, 1, 1, 4, 8, 1, 1, 2, 4\}$.
Column one lists the number 'scanned' one by one; Column two
shows the intermediate result, typically the new scanned number
is compared with the first number in result list. If they
are equal, they are enclosed in a pair of parentheses; The
last column is the result of meld, and it will be used as the
input to next step processing.

The Haskell program can be give accordingly.

\lstset{language=Haskell}
\begin{lstlisting}
consolidate = foldl meld [] where
    meld [] x = [x]
    meld (x':xs) x | x == x' = meld xs (x+x')
                   | x < x'  = x:x':xs
                   | otherwise = x': meld xs x
\end{lstlisting}

We'll analyze the performance of consolidation as a part of
pop operation in later section.

\begin{table}
\caption{Steps of consolidate numbers} \label{tb:num-consolidate}
\centering
\begin{tabular}{r | l | l }
  \hline
  number & intermediate result & result \\
  \hline
  2 & 2 & 2 \\
  1 & 1, 2 & 1, 2 \\
  1 & (1+1), 2 & 4 \\
  4 & (4+4) & 8 \\
  8 & (8+8) & 16 \\
  1 & 1, 16 & 1, 16 \\
  1 & (1+1), 16 & 2, 16 \\
  2 & (2+2), 16 & 4, 16 \\
  4 & (4+4), 16 & 8, 16 \\
  \hline
\end{tabular}
\end{table}

The tree consolidation is very similar to this algorithm except
it performs based on rank. The only thing we need to do is to
modify $meld()$ function a bit, so that it compare on ranks and
do linking instead of adding.

\be
meld(L, x) = \left \{
  \begin{array}
  {r@{\quad:\quad}l}
  \{ x \} & L = \phi \\
  meld(L', link(x, x_1)) & rank(x) = rank(x_1) \\
  \{ x \} \cup L & rank(x) < rank(x_1) \\
  \{ x_1 \} \cup meld(L', x) & otherwise
  \end{array}
\right .
\ee

The final consolidate Haskell program changes to the below version.

\lstset{language=Haskell}
\begin{lstlisting}
consolidate :: (Ord a) => [BiTree a] -> [BiTree a]
consolidate = foldl meld [] where
    meld [] t = [t]
    meld (t':ts) t | rank t == rank t' = meld ts (link t t')
                   | rank t <  rank t' = t:t':ts
                   | otherwise = t' : meld ts t
\end{lstlisting}

Figure \ref{fig:fib-meld-a} and \ref{fig:fib-meld-b} show the steps of
consolidation when processing a Fibonacci Heap contains different ranks
of trees. Comparing with table \ref{tb:num-consolidate} reveals the similarity.

\begin{figure}[htbp]
  \centering
  \subfloat[Before consolidation]{\includegraphics[scale=0.5]{img/fib-meld-01.ps}} \\
  \subfloat[Step 1, 2]{\includegraphics[scale=0.5]{img/fib-meld-02.ps}}
  \subfloat[Step 3, 'd' is firstly linked to 'c', then repeatedly linked to 'a'.]{ \hspace{0.1\textwidth} \includegraphics[scale=0.5]{img/fib-meld-03.ps} \hspace{0.1\textwidth}}
  \subfloat[Step 4]{\includegraphics[scale=0.5]{img/fib-meld-04.ps}}
  \caption{Steps of consolidation} \label{fig:fib-meld-a}
\end{figure}

\begin{figure}[htbp]
  \centering
  \subfloat[Step 5]{\includegraphics[scale=0.5]{img/fib-meld-05.ps}}
  \subfloat[Step 6]{\includegraphics[scale=0.5]{img/fib-meld-06.ps}} \\
  \subfloat[Step 7, 8, 'r' is firstly linked to 'q', then 's' is linked to 'q'.]{\includegraphics[scale=0.5]{img/fib-meld-07.ps}}
  \caption{Steps of consolidation} \label{fig:fib-meld-b}
\end{figure}

After we merge all binomial trees, including the special tree
record for the minimum element in root, in a Fibonacci heap, the heap
becomes a Binomial heap. And we lost the special tree, which gives
us the ability to return the top element in $O(1)$ time.

It's necessary to perform a $O(\lg N)$ time search to resume the
special tree. We can reuse the function $extractMin()$ defined for
Binomial heap.

It's time to give the final pop function for Fibonacci heap as all
the sub problems have been solved. Let $T_{min}$ denote the special
tree in the heap to record the minimum element in root; $\mathbb{T}$
denote the forest contains all the other trees except for the
special tree, $s$ represents the size of a heap, and function
$children()$ returns all sub trees except the root of a binomial
tree.

\be
deleteMin(H) =  \left \{
  \begin{array}
  {r@{\quad:\quad}l}
  \phi & \mathbb{T} = \phi \land children(T_{min})=\phi \\
  FibHeap(s-1, T'_{min}, \mathbb{T}') & otherwise
  \end{array}
\right .
\ee

Where

\[
  (T'_{min}, \mathbb{T}') = extractMin(consolidate(children(T_{min}) \cup \mathbb{T}))
\]

Translate to Haskell yields the below program.

\lstset{language=Haskell}
\begin{lstlisting}
deleteMin :: (Ord a) => FibHeap a -> FibHeap a
deleteMin (FH _ (Node _ x []) []) = E
deleteMin h@(FH sz minTr ts) = FH (sz-1) minTr' ts' where
    (minTr', ts') = extractMin $ consolidate (children minTr ++ ts)
\end{lstlisting} %$

The main part of the imperative realization is similar. We cut all children of
$T_{min}$ and append them to root list, then perform consolidation to merge
all trees with the same rank until all trees are unique in term of rank.

\begin{algorithmic}[1]
\Function{Delete-Min}{$H$}
  \State $x \gets T_{min}(H)$
  \If{$x \neq NIL$}
    \For{each $y \in $ \Call{Children}{$x$}}
      \State append $y$ to root list of $H$
      \State \Call{Parent}{$y$} $\gets NIL$
    \EndFor
    \State remove $x$ from root list of $H$
    \State \Call{n}{$H$} $\gets$ \Call{n}{$H$} - 1
    \State \Call{Consolidate}{$H$}
  \EndIf
  \State \Return $x$
\EndFunction
\end{algorithmic}

Algorithm \textproc{Consolidate} utilizes an auxiliary array $A$ to do the
merge job. Array $A[i]$ is defined to store the tree with rank (degree) $i$.
During the traverse of root list, if we meet another tree of rank $i$, we
link them together to get a new tree of rank $i+1$. Next we clean $A[i]$,
and check if $A[i+1]$ is empty and perform further linking if necessary.
After we finish traversing all roots, array $A$ stores all result trees
and we can re-construct the heap from it.

\begin{algorithmic}[1]
\Function{Consolidate}{$H$}
  \State $D \gets $ \textproc{Max-Degree}(\Call{n}{$H$})
  \For{$i \gets 0$ to $D$}
    \State $A[i] \gets NIL$
  \EndFor
  \For{each $x \in$ root list of $H$}
    \State remove $x$ from root list of $H$
    \State $d \gets $ \Call{Degree}{$x$}
    \While{$A[d] \neq NIL$}\
      \State $y \gets A[d]$
      \State $x \gets $ \Call{Link}{$x, y$}
      \State $A[d] \gets NIL$
      \State $d \gets d + 1$
    \EndWhile
    \State $A[d] \gets x$
  \EndFor
  \State $T_{min}(H) \gets NIL$ \Comment{root list is NIL at the time}
  \For{$i \gets 0$ to $D$}
    \If{$A[i] \neq NIL$}
      \State append $A[i]$ to root list of $H$.
      \If{$T_{min} = NIL \lor$ \Call{Key}{$A[i]$} $<$ \Call{Key}{$T_{min}(H)$}}
        \State $T_{min}(H) \gets A[i]$
      \EndIf
    \EndIf
  \EndFor
\EndFunction
\end{algorithmic}

The only unclear sub algorithm is \textproc{Max-Degree}, which can determine
the upper bound of the degree of any node in a Fibonacci Heap. We'll delay
the realization of it to the last sub section.

Feed a Fibonacci Heap shown in Figure \ref{fig:fib-meld-a} to the above algorithm,
Figure \ref{fig:fib-cons-a}, \ref{fig:fib-cons-b} and \ref{fig:fib-cons-c}
show the result trees stored in auxiliary array $A$ in every steps.

\begin{figure}[htbp]
  \centering
  \subfloat[Step 1, 2]{\includegraphics[scale=0.5]{img/fib-cons-02.ps}}
  \subfloat[Step 3, Since $A_0 \neq NIL$, 'd' is firstly linked to 'c', and clear $A_0$ to $NIL$. Again, as $A_1 \neq NIL$, 'c' is linked to 'a' and the new tree is stored in $A_2$.]{\includegraphics[scale=0.5]{img/fib-cons-03.ps}}
  \subfloat[Step 4]{\includegraphics[scale=0.5]{img/fib-cons-04.ps}}
  \caption{Steps of consolidation} \label{fig:fib-cons-a}
\end{figure}

\begin{figure}[htbp]
  \centering
  \subfloat[Step 5]{\includegraphics[scale=0.5]{img/fib-cons-05.ps}} \\
  \subfloat[Step 6]{\includegraphics[scale=0.5]{img/fib-cons-06.ps}}
  \caption{Steps of consolidation} \label{fig:fib-cons-b}
\end{figure}

\begin{figure}[htbp]
  \centering
  \subfloat[Step 7, 8, Since $A_0 \neq NIL$, 'r' is firstly linked to 'q', and the new tree is stored in $A_1$ ($A_0$ is cleared); then 's' is linked to 'q', and stored in $A_2$ ($A_1$ is cleared).]{\includegraphics[scale=0.5]{img/fib-cons-07.ps}}
  \caption{Steps of consolidation} \label{fig:fib-cons-c}
\end{figure}


Translate the above algorithm to ANSI C yields the below program.

\lstset{language = C}
\begin{lstlisting}
void consolidate(struct FibHeap* h){
  if(!h->roots)
    return;
  int D = max_degree(h->n)+1;
  struct node *x, *y;
  struct node** a = (struct node**)malloc(sizeof(struct node*)*(D+1));
  int i, d;
  for(i=0; i<=D; ++i)
    a[i] = NULL;
  while(h->roots){
    x = h->roots;
    h->roots = remove_node(h->roots, x);
    d= x->degree;
    while(a[d]){
      y = a[d];  /* Another node has the same degree as x */
      x = link(x, y);
      a[d++] = NULL;
    }
    a[d] = x;
  }
  h->minTr = h->roots = NULL;
  for(i=0; i<=D; ++i)
    if(a[i]){
      h->roots = append(h->roots, a[i]);
      if(h->minTr == NULL || a[i]->key < h->minTr->key)
	h->minTr = a[i];
    }
  free(a);
}
\end{lstlisting}

\begin{Exercise}
Implement the remove function for circular doubly linked list in your favorite
imperative programming language.
\end{Exercise}

\subsection{Running time of pop}

In order to analyze the amortize performance of pop,
we adopt potential method. Reader can refer to \cite{CLRS} for a formal
definition. In this chapter, we only give a intuitive illustration.

Remind the gravity potential energy, which is defined as
\[
E = M \cdot g \cdot h
\]

Suppose there is a complex process, which moves the object with mass $M$
up and down, and finally the object stop at height $h'$. And if there
exists  friction resistance $W_f$, We say
the process works the following power.

\[
W = M \cdot g \cdot (h' - h) + W_f
\]

\begin{figure}[htbp]
  \centering
  \includegraphics[scale=0.5]{img/potential-energy.eps}
  \caption{Gravity potential energy.}
  \label{fig:potential-energy}
\end{figure}

Figure \ref{fig:potential-energy} illustrated this concept.

We treat the Fibonacci heap pop operation in a similar
way, in order to evaluate the cost, we firstly define the potential
$\Phi(H)$ before extract the minimum element. This potential is
accumulated by insertion and merge operations executed so far.
And after tree consolidation and
we get the result $H'$, we then calculate the new potential $\Phi(H')$.
The difference between $\Phi(H')$ and $\Phi(H)$ plus the contribution
of consolidate algorithm indicates the amortized
performance of pop.

For pop operation analysis, the potential can be defined as

\be
\Phi(H) = t(H)
\ee

Where $t(H)$ is the number of trees in Fibonacci heap forest.
We have $t(H) = 1 + length(\mathbb{T})$ for any non-empty heap.

For the $N$-nodes Fibonacci heap, suppose there is an upper bound
of ranks for all trees as $D(N)$. After consolidation, it ensures
that the number of trees in the heap forest is at most $D(N)+1$.

Before consolidation, we actually did another important thing, which
also contribute to running time, we removed the root of the minimum
tree, and concatenate all children left to the forest. So consolidate
operation at most processes $D(N)+t(H)-1$ trees.

Summarize all the above factors, we deduce the amortized cost
as below.

\be
\begin{array}{lll}
T & = & T_{consolidation} + \Phi(H') -\Phi(H) \\
  & = & O(D(N)+t(H)-1) + (D(N) + 1) - t(H) \\
  & = & O(D(N))
\end{array}
\ee

If only insertion, merge, and pop function are applied to Fibonacci
heap. We ensure that all trees are binomial trees. It is easy to
estimate the upper limit $D(N)$ if $O(\lg N)$. (Suppose the extreme
case, that all nodes are in only one Binomial tree).

However, we'll show in next sub section that, there is operation can
violate the binomial tree assumption.

\subsection{Decreasing key}
\index{Fibonacci Heap!decrease key}
There is a special
heap operation left. It only makes sense for imperative settings.
It's about decreasing key of a certain node. Decreasing key plays
important role in some Graphic algorithms such as Minimum Spanning
tree algorithm and Dijkstra's algorithm \cite{CLRS}. In that case
we hope the decreasing key takes O(1) amortized time.

However, we can't define a function like $Decrease(H, k, k')$, which
first locates a node with key $k$, then decrease $k$ to $k'$ by replacement,
and then resume the heap properties. This is because the time for
locating phase is bound to $O(N)$ time, since we don't have a pointer
to the target node.

In imperative setting, we can define the algorithm as
\textproc{Decrease-Key}($H, x, k$). Here $x$ is a node in heap $H$, which
we want to decrease its key to $k$. We needn't perform a search, as
we have $x$ at hand. It's possible to give an amortized $O(1)$ solution.

When we decreased the key of a node, if it's not a root, this operation
may violate the property Binomial tree that the key of parent is
less than all keys of children. So we need to compare the decreased key
with the parent node, and if this case happens, we can cut this node
and append it to the root list. (Remind the recursive swapping solution
for binary heap which leads to $O(\lg N)$)

\begin{figure}[htbp]
  \centering
  \setlength{\unitlength}{1cm}
  \begin{picture}(12, 7)
    \put(0, 0){\includegraphics[scale=0.7]{img/cut-fib-tree.ps}}
    \put(6.7, 3){\line(1, 1){0.5}}
    \put(6.7, 3.5){\line(1, -1){0.5}}
  \end{picture}
  \caption{$x<y$, cut tree $x$ from its parent, and add $x$ to root list.} \label{fig:cut-fib-tree}
\end{figure}

Figure \ref{fig:cut-fib-tree} illustrates this situation. After decreasing
key of node $x$, it is less than $y$, we cut $x$ off its parent $y$, and
'past' the whole tree rooted at $x$ to root list.

Although we recover the property of that parent is less than all children,
the tree isn't any longer a Binomial tree after it losses some sub tree.
If a tree losses too many of its children because of cutting, we can't ensure
the performance of merge-able heap operations. Fibonacci Heap adds another
constraints to avoid such problem:

{\em If a node losses its second child, it is immediately cut from parent,
and added to root list}

The final \textproc{Decrease-Key} algorithm is given as below.

\begin{algorithmic}[1]
\Function{Decrease-Key}{$H, x, k$}
  \State \Call{Key}{$x$} $\gets k$
  \State $p \gets $ \Call{Parent}{$x$}
  \If{$p \neq NIL \land k < $ \Call{Key}{$p$}}
    \State \Call{Cut}{$H, x$}
    \State \Call{Cascading-Cut}{$H, p$}
  \EndIf
  \If{$k <$ \Call{Key}{$T_{min}(H)$}}
    \State $T_{min}(H) \gets x$
  \EndIf
\EndFunction
\end{algorithmic}

Where function \textproc{Cascading-Cut} uses the mark to determine
if the node is losing the second child. the node is marked after
it losses the first child. And the mark is cleared in \textproc{Cut}
function.

\begin{algorithmic}[1]
\Function{Cut}{$H, x$}
  \State $p \gets $ \Call{Parent}{$x$}
  \State remove $x$ from $p$
  \State \Call{Degree}{$p$} $\gets$ \Call{Degree}{$p$} - 1
  \State add $x$ to root list of $H$
  \State \Call{Parent}{$x$} $\gets NIL$
  \State \Call{Mark}{$x$} $\gets FALSE$
\EndFunction
\end{algorithmic}

During cascading cut process, if $x$ is marked, which means it has
already lost one child. We recursively performs cut and cascading cut
on its parent till reach to root.

\begin{algorithmic}[1]
\Function{Cascading-Cut}{$H, x$}
  \State $p \gets $ \Call{Parent}{$x$}
  \If{$p \neq NIL$}
    \If{\Call{Mark}{$x$} $= FALSE$}
      \State \Call{Mark}{$x$} $\gets TRUE$
    \Else
      \State \Call{Cut}{$H, x$}
      \State \Call{Cascading-Cut}{$H, p$}
    \EndIf
  \EndIf
\EndFunction
\end{algorithmic}

The relevant ANSI C decreasing key program is given as the following.

\lstset{language=C}
\begin{lstlisting}
void decrease_key(struct FibHeap* h, struct node* x, Key k){
  struct node* p = x->parent;
  x->key = k;
  if(p && k < p->key){
    cut(h, x);
    cascading_cut(h, p);
  }
  if(k < h->minTr->key)
    h->minTr = x;
}

void cut(struct FibHeap* h, struct node* x){
  struct node* p = x->parent;
  p->children = remove_node(p->children, x);
  p->degree--;
  h->roots = append(h->roots, x);
  x->parent = NULL;
  x->mark = 0;
}

void cascading_cut(struct FibHeap* h, struct node* x){
  struct node* p = x->parent;
  if(p){
    if(!x->mark)
      x->mark = 1;
    else{
      cut(h, x);
      cascading_cut(h, p);
    }
  }
}
\end{lstlisting}

\begin{Exercise}
Prove that \textproc{Decrease-Key} algorithm is amortized $O(1)$ time.
\end{Exercise}

\subsection{The name of Fibonacci Heap}
It's time to reveal the reason why the data structure is named
as 'Fibonacci Heap'.

There is only one undefined algorithm so far, \textproc{Max-Degree}($N$).
Which can determine the upper bound of degree for any node in a $N$ nodes
Fibonacci Heap. We'll give the proof by using Fibonacci series and
finally realize \textproc{Max-Degree} algorithm.

\begin{lemma}
\label{lemma:Fib-degree}
For any node $x$ in a Fibonacci Heap, denote $k = degree(x)$, and
$|x| = size(x)$, then
\be
  |x| \geq F_{k+2}
\ee

Where $F_k$ is Fibonacci series defined as the following.
\[
F_k = \left \{
  \begin{array}
  {r@{\quad:\quad}l}
  0 & k = 0 \\
  1 & k = 1 \\
  F_{k-1} + F_{k-2} & k \geq 2
  \end{array}
\right.
\]
\end{lemma}

\begin{proof}
Consider all $k$ children of node $x$, we denote them as $y_1, y_2, ..., y_k$
in the order of time when they were linked to $x$. Where $y_1$ is the
oldest, and $y_k$ is the youngest.

Obviously, $y_i \geq 0$. When we link $y_i$ to $x$, children $y_1, y_2, ..., y_{i-1}$ have already been there. And algorithm \textproc{LINK} only links
nodes with the same degree. Which indicates at that time, we have

\[
  degree(y_i) = degree(x) = i - 1
\]

After that, node $y_i$ can at most
lost 1 child, (due to the decreasing key operation) otherwise, if it
will be immediately cut off and append to root list after the second
child loss. Thus we conclude

\[
degree(y_i) \geq i-2
\]

For any $i = 2, 3, ..., k$.

Let $s_k$ be the {\em minimum possible size} of node $x$, where
$degree(x) = k$. For trivial cases, $s_0 = 1$, $s_1 = 2$, and we have

\bean
|x| & \geq & s_k \\
    & =   & 2 + \sum_{i=2}^{k} s_{degree(y_i)} \qquad \\
    & \geq & 2 + \sum_{i=2}^{k} s_{i-2}
\eean

We next show that $s_k > F_{k+2}$. This can be proved by induction.
For trivial cases, we have $s_0 = 1 \geq F_2 = 1$, and $s_1 = 2 \geq F_3 = 2$.
For induction case $k \geq 2$. We have

\bean
|x| & \geq & s_k \\
    & \geq & 2 + \sum_{i=2}^{k} s_{i-2} \\
    & \geq & 2 + \sum_{i=2}^{k} F_i \\
    & =    & 1 +  \sum_{i=0}^{k} F_i \\
\eean

At this point, we need prove that

\be
F_{k+2} = 1 +  \sum_{i=}^{k} F_i
\ee

This can also be proved by using induction:
\begin{itemize}
\item Trivial case, $F_2 = 1 + F_0 = 2$
\item Induction case,
\bean
  F_{k+2} & = & F_{k+1} + F_k \\
         & = & 1 + \sum_{i=0}^{k-1}F_i + F_k \\
         & = & 1 + \sum_{i=0}^{k} F_i
\eean
\end{itemize}

Summarize all above we have the final result.
\be
N \geq |x| \geq F_k+2
\ee
\end{proof}

Recall the result of AVL tree, that $F_k \geq \Phi^k$, where
$\Phi = \frac{1+\sqrt{5}}{2}$ is the golden ratio. We also proved
that pop operation is amortized $O(\lg N)$ algorithm.

Based on this result. We can define Function $MaxDegree$ as the following.

\be
  MaxDegree(N) = 1 + \lfloor \log_{\Phi} N \rfloor
\ee

The imperative \textproc{Max-Degree} algorithm can also be realized by
using Fibonacci sequences.

\begin{algorithmic}[1]
\Function{Max-Degree}{$N$}
  \State $F_0 \gets 0$
  \State $F_1 \gets 1$
  \State $k \gets 2$
  \Repeat
    \State $F_k \gets F_{k_1} + F_{k_2}$
    \State $k \gets k+1$
  \Until{$F_k < N$}
  \State \Return $k-2$
\EndFunction
\end{algorithmic}

Translate the algorithm to ANSI C given the following program.

\lstset{language=C}
\begin{lstlisting}
int max_degree(int n){
  int k, F;
  int F2 = 0;
  int F1 = 1;
  for(F=F1+F2, k=2; F<n; ++k){
    F2 = F1;
    F1 = F;
    F = F1 + F2;
  }
  return k-2;
}
\end{lstlisting}

% ================================================================
%                 Pairing Heaps
% ================================================================

\section{Pairing Heaps}
\label{pairing-heap} \index{Pairing heap}
Although Fibonacci Heaps provide excellent performance theoretically,
it is complex to realize. People find that the constant behind the
big-O is big. Actually, Fibonacci Heap is more significant in theory
than in practice.

In this section, we'll introduce another solution, Pairing heap,
which is one of the best heaps ever known in terms of performance.
Most operations including insertion, finding minimum element (top),
merging are all bounds to $O(1)$ time, while deleting minimum element (pop)
is conjectured to amortized $O(\lg N)$ time \cite{pairing-heap}
\cite{okasaki-book}. Note that this is still
a conjecture for 15 years by the time I write this chapter. Nobody has been
proven it although there are much experimental data support the
$O(\lg N)$ amortized result.

Besides that, pairing heap is simple. There exist both elegant
imperative and functional implementations.

% ================================================================
%                 Definition
% ================================================================
\subsection{Definition}
\index{Pairing heap!definition}

Both Binomial Heaps and Fibonacci Heaps are realized with forest.
While a pairing heaps is essentially a K-ary tree. The minimum element
is stored at root. All other elements are stored in sub trees.

The following Haskell program defines pairing heap.

\lstset{language=Haskell}
\begin{lstlisting}
data PHeap a = E | Node a [PHeap a]
\end{lstlisting}

This is a recursive definition, that a pairing heap is either empty
or a K-ary tree, which is consist of a root node, and a list of sub trees.

Pairing heap can also be defined in procedural languages, for example
ANSI C as below. For illustration purpose, all heaps we mentioned later
are minimum-heap, and we assume the type of key is integer \footnote{We
can parametrize the key type with C++ template, but this is beyond
our scope, please refer to the example programs along with
this book}. We use same linked-list based left-child, right-sibling
approach (aka, binary tree representation\cite{CLRS}).

\lstset{language=C}
\begin{lstlisting}
typedef int Key;

struct node{
  Key key;
  struct node *next, *children, *parent;
};
\end{lstlisting}

Note that the parent field does only make sense for decreasing key
operation, which will be explained later on. we can omit it for the
time being.


% ================================================================
%          Basic Heap operations
% ================================================================
\subsection{Basic heap operations}
In this section, we first give the merging operation for pairing
heap, which can be used to realize the insertion. Merging, insertion,
and finding the minimum element are relative trivial compare to
the extracting minimum element operation.

\subsubsection{Merge, insert, and find the minimum element (top)}
\index{Pairing heap!insert} \index{Pairing heap!top}
\index{Pairing heap!find min}
The idea of merging is similar to the linking algorithm we shown
previously for Binomial heap. When we merge two pairing heaps, there
are two cases.

\begin{itemize}
\item Trivial case, one heap is empty, we simply return the other
heap as the result;

\item Otherwise, we compare the root element of the two heaps, make
the heap with bigger root element as a new children of the other.
\end{itemize}

Let $H_1$, and $H_2$ denote the two heaps, $x$ and $y$ be the root
element of $H_1$ and $H_2$ respectively. Function $Children()$
returns the children of a K-ary tree. Function $Node()$ can
construct a K-ary tree from a root element and a list of children.

\be
merge(H_1, H_2) = \left \{
  \begin{array}
  {r@{\quad:\quad}l}
  H_1 & H_2 = \phi \\
  H_2 & H_1 = \phi \\
  Node(x, \{H_2\} \cup Children(H_1)) & x < y \\
  Node(y, \{H_1\} \cup Children(H_2)) & otherwise
  \end{array}
\right .
\ee

Where
\[
\begin{array}{l}
x = Root(H_1) \\
y = Root(H_2)
\end{array}
\]

It's obviously that merging algorithm is bound to $O(1)$ time
\footnote{Assume $\cup$ is constant time operation, this is true
for linked-list settings, including 'cons' like operation in
functional programming languages.}.
The $merge$ equation can be translated to the following Haskell program.

\lstset{language=Haskell}
\begin{lstlisting}
merge :: (Ord a) => PHeap a -> PHeap a -> PHeap a
merge h E = h
merge E h = h
merge h1@(Node x hs1) h2@(Node y hs2) =
    if x < y then Node x (h2:hs1) else Node y (h1:hs2)
\end{lstlisting}

Merge can also be realized imperatively. With left-child, right
sibling approach, we can just link the heap, which is in fact a
K-ary tree, with larger key as the first new child of the other.
This is constant time operation as described below.

\begin{algorithmic}[1]
\Function{Merge}{$H_1, H_2$}
  \If{$H_1 = $ NIL}
    \State \Return $H_2$
  \EndIf
  \If{$H_2 = $ NIL}
    \State \Return $H_1$
  \EndIf
  \If{\Call{Key}{$H_2$} $<$ \Call{Key}{$H_1$}}
    \State \Call{Exchange}{$H_1 \leftrightarrow H_2$}
  \EndIf
  \State Insert $H_2$ in front of \Call{Children}{$H_1$}
  \State \Call{Parent}{$H_2$} $\gets H_1$
  \State \Return $H_1$
\EndFunction
\end{algorithmic}

Note that we also update the parent field accordingly. The ANSI C
example program is given as the following.

\lstset{language=C}
\begin{lstlisting}
struct node* merge(struct node* h1, struct node* h2){
  if(h1 == NULL)
    return h2;
  if(h2 == NULL)
    return h1;
  if(h2->key < h1->key)
    swap(&h1, &h2);
  h2->next = h1->children;
  h1->children = h2;
  h2->parent = h1;
  h1->next = NULL; /*Break previous link if any*/
  return h1;
}
\end{lstlisting}

Where function swap() is defined in a similar way as Fibonacci Heap.

With merge defined, insertion can be realized as same as Fibonacci Heap
in Equation \ref{eq:fib-insert}. Definitely it's $O(1)$ time operation.
As the minimum element is always stored in root, finding it is trivial.

\be
top(H) = Root(H)
\ee

Same as the other two above operations, it's bound to $O(1)$ time.

\begin{Exercise}
Implement the insertion and top operation in your favorite programming
language.
\end{Exercise}

\subsubsection{Decrease key of a node}
\index{pairing heap!decrease key}
There is another trivial operation, to decrease key of a given node,
which only makes sense in imperative settings as we explained in Fibonacci
Heap section.

The solution is simple, that we can cut the node with the new smaller
key from it's parent along with all its children. Then merge it again
to the heap. The only special case is that if the given node is the
root, then we can directly set the new key without doing anything else.

The following algorithm describes this procedure for a given node $x$, with
new key $k$.

\begin{algorithmic}[1]
\Function{Decrease-Key}{$H, x, k$}
  \State \Call{Key}{$x$} $\gets k$
  \If{\Call{Parent}{$x$} $\neq$ NIL}
    \State Remove $x$ from \textproc{Children}(\Call{Parent}{$x$})
  \EndIf
  \Call{Parent}{$x$} $\gets$ NIL
  \State \Return \Call{Merge}{$H, x$}
\EndFunction
\end{algorithmic}

The following ANSI C program translates this algorithm.

\lstset{language=C}
\begin{lstlisting}
struct node* decrease_key(struct node* h, struct node* x, Key key){
  x->key = key; /* Assume key <= x->key */
  if(x->parent)
    x->parent->children = remove_node(x->parent->children, x);
  x->parent = NULL;
  return merge(h, x);
}
\end{lstlisting}

\begin{Exercise}
Implement the program of removing a node from the children of its
parent in your favorite imperative programming language. Consider
how can we ensure the overall performance of decreasing key is
O(1) time? Is left-child, right sibling approach enough?
\end{Exercise}

\subsubsection{Delete the minimum element from the heap (pop)}
\index{Pairing heap!pop} \index{Pairing heap!delete min}
Since the minimum element is always stored at root, after delete it
during popping, the rest things left are all sub-trees. These trees
can be merged to one big tree.

\be
  pop(H) = mergePairs(Children(H))
\ee

Pairing Heap uses a special approach that it merges every two sub-trees
from left to right in pair. Then
merge these paired results from right to left which forms a final
result tree. The name of `Pairing Heap' comes from the characteristic
of this pair-merging.

Figure \ref{fig:merge-pairs} and \ref{fig:merge-right} illustrate the procedure of pair-merging.

\begin{figure}[htbp]
  \centering
  \subfloat[A pairing heap before pop.]{\includegraphics[scale=0.5]{img/pairing-hp.ps}} \\
  \subfloat[After root element 2 being removed, there are 9 sub-trees left.]{\includegraphics[scale=0.5]{img/pairs.ps}} \\
  \subfloat[Merge every two trees in pair, note that there are odd number trees, so the last one needn't merge.]{\includegraphics[scale=0.5]{img/pairs-merge.ps}} \\
  \caption{Remove the root element, and merge children in pairs.} \label{fig:merge-pairs}
\end{figure}

\begin{figure}[htbp]
  \centering
  \subfloat[Merge tree with 9, and tree with root 6.]{\hspace{0.2\textwidth}\includegraphics[scale=0.5]{img/right-merge-1.ps}\hspace{0.2\textwidth}}
  \subfloat[Merge tree with root 7 to the result.]{\hspace{0.1\textwidth}\includegraphics[scale=0.5]{img/right-merge-2.ps}\hspace{0.1\textwidth}} \\
  \subfloat[Merge tree with root 3 to the result.]{\hspace{0.1\textwidth}\includegraphics[scale=0.5]{img/right-merge-3.ps}\hspace{0.1\textwidth}}
  \subfloat[Merge tree with root 4 to the result.]{\hspace{0.1\textwidth}\includegraphics[scale=0.5]{img/right-merge-4.ps}\hspace{0.1\textwidth}}
  \caption{Steps of merge from right to left.} \label{fig:merge-right}
\end{figure}

The recursive pair-merging solution is quite similar to the bottom up
merge sort\cite{okasaki-book}. Denote the children of a pairing
heap as $A$, which is a list of trees of $\{ T_1, T_2, T_3, ..., T_m\}$
for example. The $mergePairs()$ function can be given as below.

\be
mergePairs(A) = \left \{
  \begin{array}
  {r@{\quad:\quad}l}
  \Phi & A = \Phi \\
  T_1 & A = \{ T_1 \} \\
  merge(merge(T_1, T_2), mergePairs(A')) & otherwise
  \end{array}
\right .
\ee

where

\[
A' = \{ T_3, T_4, ..., T_m\}
\]

is the rest of the children without the first two trees.

The relative Haskell program of popping is given as the following.

\lstset{language=Haskell}
\begin{lstlisting}
deleteMin :: (Ord a) => PHeap a -> PHeap a
deleteMin (Node _ hs) = mergePairs hs where
    mergePairs [] = E
    mergePairs [h] = h
    mergePairs (h1:h2:hs) = merge (merge h1 h2) (mergePairs hs)
\end{lstlisting}

The popping operation can also be explained in the following
procedural algorithm.

\begin{algorithmic}[1]
\Function{Pop}{$H$}
  \State $L \gets NIL$
  \For{every 2 trees $T_x$, $T_y \in$ \Call{Children}{$H$} from left to right}
    \State Extract $x$, and $y$ from \Call{Children}{$H$}
    \State $T \gets $ \Call{Merge}{$T_x, T_y$}
    \State Insert $T$ at the beginning of $L$
  \EndFor
  \State $H \gets $ \Call{Children}{$H$} \Comment{$H$ is either $NIL$ or one tree.}
  \For{$\forall T \in L$ from left to right}
    \State $H \gets $ \Call{Merge}{$H, T$}
  \EndFor
  \State \Return $H$
\EndFunction
\end{algorithmic}

Note that $L$ is initialized as an empty linked-list, then the algorithm
iterates every two trees in pair in the children of the K-ary tree, from
left to right, and performs merging, the result is inserted at the beginning
of $L$. Because we insert to front end, so when we traverse $L$ later on,
we actually process from right to left. There may be odd number of sub-trees
in $H$, in that case, it will leave one tree after pair-merging. We
handle it by start the right to left merging from this left tree.

Below is the ANSI C program to this algorithm.

\lstset{language=C}
\begin{lstlisting}
struct node* pop(struct node* h){
  struct node *x, *y, *lst = NULL;
  while((x = h->children) != NULL){
    if((h->children = y = x->next) != NULL)
      h->children = h->children->next;
    lst = push_front(lst, merge(x, y));
  }
  x = NULL;
  while((y = lst) != NULL){
    lst = lst->next;
    x = merge(x, y);
  }
  free(h);
  return x;
}
\end{lstlisting}

The pairing heap pop operation is conjectured to be amortized $O(\lg N)$
time \cite{pairing-heap}.

\begin{Exercise}
Write a program to insert a tree at the beginning of a linked-list
in your favorite imperative programming language.
\end{Exercise}

\subsubsection{Delete a node}
\index{pairing heap!delete}
We didn't mention delete in Binomial heap or Fibonacci Heap. Deletion
can be realized by first decreasing key to minus infinity ($-\infty$), then
performing pop. In this section, we present another solution for
delete node.

The algorithm is to define the function $delete(H, x)$, where $x$ is
a node in a pairing heap $H$ \footnote{Here the semantic of $x$ is a
reference to a node.}.

If $x$ is root, we can just perform a pop operation. Otherwise, we
can cut $x$ from $H$, perform a pop on $x$, and then merge the pop
result back to $H$. This can be described as the following.

\be
delete(H, x) = \left \{
  \begin{array}
  {r@{\quad:\quad}l}
  pop(H) & x \quad \text{is root of} \quad H \\
  merge(cut(H, x), pop(x)) & otherwise
  \end{array}
\right .
\ee

As delete algorithm uses pop, the performance is conjectured to be
amortized $O(1)$ time.

\begin{Exercise}
\begin{itemize}
\item Write procedural pseudo code for delete algorithm.

\item Write the delete operation in your favorite imperative programming
language

\item Consider how to realize delete in purely functional setting.
\end{itemize}
\end{Exercise}

% ================================================================
%                 Short summary
% ================================================================
\section{Notes and short summary}

In this chapter, we extend the heap implementation from binary tree to
more generic approach. Binomial heap and Fibonacci heap use Forest of
K-ary trees as under ground data structure, while Pairing heap use
a K-ary tree to represent heap. It's a good point to post pone some
expensive operation, so that the over all amortized performance is
ensured. Although Fibonacci Heap gives good performance in theory, the
implementation is a bit complex. It was removed in some latest textbooks.
We also present pairing heap, which is easy to realize and have good
performance in practice.

The elementary tree based data structures are all introduced in this
book. There are still many tree based data structures which we can't
covers them all and skip here. We encourage the reader to refer to
other textbooks about them. From next chapter, we'll introduce generic
sequence data structures, array and queue.

% ================================================================
%                 Appendix
% ================================================================

\begin{thebibliography}{99}

\bibitem{K-ary-tree}
K-ary tree, Wikipedia. http://en.wikipedia.org/wiki/K-ary\_tree

\bibitem{CLRS}
Thomas H. Cormen, Charles E. Leiserson, Ronald L. Rivest and Clifford Stein. ``Introduction to Algorithms, Second Edition''. The MIT Press, 2001. ISBN: 0262032937.

\bibitem{okasaki-book}
Chris Okasaki. ``Purely Functional Data Structures''. Cambridge university press, (July 1, 1999), ISBN-13: 978-0521663502

\bibitem{wiki-pascal-triangle}
Wikipedia, ``Pascal's triangle''. http://en.wikipedia.org/wiki/Pascal's\_triangle

\bibitem{hackage-fibq}
Hackage. ``An alternate implementation of a priority queue based on a Fibonacci heap.'', http://hackage.haskell.org/packages/archive/pqueue-mtl/1.0.7/doc/html/src/Data-Queue-FibQueue.html

\bibitem{okasaki-fibh}
Chris Okasaki. ``Fibonacci Heaps.'' http://darcs.haskell.org/nofib/gc/fibheaps/orig

\bibitem{pairing-heap}
Michael L. Fredman, Robert Sedgewick, Daniel D. Sleator, and Robert E. Tarjan. ``The Pairing Heap: A New Form of Self-Adjusting Heap'' Algorithmica (1986) 1: 111-129.

\end{thebibliography}

\ifx\wholebook\relax \else
\end{document}
\fi
