\documentclass{article}

\usepackage{subfig}
\usepackage{tikz}

\usepackage{amsmath}
\usepackage{listings}
\usepackage{algorithm}
\usepackage[noend]{algpseudocode} %for pseudo code, include algorithmicsx automatically

% for better Haskell code outlook
%% \lstdefinelanguage{Haskell}{
%%   basicstyle=\small\ttfamily,
%%   flexiblecolumns=false,
%%   basewidth={0.5em,0.45em},
%%   literate={+}{{$+$}}1 {/}{{$/$}}1 {*}{{$*$}}1 {=}{{$=$}}1
%%            {>}{{$>$}}1 {<}{{$<$}}1 {\\}{{$\lambda$}}1
%%            {\\\\}{{\char`\\\char`\\}}1
%%            {->}{{$\rightarrow$}}2 {>=}{{$\geq$}}2 {<-}{{$\leftarrow$}}2
%%            {<=}{{$\leq$}}2 {=>}{{$\Rightarrow$}}2
%%            {\ .}{{$\circ$}}2 {\ .\ }{{$\circ$}}2
%%            {>>}{{>>}}2 {>>=}{{>>=}}2
%%            {|}{{$\mid$}}1
%% }[keywords,comments,strings]

\lstloadlanguages{C, Haskell, Python}

\lstset{
  showstringspaces = false
}

\begin{document}

\title{The longest palindrome}
\author{Larry LIU Xinyu}
\maketitle

\section{The problem}
Palindrome is a symmetric sequence of elements. It doesn't change when gets reversed.
For example, the English word `madam' is a palindrome. If $S$ is a palindrome, we have
$S = reverse(S)$. Palindrome can be a number, a string, a piece of DNA genome,
or even music. A palindromic sub-string part of the string that is a palindrome.
For example `issi' is a palindromic sub-string in word `Mississippi'. There can be
multiple palindromic sub-strings. The problem is to find the longest palindromic
sub-string. Given `Mississippi` for example, the longest plindromic sub-string is
`ississi'.

\section{Manacher's algorithm}

There are several methods to solve this problem. The brute-force solution is to
enumerate all the sub-strings, filter the palindromic ones, and pick the longest.
There are $n(n+1)/2$ sub-strings where $n$ is the length of the string (Empty
string is ignored). Thus the performance is quadratic.

Another method is to use suffix tree. If $w$ is a palindromic sub-string of
$S$, then it must be sub-string of $reverse(S)$ also.
For example, ``issi'' is a palindromic sub-string of ``Mississippi''.
and its reversed form ``ippississiM''.

Based on this fact, we can find the longest palindrome by
searching the longest common sub-string for $S$ and $reverse(S)$.

\begin{equation}
LCS(T_{\textrm{suffix}}(S + reverse(S)))
\end{equation}

Function $LCS$ finds the
longest common sub-string in the suffix tree in linear time.

The key point is to construct the suffix tree efficiently. There are
some good algorithms achieves linear time performace, like Ukkonen's algorithm\cite{Ukkonen95}.
Generalised suffix tree can find the longest palindromic sub-string in linear time.

Suffix array provides a easier solution than suffix tree. But it downgrades
the performance to $O(n \lg n)$.

We'll explain a linear time algorithm found by Glenn K. Manacher in 1975 \cite{Manacher75}.
This method scan the string and reuse the information gained during the scan.

For given string $S = \{s_1, s_2, ..., s_n\}$, let $P = \{p_1, p_2, ... p_n\}$ be a table.
$p_i$ tells us how long we can extend from the $i$-th element in $S$ to
left and right to form a palindrome. In other words the sub-string
$\{s_{i-p_i}, s_{i-p_i+1}, ..., s_i, s_{i+1}, ..., s_{i+p_i}\}$ is the longest one at the
center of $s_i$. We donote this sub-string as $S_{(i, p_i)}$.
Below table shows an example for string `eneven'.

%\begin{table}
\begin{tabular}{|c|c|c|c|c|c|c|}
\hline
$S$ & e & n & e & v & e & n \\
\hline
$P$ & 1 & 2 & 1 & 3 & 1 & 1 \\
\hline
$S_{(i, p_i)}$ & e & ene & e & neven & e & n \\
\hline
\end{tabular}
%\end{table}

The first value $p_1 = 1$, the palindrome at the center of the first element contains
only one character ``e''. The second value $p_2 = 2$, it means we can extend from
the second element, which is `n' to 2 characters to left and right. It gives the
palindrome ``ene''. The fourth value $p_4 = 3$, we can extend 3 characters at the
center of of `v' to both sides to get the palindrome ``neven''. We can view
$p_i$ as the `radius' of the palindrome at center $i$.


\section{The brute-force scan}

\begin{thebibliography}{99}

\bibitem{Ukkonen95}
Esko Ukkonen. ``On-line construction of suffix trees''. Algorithmica 14 (3): 249--260. doi:10.1007/BF01206331. http://www.cs.helsinki.fi/u/ukkonen/SuffixT1withFigs.pdf

\bibitem{Manacher75}
Manacher, Glenn (1975), ``A new linear-time `on-line' algorithm for finding the smallest initial palindrome of a string'', Journal of the ACM 22 (3): 346�C351, doi:10.1145/321892.321896

\bibitem{wiki-longest-palindrome}
Longest palindromic substring. Wikipedia. http://en.wikipedia.org/wiki/Longest\_palindromic\_substring

\end{thebibliography}

\end{document}
