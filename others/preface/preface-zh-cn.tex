\ifx\wholebook\relax \else
% ------------------------

% \documentclass[UTF8]{ctexart}
\documentclass[UTF8]{article}

%
% loading packages
%

\RequirePackage{ifpdf}
\RequirePackage{ifxetex}

%
%
\ifpdf
  \RequirePackage[pdftex,%
       bookmarksnumbered,%
              colorlinks,%
          linkcolor=blue,%
              hyperindex,%
        plainpages=false,%
       pdfstartview=FitH]{hyperref}
\else\ifxetex
  \RequirePackage[bookmarksnumbered,%
               colorlinks,%
           linkcolor=blue,%
               hyperindex,%
         plainpages=false,%
        pdfstartview=FitH]{hyperref}
\else
  \RequirePackage[dvipdfm,%
        bookmarksnumbered,%
               colorlinks,%
           linkcolor=blue,%
               hyperindex,%
         plainpages=false,%
        pdfstartview=FitH]{hyperref}
\fi\fi
%\usepackage{hyperref}

% other packages
%--------------------------------------------------------------------------
\usepackage{graphicx, color}
\usepackage{subfig}
\usepackage{tikz}
\usetikzlibrary{matrix,positioning}

\usepackage{amsmath, amsthm, amssymb} % for math
\usepackage{exercise} % for exercise
\usepackage{import} % for nested input

%
% for programming
%
\usepackage{verbatim}
\usepackage{listings}
%\usepackage{algorithmic} %old version; we can use algorithmicx instead
%\usepackage[plain]{algorithm} %remove rule (horizontal line on top/below the algorithm
\usepackage{algorithm} %to remove rules change to \usepackage[plain]{algorithm}
%\usepackage{algorithm2e}
\usepackage[noend]{algpseudocode} %for pseudo code, include algorithmicsx automatically
\usepackage{appendix}
\usepackage{makeidx} % for index support
\usepackage{titlesec}

\usepackage[cm-default]{fontspec}
\usepackage{xunicode}
%\usepackage{fontenc}
\usepackage{textcomp}

% detect and select Chinese font
% ------------------------------
% the following cmd can list all availabe Chinese fonts in host.
% fc-list :lang=zh
\def\myfont{STSong}  % Under Mac OS X
\def\linuxfallback{WenQuanYi Micro Hei} % Under Linux
\def\winfallback{SimSun} % Under Windows
\suppressfontnotfounderror1 % Avoid setting exit code (error level) to break make process
\count255=\interactionmode
\batchmode
\font\foo="\myfont"\space at 10pt
\ifx\foo\nullfont
  \font\foo = "\linuxfallback"\space at 10pt
  \ifx\foo\nullfont
    \font\foo = "\winfallback"\space at 10pt
    \ifx\foo\nullfont
      \errorstopmode
      \errmessage{no suitable Chinese font found}
    \else
      \let\myfont=\winfallback % Windows
    \fi
  \else
    \let\myfont=\linuxfallback % Linux
  \fi
\fi
\interactionmode=\count255
\setmainfont[Mapping=tex-text]{\myfont}
\setmonofont{Monaco}   % 英文等宽字体

\XeTeXlinebreaklocale "zh"  % to solve the line breaking issue
\XeTeXlinebreakskip = 0pt plus 1pt minus 0.1pt

\titleformat{\paragraph}
{\normalfont\normalsize\bfseries}{\theparagraph}{1em}{}
\titlespacing*{\paragraph}
{0pt}{3.25ex plus 1ex minus .2ex}{1.5ex plus .2ex}

\lstdefinelanguage{Smalltalk}{
  morekeywords={self,super,true,false,nil,thisContext}, % This is overkill
  morestring=[d]',
  morecomment=[s]{"}{"},
  alsoletter={\#:},
  escapechar={!},
  literate=
    {BANG}{!}1
    {UNDERSCORE}{\_}1
    {\\st}{Smalltalk}9 % convenience -- in case \st occurs in code
    % {'}{{\textquotesingle}}1 % replaced by upquote=true in \lstset
    {_}{{$\leftarrow$}}1
    {>>>}{{\sep}}1
    {^}{{$\uparrow$}}1
    {~}{{$\sim$}}1
    {-}{{\sf -\hspace{-0.13em}-}}1  % the goal is to make - the same width as +
    %{+}{\raisebox{0.08ex}{+}}1		% and to raise + off the baseline to match -
    {-->}{{\quad$\longrightarrow$\quad}}3
	, % Don't forget the comma at the end!
  tabsize=2
}[keywords,comments,strings]

% for literate Haskell code
\lstdefinestyle{Haskell}{
  flexiblecolumns=false,
  basewidth={0.5em,0.45em},
  morecomment=[l]--,
  literate={+}{{$+$}}1 {/}{{$/$}}1 {*}{{$*$}}1 {=}{{$=$}}1
           {>}{{$>$}}1 {<}{{$<$}}1 {\\}{{$\lambda$}}1
           {\\\\}{{\char`\\\char`\\}}1
           {->}{{$\rightarrow$}}2 {>=}{{$\geq$}}2 {<-}{{$\leftarrow$}}2
           {<=}{{$\leq$}}2 {=>}{{$\Rightarrow$}}2
           {\ .}{{$\circ$}}2 {\ .\ }{{$\circ$}}2
           {>>}{{>>}}2 {>>=}{{>>=}}2
           {|}{{$\mid$}}1
}

\lstloadlanguages{C, C++, Lisp, Haskell, Python, Smalltalk}

\lstset{
  basicstyle=\small\ttfamily,
  commentstyle=\rmfamily,
  texcl=true,
  showstringspaces = false,
  upquote=true,
  flexiblecolumns=false
}

% ======================================================================

\def\BibTeX{{\rm B\kern-.05em{\sc i\kern-.025em b}\kern-.08em
    T\kern-.1667em\lower.7ex\hbox{E}\kern-.125emX}}

%
% mathematics
%
\newcommand{\be}{\begin{equation}}
\newcommand{\ee}{\end{equation}}
\newcommand{\bmat}[1]{\left( \begin{array}{#1} }
\newcommand{\emat}{\end{array} \right) }
\newcommand{\VEC}[1]{\mbox{\boldmath $#1$}}

% numbered equation array
\newcommand{\bea}{\begin{eqnarray}}
\newcommand{\eea}{\end{eqnarray}}

% equation array not numbered
\newcommand{\bean}{\begin{eqnarray*}}
\newcommand{\eean}{\end{eqnarray*}}

\newtheorem{theorem}{Theorem}[section]
\newtheorem{lemma}[theorem]{引理}
\newtheorem{proposition}[theorem]{Proposition}
\newtheorem{corollary}[theorem]{Corollary}

% 中文书籍设置
% ====================================
\renewcommand\contentsname{目\ 录}
%\renewcommand\listfigurename{插图目录}
%\renewcommand\listtablename{表格目录}
\renewcommand\figurename{图}
\renewcommand\tablename{表}
\renewcommand\proofname{证明}
\renewcommand\ExerciseName{练习}
%\renewcommand{\algorithmcfname}{算法}

\ifx\wholebook\relax
\renewcommand\bibname{参\ 考\ 文\ 献}                    %book类型
%\newtheorem{Definition}[Theorem]{定义}
\newtheorem{Theorem}{定理}[chapter]
\newtheorem{example}{例题}[chapter]
\else
\renewcommand\refname{参\ 考\ 文\ 献}
\fi

%\setcounter{secnumdepth}{4}
\titleformat{\chapter}
  {\normalfont\bfseries\Large}
  {第\arabic{chapter}章}
  {12pt}{\Large}
%% \titleformat{\subsection}
%%   {\normalfont\bfseries\large}
%%   {\CJKnumber{\arabic{subsection}}、}
%%   {12pt}{\large}
%% \titleformat{\subsubsection}
%%   {\normalfont\bfseries\normalsize}
%%   {\arabic{subsubsection}.}
%%   {12pt}{\normalsize}

%\renewcommand{\baselinestretch}{1.5}                        %文章行间距为1.5倍。

\setcounter{tocdepth}{4}
\setcounter{secnumdepth}{4}


\setcounter{page}{1}

\begin{document}

%--------------------------

% ================================================================
%                 COVER PAGE
% ================================================================

\title{前言}

\author{刘新宇
\thanks{{\bfseries 刘新宇} \newline
  Email: liuxinyu95@gmail.com \newline}
  }

\maketitle
\fi

\markboth{前言}{初等算法}

% ================================================================
%                 Why
% ================================================================
\section{算法有用么?}
\label{why}

“算法有用么?”经常有人问我这个问题。很多人在工作中根本不用算法。偶尔碰到的时候,也不过是使用一些实现好的库。例如C++标准模版库STL中有现成的排序、查找函数;常用的数据结构如向量(vector)、队列(queue)、集合(set)也都实现好了。日常工作中了解如何使用这些库似乎就足够了。

算法在解决一些“有趣”的问题时,会扮演关键角色。但是这些问题本身的价值,却是仁者见仁、智者见智。

让我们用例子来说话吧。下面两道题目,即使是初学编程的新手,似乎也很容易解决。

% ================================================================
%      Mininum free ID problem. The power of algorithms
% ================================================================
\section{最小可用ID,算法的威力}
\label{min-free} \index{最小可用数}

这道题目来自Richard Bird书中的第一章\cite{fp-pearls}。现代社会中,有很多服务依赖一种被称为ID的概念。例如身份证就是一种ID,银行账户也是一种ID,电话号码本质上也是一种ID。假设我们使用非负整数作为某个系统的的ID,所有用户都由一个ID唯一确定。任何时间,这个系统中有些ID处在使用中的状态,有些ID则可以用于分配给新用户。现在的问题是,怎样才能找到最小的可分配ID呢?例如下面的列表记录了当前正在被使用的ID:

\begin{verbatim}
[18, 4, 8, 9, 16, 1, 14, 7, 19, 3, 0, 5, 2, 11, 6]
\end{verbatim}

最小可分配的ID,也就是不在这个列表中的最小整数是10。这个题目看上去是如此简单,我们可以立即写出下面解法:

\begin{algorithmic}[1]
\Function{Min-Free}{$A$}
  \State $x \gets 0$
  \Loop
    \If{$x \notin A$}
      \State \Return $x$
    \Else
      \State $x \gets x + 1$
    \EndIf
  \EndLoop
\EndFunction
\end{algorithmic}

其中符号$\notin$的实现如下:

\begin{algorithmic}[1]
\Function{`$\notin$'}{$x, X$}
  \For{$i \gets 1 $ to $|X|$}
    \If{$x = X[i]$}
      \State \Return False
    \EndIf
  \EndFor
  \State \Return True
\EndFunction
\end{algorithmic}

有些编程语言内置了这一线性查找的实现,例如Python。我们可以直接将这一解法翻译成下面的程序。

\lstset{language=Python}
\begin{lstlisting}
def brute_force(lst):
    i = 0
    while True:
        if i not in lst:
            return i
        i = i + 1
\end{lstlisting}

但是这道题目仅仅是看上去简单。在一个存储了几百万个ID的大型系统中,这个方法的的性能很差。对于一个长度为n的ID列表,它需要$O(n^2)$的时间才能找到最小可分配的ID。在我的计算机上(双核2.10GHz处理器,2G内存),使用这一方法的C语言程序平均要5.4秒才能在十万个ID中找到答案。当ID的数量上升到一百万时,平均用时则长达8分钟。

\subsection{改进一}
改进这一解法的关键基于这一事实:对于任何$n$个非负整数$x_1, x_2, ..., x_n$,如果存在小于$n$的可用整数,必然存在某个$x_i$不在$[0, n)$这个范围内。否则这些整数一定是$0, 1, ..., n-1$的某个排列,这种情况下,最小的可用整数是$n$。于是我们有如下结论:

\be
minfree(x_1, x_2, ..., x_n) \leq n
\label{eq:min-free}
\ee

根据这一结论,我们可以用一个长度为$n+1$的数组,来标记区间$[0, n]$内的某个整数是否可用。

\begin{algorithmic}[1]
\Function{Min-Free}{$A$}
  \State $F \gets [False, False, ..., False]$ where $|F| = n+1$
  \For{$\forall x \in A$}
    \If{$x < n$}
      \State $F[x] \gets$ True
    \EndIf
  \EndFor
  \For{$i \gets [0, n]$}
    \If{$F[i] =$ False}
      \State \Return $i$
    \EndIf
  \EndFor
\EndFunction
\end{algorithmic}

其中第2行将标志数组中的所有值初始化为False,这一步骤需要$O(n)$的时间。接着我们遍历$A$中的所有元素,只要小于$n$,就将相应的标记置为True。这一过程也需要$O(n)$的时间。最后我们线性查找标志数组中第一个值为Fasle的位置。整个算法的性能是线性时间$O(n)$的。注意,我们使用了$n+1$个标志,而不是$n$个标志。这样无需额外处理,就可以应对$sorted(A) = [0, 1, 2, ..., n-1]$
的特殊情况。

虽然这个方法只需要线性时间,但是它需要$O(n)$的空间来存储标志。

这一方法比之前的暴力解法快很多。在我的计算机上,相应的Python程序平均只用0.02秒,就可以在十万个整数中找到答案。

我们还可以继续优化。每次查找,我们都要申请长度为$n+1$的数组;查找结束后,这个数组又被释放掉了。反复的申请和释放会占用不少时间。我们可以预先准备好足够长的数组,然后每次查找都复用它。另外,我们可以使用二进制的位来保存标志,这样能节约不少空间。下面的C语言程序实现了这两点小改进。

\lstset{language=C}
\begin{lstlisting}
#define N 1000000 // 1 million
#define WORD_LENGTH sizeof(int) * 8

void setbit(unsigned int* bits, unsigned int i) {
    bits[i / WORD_LENGTH] |= 1<<(i % WORD_LENGTH);
}

int testbit(unsigned int* bits, unsigned int i) {
    return bits[i/WORD_LENGTH] & (1<<(i % WORD_LENGTH));
}

unsigned int bits[N/WORD_LENGTH+1];

int min_free(int* xs, int n) {
    int i, len = N/WORD_LENGTH+1;
    for(i=0; i<len; ++i)
        bits[i]=0;
    for(i=0; i<n; ++i)
        if(xs[i]<n)
            setbit(bits, xs[i]);
    for(i=0; i<=n; ++i)
        if(!testbit(bits, i))
            return i;
}
\end{lstlisting}

在我的计算机上,这段C程序处理一百万个整数,平均用时仅仅0.023秒。最后一个for循环还能
进一步改进如下,但这些都是一些微调了。

\begin{lstlisting}
for(i=0; ; ++i)
    if(~bits[i] !=0)
        for(j=0; ; ++j)
	        if(!testbit(bits, i*WORD_LENGTH+j))
	            return i*WORD_LENGTH+j;
\end{lstlisting}

\subsection{改进二、分而治之}
我们在速度上的改进是以空间上的消耗为代价的。由于维护了一个长度为$n$的标志数组,当$n$很大时,空间上的性能就成了新的瓶颈。

分而治之的典型策略是将问题分解为若干规模较小的子问题,然后逐步解决它们以得到最终的结果。

我们可以将所有满足$x_i \leq \lfloor n/2 \rfloor$的整数放入一个子序列$A'$;将剩余的其他整数放入另外一个序列$A''$。根据公式\ref{eq:min-free},如果序列$A'$的长度正好是$\lfloor n/2 \rfloor$,这说明前一半的整数已经“满了”,最小的可用整数一定可以在$A''$中递归地找到。否则,最小的可用整数可以在$A'$中找到。总之,通过这一划分,问题的规模减小了。

需要注意的是,当我们在子序列$A''$中递归查找时,边界情况发生了一些变化,我们不再是从0开始寻找最小可用整数,查找的下界变成了$\lfloor n/2 \rfloor + 1$。因此我们的算法应定义为$minfree(A, l, u)$,其中$l$是上界,$u$是下界。

递归结束的边界条件是当待查找的序列变为空的时候,此时我们只需要返回下界作为结果即可。

根据上述思路,分而治之的解法可以形式化地定义为一个函数:

\[
minfree(A) = search(A, 0, |A|-1)
\]

\[
search(A, l, u) = \left \{
       \begin{array}
       {r@{\quad:\quad}l}
       l & A = \phi \\
       search(A'', m+1, u) &  |A'| = m - l + 1 \\
       search(A',  l, m) & otherwise
       \end{array}
\right.
\]

其中
%TODO: Add reference to appendix for notation.

\[ \begin{array}{l}
m = \displaystyle \lfloor \frac{l+u}{2} \rfloor \\
A'  = \{ \forall x \in A \wedge x \leq m \} \\
A'' = \{ \forall x \in A \wedge x > m \} \\
\end{array} \]

这一方法并不需要额外的空间\footnote{有人认为需要$O(\lg n)$的栈空间来做递归调用的簿记(book-keeping)。我们稍后会看到,这一调用实际上是尾递归,有些编译器,例如gcc可以通过-O2选项消除递归。我们也可以手工将递归转换为迭代。}。每次调用需要进行$O(|A|)$次比较来划分出子序列$A'$和$A''$。之后,问题的规模减半,所以这个算法用时为$T(n) = T(n/2) + O(n)$,化简可知其结果为$O(n)$。我们也可以这样分析其复杂度:第一次需要$O(n)$次比较来划分子序列$A'$和$A''$,第二次仅需要比较$O(n/2)$次,第三次需要比较$O(n/4)$次……总时间为$O(n + n/2 + n/4 + ...) = O(2n) = O(n)$。

在有些函数式编程语言,例如Haskell中,划分一个序列已经被作为库函数提供了。下面的例子代码实现了分而治之的算法。

%\lstset{language=Haskell}
\begin{lstlisting}[style=Haskell]
import Data.List

minFree xs = bsearch xs 0 (length xs - 1)

bsearch xs l u | xs == [] = l
               | length as == m - l + 1 = bsearch bs (m+1) u
               | otherwise = bsearch as l m
    where
      m = (l + u) `div` 2
      (as, bs) = partition (<=m) xs
\end{lstlisting}
\lstset{}

\subsection{简洁与性能――鱼和熊掌}
使用命令式编程语言的读者可能会担心这种实现的性能。对于最小的可分配ID问题,递归的深度为$O(\lg n)$,于是调用栈的大小也是$O(\lg n)$。因此空间复杂度并不能被忽略。实际上,我们可以通过将递归转换为迭代来避免空间上的占用\footnote{由于我们的函数是尾递归形式,大多数函数式编程语言会自动转换优化尾递归函数。},如下面的C语言例子程序。

\lstset{language=C}
\begin{lstlisting}
int min_free(int* xs, int n) {
    int l=0;
    int u=n-1;
    while(n) {
        int m = (l + u) / 2;
        int right, left = 0;
        for(right = 0; right < n; ++ right)
            if(xs[right] <= m) {
                swap(xs[left], xs[right]);
                ++left;
            }
        if(left == m - l + 1) {
            xs = xs + left;
            n  = n - left;
            l  = m+1;
        } else {
            n = left;
            u = m;
        }
    }
    return l;
}
\end{lstlisting}

这段程序使用了类似“快速排序”中的分割方法将数组中的元素分成两部份。所有\texttt{left}之前的元素都不大于\texttt{m},而所有\texttt{left}和\texttt{right}之间的元素都大于\texttt{m},如图\ref{fig:divide}所示。

\begin{figure}[htbp]
  \centering
  \includegraphics[scale=1]{img/divide-by-m.ps}
  \caption{数组划分的过程。所有位于$0 \leq i < left$的元素满足$x[i] \leq m$,所有位于$left \leq i < right$的元素满足$x[i] > m$,剩余的元素尚未处理。} \label{fig:divide}
\end{figure}

这一程序运行快速并且不需要额外的栈空间。但是和前面的Haskell程序比起来,并不那么直观、简洁,需要仔细阅读。有时我们需要在简洁与性能之间进行平衡。

\section{丑数――数据结构的威力}

如果说最小可用ID问题还有一些应用价值,那么接下来这个问题就纯粹是为了“有趣”了。我们要寻找第1500个“丑数”。所谓丑数,就是只含有2、3或5这三个因子的自然数。前三个丑数按照定义分别是2、3和5。数字$60 = 2^23^15^1$是第25个丑数。数字$21 = 2^03^17^1$由于含有因子7,所以不是丑数。前10个丑数如下表:

2,3,4,5,6,8,9,10,12,15

如果我们认为$1=2^03^05^0$也是一个合法的丑数,则1就是第一个丑数。

\subsection{暴力解法}

这道题目看起来并不复杂,我们可以从1开始,逐一检查所有自然数,对于每个整数,我们用除法把所有的2、3和5的因子都去掉,如果结果是1,则找到了一个丑数,当遇到第$n=1500$个丑数时就找到答案了。

\begin{algorithmic}[1]
\Function{Get-Number}{$n$}
  \State $x \gets 1$
  \State $i \gets 0$
  \Loop
    \If{\Call{Valid?}{$x$}}
      \State $i \gets i + 1$
      \If{$i = n$}
        \State \Return $x$
      \EndIf
    \EndIf
    \State $x \gets x + 1$
  \EndLoop
\EndFunction
\Statex
\Function{Valid?}{$x$}
  \While{$x \bmod 2 = 0$}
    \State $x \gets x / 2$
  \EndWhile
  \While{$x \bmod 3 = 0$}
    \State $x \gets x / 3$
  \EndWhile
  \While{$x \bmod 5 = 0$}
    \State $x \gets x / 5$
  \EndWhile
  \If{$x = 1$}
    \State \Return $True$
  \Else
    \State \Return $False$
  \EndIf
\EndFunction
\end{algorithmic}

这一暴力解法对于较小的$n$没有问题。但是根据这个方法编写的C语言程序,在我的计算机上耗时40.39秒才找到了第1500个丑数(859963392)。当试图求第15000个丑数时,程序运行了10分钟也没能找到答案,我只好把它强行停止。

\subsection{改进一、构造性解法}
在上面的暴力解法中,取模运算和除法运算很耗时\cite{Bentley}。并且这些运算被循环执行了很多次。我们可以转换一下思路,不再检查一个数是否仅含有是2、3或5的因子,而是从这三个因子中构造需要的整数。

我们从1开始,分别乘以2或3或5来生成整数。这样问题就变成如何依次生成丑数。我们可以使用队列这种数据结构来解决这个问题。

队列从一侧放入元素,然后从另一侧取出元素。所以先放入的元素会先被取出。这一特性被称为先进先出FIFO(First-In-First-Out)。

我们的思路是先把1作为唯一的元素放入队列,然后我们不断从队列另一侧取出元素,分别乘以2、3和5,这样就得到了3个新的元素。然后把它们按照大小顺序放入队列。注意,这样产生的整数有可能已经在队列中存在了。这种情况下,我们需要丢弃重复产生的元素。另外新产生的整数还有可能小于队列尾部的某些元素,所以我们在插入时,需要保持它们在队列中的大小顺序。图\ref{fig:queues}描述了这一思路的步骤。

根据这一思路的算法实现如下:

%\begin{algorithm}
\begin{algorithmic}[1]
\Function{Get-Number}{$n$}
  \State $Q \gets NIL$
  \State \Call{Enqueue}{$Q, 1$}
  \While{$n > 0$}
    \State $x \gets$ \Call{Dequeue}{$Q$}
    \State \Call{Unique-Enqueue}{$Q, 2x$}
    \State \Call{Unique-Enqueue}{$Q, 3x$}
    \State \Call{Unique-Enqueue}{$Q, 5x$}
    \State $n \gets n-1$
  \EndWhile
  \State \Return $x$
\EndFunction
\Statex
\Function{Unique-Enqueue}{$Q, x$}
  \State $i \gets 0$
  \While{$i < |Q| \wedge Q[i] < x$}
    \State $i \gets i + 1$
  \EndWhile
  \If{$i < |Q| \wedge x = Q[i]$}
    \State \Return
  \EndIf
  \State \Call{Insert}{$Q, i, x$}
\EndFunction
\end{algorithmic}
%\end{algorithm}

\begin{figure}[htbp]
  \renewcommand*\thesubfigure{\arabic{subfigure}}
  \centering
  \subfloat[初始状态,队列仅含有唯一的元素1]{\includegraphics[scale=0.5]{img/q1.ps}} \hspace{.1\textwidth}
  \subfloat[新产生的元素2、3和5加入队列]{\includegraphics[scale=0.5]{img/q2.ps}} \\
  \subfloat[新产生的元素4、6和10按照顺序被插入队列]{\includegraphics[scale=0.5]{img/q3.ps}} \hspace{.1\textwidth}
  \subfloat[新产生的元素9和15加入队列,重复元素6被丢弃]{\includegraphics[scale=0.5]{img/q4.ps}}
  \caption{使用队列依次生成丑数的前4个步骤} \label{fig:queues}
\end{figure}

在将元素插入队列时,算法需要$O(|Q|)$时间找到合适位置。如果已经存在,则直接返回。

粗略估计,队列的长度会随着$n$增加(每取出一个元素会插入最多三个新元素,增加的比率$\leq 2$),所以总运行时间为$O(1+2+3+...+n) = O(n^2)$。

图\ref{fig:big-O-1q}的数据显示了队列的访问次数和$n$之间的关系,这些点连成了二次曲线,反映了算法的复杂度是$O(n^2)$。

\begin{figure}[htbp]
  \centering
  \includegraphics[scale=0.5]{img/big-O-1q.eps}
  \caption{队列访问次数和$n$的关系} \label{fig:big-O-1q}
\end{figure}

依照此方法实现的C语言程序仅用时0.016秒就输出了正确答案859963392,比暴力解法快了2500倍。

%% Functional 1Q solution
这一解法也可以用递归的方式给出,令$X$为所有仅含有因子2、3或5的整数的无穷序列。下面的等式给出了一个有趣的关系。

\be
  X = \{1\} \cup \{2x: \forall x \in X\} \cup \{3x: \forall x \in X \} \cup \{5x: \forall x \in X \}
\ee

其中符号$\cup$表示去除重复并保持大小顺序。若$X=\{x_1, x_2, x_3...\}$,$Y=\{y_1, y_2, y_3, ...\}$,$X' = \{x_2, x_3, ...\}$,$Y'=\{y_2, y_3, ...\}$,我们可以定义$\cup$如下:

\[
X \cup Y = \left \{
  \begin{array}{r@{\quad:\quad}l}
  X & Y = \phi \\
  Y & X = \phi \\
  \{ x_1, X' \cup Y \} & x_1 < y_1 \\
  \{ x_1, X' \cup Y' \} & x_1 = y_1 \\
  \{ y_1, X \cup Y' \} & x_1 > y_1
  \end{array}
\right.
\]

在支持惰性求值的函数式编程语言,例如Haskell中,上述无穷序列及函数可以定义为如下代码:

\begin{lstlisting}[style=Haskell]
ns = 1:merge (map (*2) ns) (merge (map (*3) ns) (map (*5) ns))

merge [] l = l
merge l [] = l
merge (x:xs) (y:ys) | x <y = x : merge xs (y:ys)
                    | x ==y = x : merge xs ys
                    | otherwise = y : merge (x:xs) ys
\end{lstlisting}

通过求\texttt{ns !! (n-1)},我们可以得到第1500个丑数:

\begin{verbatim}
>ns !! (1500-1)
859963392
\end{verbatim}

\subsection{改进二、使用多个队列}
上面的解法虽然比暴力法快了很多,但是仍然有一些不足。它会产生很多的重复的元素,并且最终都被丢弃了。其次,它需要扫描队列以保证队列中的元素有序。因此入队操作从常数时间$O(1)$退化为线性时间$O(|Q|)$。

我们可以用三个队列来进行改进。这三个队列表示为$Q_2$,$Q_3$和$Q_5$。它们初始化为$Q_2=\{ 2 \}$,$Q_3 = \{ 3\}$和$Q_5 = \{ 5 \}$。我们每次从这三个队列的头部选择最小的一个元素$x$取出,然后进行下面的检查:

\begin{itemize}
\item 如果$x$是从$Q_2$取出的,我们将$2x$加入$Q_2$,$3x$加入$Q_3$,$5x$加入$Q_5$。
\item 如果$x$是从$Q_3$取出的,我们只将$3x$加入$Q_3$,$5x$加入$Q_5$,而不需要将$2x$加入$Q_2$。这是因为$2x$已经在$Q_3$中了。
\item 如果$x$是从$Q_5$取出的,我们只将$5x$加入$Q_5$,而不需要处理$2x$和$3x$了。
\end{itemize}

我们不断从这三个队列中取出最小的,直到取出第$n$个元素。图\ref{fig:q235}给出了构造丑数的前4步。

\begin{figure}[htbp]
  \renewcommand*\thesubfigure{\arabic{subfigure}}
  \centering
  \subfloat[初始状态,2、3和5作为三个队列的唯一元素。新元素4、6和10被分别加入三个队列。]{\includegraphics[scale=0.5]{img/q235-1.ps}}
  \subfloat[新元素9和15被加入队列。]{\includegraphics[scale=0.5]{img/q235-2.ps}} \\
  \subfloat[新元素8、12和20被加入队列。]{\includegraphics[scale=0.5]{img/q235-3.ps}} \\
  \subfloat[新元素25被加入队列。]{\includegraphics[scale=0.5]{img/q235-4.ps}}
  \caption{使用三个队列$Q_2$、$Q_3$和$Q_5$来构造丑数的前4步}
  \label{fig:q235}
\end{figure}

按照这个思路,算法可以实现如下。

\begin{algorithmic}[1]
\Function{Get-Number}{$n$}
  \If{$n = 1$}
    \State \Return $1$
  \Else
    \State $Q_2 \gets \{ 2 \}$
    \State $Q_3 \gets \{ 3 \}$
    \State $Q_5 \gets \{ 5 \}$
    \While{$n > 1$}
      \State $x \gets min($\Call{Head}{$Q_2$}, \Call{Head}{$Q_3$}, \Call{Head}{$Q_5$}$)$
      \If{$x = $ \Call{Head}{$Q_2$}}
        \State \Call{Dequeue}{$Q_2$}
        \State \Call{Enqueue}{$Q_2, 2x$}
        \State \Call{Enqueue}{$Q_3, 3x$}
        \State \Call{Enqueue}{$Q_5, 5x$}
      \ElsIf{$x=$ \Call{Head}{$Q_3$}}
        \State \Call{Dequeue}{$Q_3$}
        \State \Call{Enqueue}{$Q_3, 3x$}
        \State \Call{Enqueue}{$Q_5, 5x$}
      \Else
        \State \Call{Dequeue}{$Q_5$}
        \State \Call{Enqueue}{$Q_5, 5x$}
      \EndIf
      \State $n \gets n - 1$
    \EndWhile
    \State \Return $x$
  \EndIf
\EndFunction
\end{algorithmic}

算法循环$n$次,每次循环,它从三个队列中取出最小的一个元素,这一步需要常数时间。接着它根据取出元素所在的队列,产生一到三个新元素放入队列,这一步也是常数时间。因此整个算法是$O(n)$的。按照此算法实现的C++程序如下,它仅用了不到1$\mu$秒就输出了第1500个丑数859963392。

\lstset{language=C++}
\begin{lstlisting}
typedef unsigned long Integer;

Integer get_number(int n) {
    if(n==1)
        return 1;
    queue<Integer> Q2, Q3, Q5;
    Q2.push(2);
    Q3.push(3);
    Q5.push(5);
    Integer x;
    while(n-- > 1) {
        x = min(min(Q2.front(), Q3.front()), Q5.front());
        if(x==Q2.front()) {
            Q2.pop();
            Q2.push(x*2);
            Q3.push(x*3);
            Q5.push(x*5);
        } else if(x==Q3.front()) {
            Q3.pop();
            Q3.push(x*3);
            Q5.push(x*5);
        } else {
            Q5.pop();
            Q5.push(x*5);
        }
    }
    return x;
}
\end{lstlisting}

这一解法也可以用函数式的方式实现。我们定义函数$take(n)$,返回第$n$个仅由2、3或5为因子构成的整数。

\[
  take(n) = f(n, \{1\}, \{2\}, \{3\}, \{5\})
\]
其中
\[
 f(n, X, Q_2, Q_3, Q_5) = \left \{
  \begin{array}{r@{\quad:\quad}l}
  X & n = 1 \\
  f(n-1, X \cup \{x\}, Q_2', Q_3', Q_5') & otherwise
  \end{array}
\right.
\]

\[
 x = min(Q_{21}, Q_{31}, Q_{51})
\]
\[
 Q_2', Q_3', Q_5' = \left \{
 \begin{array}{r@{\quad:\quad}l}
 \{Q_{22}, Q_{23}, ...\} \cup \{2x\}, Q_3 \cup \{3x\}, Q_5 \cup \{5x\} & x = Q_{21} \\
 Q_2, \{Q_{32}, Q_{33}, ...\} \cup \{3x\}, Q5 \cup \{5x\} & x = Q_{31} \\
 Q_2, Q_3, \{Q_{52}, Q_{53}, ...\} \cup \{5x\} & x = Q_{51}
 \end{array}
 \right.
\]

下面的Haskell程序实现了上面的定义。

\begin{lstlisting}[style=Haskell]
ks 1 xs _ = xs
ks n xs (q2, q3, q5) = ks (n-1) (xs++[x]) update
    where
      x = minimum $ map head [q2, q3, q5]
      update | x == head q2 = ((tail q2)++[x*2], q3++[x*3], q5++[x*5])
             | x == head q3 = (q2, (tail q3)++[x*3], q5++[x*5])
             | otherwise = (q2, q3, (tail q5)++[x*5])

takeN n = ks n [1] ([2], [3], [5])
\end{lstlisting} %$

执行\texttt{last takeN 1500}就可输出答案859963392。

% ================================================================
%                 Short summary
% ================================================================
\section{小结}
回顾这两个有趣的例题,暴力解法都捉襟见肘。对于第一题,暴力解法尚能解决较短的列表,而对于第二题,暴力解法根本行不通。

第一个例子展示了算法的力量,第二个例子展示了数据结构的重要性。有很多有趣的题目,在计算机发明之前很难解决。但是通过编程和使用计算机,我们可以用和传统方式完全不同的方法找到答案。和中小学数学课上所学的方法相比,这样的方法并没有被普遍教授。

虽然优秀的算法、数据结构和数学书籍汗牛充栋,但是对过程式的解法和函数式的解法进行对比的却寥寥无几。从上面的例子中,可以看到有时函数式解法十分简洁,并且很接近我们在数学课上所熟悉的思考方式。

本书力图同时介绍命令式和函数式的算法和数据结构。Okasaki的著作\cite{okasaki-book}中有很多函数式的数据结构可供进一步参考。关于命令式的内容可以参考一些经典的教科书\cite{CLRS}以及维基百科。本书的例子代码使用了多种编程语言,包括C、C++、Python、Haskell和Lisp方言Scheme,读者可以从 https://github.com/liuxinyu95/AlgoXY 上下载本书的全部例子代码。为了让具有不同背景的读者都容易阅读,所有算法都提供了伪代码和数学函数描述。

由于时间仓促,书中难免存在错误,欢迎广大读者和专家批评指正,提供意见和反馈。本书作者电子邮箱:liuxinyu95@gmail.com。

\section{内容组织}
在接下来的章节中,我们将先介绍基本的数据结构,此后的一些算法都会用到它们。

我们首先介绍数据结构中的“Hello world”――二叉搜索树,接下来讲解如何解决二叉树的平衡问题。然后介绍更多有趣的树,其中Trie、Patricia和后缀树可以用于文字处理,而B树则广泛应用于文件系统和数据库。

第二部份是关于堆的。我们给出一个抽象堆的定义,然后介绍使用数组和各种二叉树实现的二叉堆(Binary Heap)。接着扩展到其他的堆包括二项式堆、斐波那契堆和Pairing堆。

数组和队列通常被认为是简单的数据结构,但我们将在第三部份看到,它们实现起来并不容易。

作为基本的排序算法,我们将介绍命令式和函数式的插入排序,快速排序和归并排序等算法。

最后的部份是关于查找和搜索的,除了基本算法,我们也会介绍诸如KMP这样的文字匹配算法。

本书的附录介绍了关于链表的基本内容。

\ifx\wholebook\relax \else
\begin{thebibliography}{99}

\bibitem{fp-pearls}
Richard Bird. ``Pearls of functional algorithm design''. Cambridge University Press; 1 edition (November 1, 2010). ISBN-10: 0521513383

\bibitem{Bentley}
Jon Bentley. ``Programming Pearls(2nd Edition)''. Addison-Wesley Professional; 2 edition (October 7, 1999). ISBN-13: 978-0201657883 (中文版:《编程珠玑》)

\bibitem{okasaki-book}
Chris Okasaki. ``Purely Functional Data Structures''. Cambridge university press, (July 1, 1999), ISBN-13: 978-0521663502

\bibitem{CLRS}
Thomas H. Cormen, Charles E. Leiserson, Ronald L. Rivest and Clifford Stein. ``Introduction to Algorithms, Second Edition''. The MIT Press, 2001. ISBN: 0262032937. (中文版:《算法导论》)

\end{thebibliography}

\expandafter\enddocument
%\end{document}

\fi
